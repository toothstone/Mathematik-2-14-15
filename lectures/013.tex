\begin{beispiel*}
$u' = u \sin x + \sin^3 x, \ u(\frac{\pi}{2}) = u_0 \\
G(x) = \cos(x)
\Rightarrow u(x) = 
\left( u_0 +
\int\limits_{\frac{\pi}{2}}^x \sin^3 t e^{-\cos t} \mathrm{d}t 
\right)
e^{\cos x} \\
= \left( u_0 - \int\limits_0^{\cos x} (1-s^2) \mathrm{d}s )  \right) e^{\cos x}
\ | \ (s = $ konstant$) \\
= \left(u_0 - \left[(s^2 + 2s + 1) e^{-s} \mathrm{d}s \right]_0^{\cos x} \right)
e^{\cos x}
= u_0 e^{\cos x} - \cos^2 x - 2\cos x - 1 + e^{\cos x}
$

\end{beispiel*}

\subsection{Exakte Differentialgleichungen}

\emph{Idee:} versuche Lösungskurven von Dgl. als Niveaulinien einer Funktion 
$F(x,y) $ zu beschreiben.

\begin{beispiel*}
konzentrische Kreise $F(x,y)  = x^2 + y^2 = r^2 $ \\
sind Lösungen von Dgl. $y'(x) = - \frac{x}{y} $ (=Anstieg) für $ y \neq 0, \ 
x'(y)  = - \frac{y}{x} $ für $x \neq 0 $
\\
\textbf{Skizzen fehlen!}
\end{beispiel*}

Betrachte allgemeiner Dgl's der Form:

a) $g(x,y) + h(x,y) y' = 0 $ bzw. \\
b) $g(x,y)x + h(x,y) = 0 $ \\
Formale Schreibweise für beide Fälle:
\begin{equation}
    g(x,y) \mathrm{d}x + h(x,y) \mathrm{d}y = 0 
\end{equation}

\begin{definition}[exakt]
Eine Differenzialgleichung heißt \textbf{exakt} im Gebiet $D \subset \mathbb{R}^2 $ 
falls $(g,h) $ ein Gradientenfeld ist,
d.h. es existiert $F \in C^1(D,\mathbb{R}) $ mit \\
$F_x = g, \ F_y=h $ in $D$
$\Rightarrow F$ wird dann Stammfunktion genannt. \\
(Existenz und Berechnung von Stammfunktionen s. Kapitel 33)
\end{definition}

\begin{satz}
\mbox{}\\
Sei $g,h $ stetig im Gebiet $D \in \mathbb{R}^2$,
(4) sei exakt, $F$ die zugehörige Stammfunktion
$\Longrightarrow 
\begin{cases}
    a) y = y(x) $ ist Lösung von (4) $\Longleftrightarrow F(x,y(x)) = $ const $\\
    $b) $x = x(y) $ ist Lösung von (4) $\Longleftrightarrow F(x(y),x) = $ const$
\end{cases}
$
\end{satz}

\begin{proof}
zu a)$\ldots$
hat Frau Milbers weggewischt!
\end{proof}

Auflösung $F(x,y) = c$
nach $x$ bzw. $y$ in Umgebung von $(\tilde{x},\tilde{y}) \in D $:\\
Satz über implizite Funktionen liefert 
$F_y(\tilde{x}, \tilde{y}) = h(\tilde{x},\tilde{y}) \neq 0 \\
\Rightarrow $ Lösung $y=y(x) $ existiert lokal \\
$F_x(\tilde{x}, \tilde{y}) = h(\tilde{x},\tilde{y}) \neq 0 \\
\Rightarrow $ Lösung $x=x(y) $ existiert lokal

\begin{beispiel*}
$2xy \mathrm{d}x + x^2 \mathrm{d}y = 0 $ auf $D = \mathbb{R}^2 $
d.h $g(x,y) = 2xy, \ h(x,y) = x^2 $\\
Integrabilitätsbedingung:
$g_y = 2x, \ h_x = 2x \Rightarrow $ erfüllt \\
Berechne SF: (vgl. Kap. 33)\\
Wähle $(x_0, y_0) = (0,0) $
Kurve z.B. $\xi(s) = s(x), \ \eta(s) = sy $\\
\textbf{Skizze fehlt}\\
$\Rightarrow F(x,y) = \int\limits_0^1 2(sx)(sy)x + (sx)^2 y \mathrm{d}s \\
= x^2 y \int\limits_0^1 (2s^2 + s^2) \mathrm{d}s = x^2 y [s^3]_0^1 = x^2 y $\\
Probe: 
$F_x = g, F_y = h $\\
Löse: $F(x,y) = x^2 y = c $\\
a) $F_x = 2xy \neq 0 $ für $x,y \neq 0, \ y= \frac{c}{x^2} $\\
b) $F_y = x^2 \neq 0 $ für $ x \neq 0, \ x = \pm \sqrt{\frac{c}{y}} $ 
(nur für $ cy > 0 $

\end{beispiel*}

Hinweis: Falls (4) nicht exakt ist kann das gelegentlich
durch Multiplikation mit geeigneter Funktion $M(x)$ bzw $M(y)$ erreicht werden.\\
($M$ - integriederender Faktor)\\
$\rightarrow $ SeSt. (betr. z.B. $(2 x^2 + 2 xy^2 + 1)y \mathrm{d}x +
3(y^2 + x)  \mathrm{d}y = 0 $)

\subsection{Implizite Differentialgleichungen}
\begin{equation}
f(x,u,u') = 0, \ u(x_0) = u_0
\end{equation}

$f$ sei stetig in $D \subset \mathbb{R}^3 $\\
Problem: Dgl. (5) liefert evtl. keine eindeutiges Richtungsfeld, d.h. zu festem $(x,u)$
hat man verschiedene Anstiege $u'$\\
\textbf{Skizze fehlt!}\\
Ausweg: Fixiere "Lösungszweig" durch lokale Auflösung von $f(x,u,p) = 0 $ nach $p$\\
Satz über implizite Funktionen liefert:

\begin{lemma}
$f, f_p$ seien stetig in Umgebung von $(x_0, u_0, p_0) \in D $ und 
$f(x_0, u_0, p_0) = 0, \ f_p(x_0, u_0, p_0) \neq 0
\Rightarrow $ Gleichung $f(x, u, p) = 0 $ 
besitzt auf (kleiner) Umgebung $U$ von $(x_0, u_0) $
eine eindeutige stetige Lösung $p = \tilde{f}(x,u) $, d.h.
$f(x, u, \tilde{f}(x, u)) = 0 $ auf $U, \ \tilde{f}(x_0, u_0)$
folglich ist (5) lokal äquivalent zur expliziten Dgl. $u' = \tilde{f}(x,u) $
und kann mit bisherigen Methoden gelöst werden.

\end{lemma}

Erkenntnis: Bei Mehrdeutigkeit in (5) muss man neben AW
$u(x_0) = u_0 $ noch einen Lösungszweig auswählen.
Dies kann man i.A. durch Wahl eines der möglichen Werte von $u'(x_0) $ tun.

\begin{beispiel*}
$u'^2 - 4 x^2 = 0 $\\
\textbf{Skizze fehlt}\\
$\Rightarrow u' = 2x $ oder $u' = -2x
u(x) = x^2 + c $ oder $u(x) = -x^2 + c $\\
AWP: $u(1) = 1, \ u_1(x) = x^2, \ u_2(x) = 2 - x^2 $\\
Auswahl einer Lösung mittels:
$u'(1) = \pm 2 $
aber falls $x_0 = 0 \\
\Rightarrow f(x_0, u_0, p_0) = p_0^2 - 4 x_0^2 = 0 \Rightarrow $
nur für $p_0 = 0 $\\
wegen $f_p(x_0, u_0, p_0) = 2p_0 = 0 $ für $p_0 = 0$ Lemma (4) nicht anwendbar \\
$\Rightarrow $ AWP $u(0) = u_0 $ kann durch Zusatzforderung $u'(0) = 0 $
\emph{nicht} eindeutig gelöst werden.

\end{beispiel*}

\subsection{Potenzreihenansatz}
Problem: explizite Lösung häufig nicht möglich (insb. Anwendungen)

Ausweg: Näherungslösung durch (abgebrochene) Potenzreihe\\
(stillschweigende Annahme: Lösung $u(x)$ analytisch)
$u' = f(x,u), \ u(x_0) = u_0 $

Ansatz:
$u(x) = \sum\limits_{k=0}^{\infty} a_k (x-x_0)^k 
\Rightarrow a_k = \frac{u^{(k) x_0}}{k!} $\\
AW: $a_0 = u_0 $

2 Methoden:
\begin{itemize}
    \item[i)] Differentiation der Dgl. liefert\\
    $u'(x_0) = f(x_0, u(x_0)) \Rightarrow a_1 = \ldots \\
    u''(x_0) = f_x (x_0, u(x_0)) + f_u (x_0, u(x_0))
    \underbrace{u'(x_0}_{=f(x_0, u(x_0))}
    \Rightarrow a_2 = \ldots \\
    u'''(x_0) = f_{xx} + f_{xu} u' + f_{ux} u'^2 + f_{uu} u''
    \Rightarrow a_3 = \ldots
    $
    \item[ii)] falls $f$ analytisch ist, d.h.
    $f(x,u) = \sum\limits_{k,l = 0}^{\infty} b_{kl} (x-x_0)^k (u-u_0)^l $\\
    Dgl. liefert:
    $\sum\limits_{k=1}^{\infty} k a_k (x-x_0)^{k-1} \\
    = \sum\limits_{k,l = 0}^{\infty} b_{kl} (x-x_0)^k 
    \left( \sum\limits_{m=1}^{\infty} a_m (x-x_0)^m \right)^2 $\\
    Koeffizientenvergleich liefert Rekursionsformeln für $a_k$
\end{itemize}

\begin{beispiel*}
$u' = x^2 + u^2, \ u(0) = 1 $\\
nach i):\\
$u'|_{x=0} = x^2 + u^2 |_0 = 1 \\
u'' |_0 = 2x + 2uu' |_0 = 2 \\
u'''|_0 = 2 +2 u'^2 + 2 uu'' |_0 = 8 \\
u'''' |_0 = 4 u'u'' + 2 u'u'' + 2 u'u''' |_0 = 28 \\
\Rightarrow u(x) = 1 + x + x^2 + 4/3 x^3 + 7/6 x^4 + \ldots
$

nach ii): Dgl. \\
$\sum\limits_{k=1}^{\infty} k a_k x^{k-1} 
= x^2 + \left( \sum\limits_{k=0}^{\infty} a_k x^k \right)^2 \\
\sum\limits_{k=0}^{\infty} (k+1) a_{k+1} x^k 
= x^2 + \sum\limits_{k=0}^{\infty} x^k \sum\limits_{l=0}^k a_l a_{k-l} \\
\Rightarrow (k+1) a_{k+1} = \sum\limits_{l=0}^k a_l a_{k-l} 
\begin{cases}
    + 1 $ für $k = 2\\
    + 0 $ sonst $
\end{cases}
|k = 0,1,2,\ldots \\
$
Anfangswert: $ a_0 = 1 $
\begin{itemize}
    \item $k = 0: a_1 = a_0^2 \Rightarrow a_1 = 1 $
    \item $k = 1: 2a_2 = 2 a_0 a_1 \Rightarrow a_2 = 1 $
    \item $k = 2: 3a_3 = 2 a_0 a_2 + a_1^2 + 1 \Rightarrow a_3 = \frac{4}{3} $
    \item $k = 3: 4a_4 = 2 a_0 a_3 + 2 a_1 a_2 \Rightarrow a_4 = \frac{7}{6} $
\end{itemize}
$\Rightarrow u(x) = 1+ x + x^2 + \frac{4}{3} x^3 + \frac{7}{6} x^4 + \ldots $\\
Lokale Interpretation über die Lösung:\\
\textbf{Skizze fehlt}
\end{beispiel*}

\section{Existenztheorie und allgemeine Eigenschaften}
\emph{Motivation:}
Bewegung eines Massepunktes im Kraftfeld $f$,
gegebene Masse $m$:
Kraftfeld 
$f: \underbrace{\mathbb{R}^3}_{\text{Ort} }
\times \underbrace{\mathbb{R}}_{\text{Zeit} }
\rightarrow \underbrace{\mathbb{R}^3}_{\text{Kraftvektor}}:
f(u,t) = 
\begin{pmatrix}
    f_1(u,t) \\
    f_2(u,t) \\
    f_3(u,t) \\
\end{pmatrix}
$ \\
gesucht:
Bewegungskurve:
$u(t) = 
\begin{pmatrix}
    u_1(t) \\
    u_2(t) \\
    u_3(t) \\
\end{pmatrix}
$\\
Newtonsches Bewegungsgesetz:
Eindeutige Lösung sollte existieren falls Ort und Geschwindigkeit vorgeschrieben sind

Problematik:
\begin{itemize}
    \item Behandlung von Systemen
    \item Behandlung höherer Ableitungen
\end{itemize}
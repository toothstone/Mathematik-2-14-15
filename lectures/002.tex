\begin{sa}

(lokale Darstellung einer Mannigfaltigkeit als Graph)

Sei $M \subset  \mathbb{R}^{n}$ eine d-dimensionale $C^{q}$-Mannigfaltigkeit.

$\Longleftrightarrow \forall u \in M \subset \mathbb{R}^n $ existiert eine Umgebung
$U$ von $u$  bezüglich $M$, $W \subset \mathbb{R}^n $ offen, 
$f \in C^q (W, \mathbb{R}^{n-d})$ und eine Permutation $\pi$ von Koordinaten in
$\mathbb{R}^n $ mit $ \psi (W) = U $ für $ \psi (v) \coloneqq \psi (v, f(v)) 
\forall v \in W $ (das heißt $U$ ist Graph von f).

\textbf{somit:} $M$ ist $C^q$-Mannigfaltigkeit genau dann, wenn $M$ lokaler Graph
einer $C^q$-Funktion $f$ ist (vergleiche Beispiel 29.2 und 4).

\end{sa}

\begin{proof}

$"\Leftarrow"$: folgt aus Beispielen 29.2 und 29.4. \\
$"\Rightarrow"$: fixiere $\tilde{u} \in M $, sei $\varphi : \tilde{V} \in \mathbb{R}^d
\rightarrow \tilde{U} \subset \mathbb{R}^n $ und der zugehörige $C^q$-Parameter 
$\tilde{u} = \varphi (\tilde{x})$ \\
$\varphi' (x)$ ist regulär $\xRightarrow{\text{ evtl. $\pi$ der Zeilen }} \varphi_I ' 
(\tilde{x}) \subseteq \mathbb{R}^{d \times d} $ ist regulär für $\varphi (x) =
    \begin{pmatrix}
    \varphi_I (x) \left[ \in \mathbb{R}^d \right]\\
    \varphi_{II} (x) \left[ \in \mathbb{R}^{n-d} \right]
    \end{pmatrix}
$ \\
Zerlege $u = \pi (v,w) $ mit $v \in \mathbb{R}^d $, 
d.h. $\tilde{u} = \pi (\tilde{v},\tilde{w})$

\textbf{Skizzen fehlen!}

$\xRightarrow{\text{Theorem über inverse Fkt.}} \exists V \subset \tilde{V} $ offen,
$\tilde{x} \in V, W \subset \mathbb{R}^d $ offen, $\tilde{v} \in W $
mit $\varphi_I ^{-1} : W \rightarrow V $ Homöomorphismus und $C^q$-Abbildung,
$\varphi_I ^{-1} (\tilde{v}) = \tilde{x} $ \\
mit $f(v) \coloneqq \varphi_{II} \left(\varphi_I ^{-1} (v)\right) \forall v \in W $ ist 
$f \in C^q \left(W, \mathbb{R}^{n-d}\right) $ \\
und $\psi (v) \coloneqq \varphi \left(\varphi_I ^{-1} (v) \right) =
\left( \varphi_I \left(\varphi_I ^{-1} (v) \right( ,
\varphi_{II} \left(\varphi_I ^ {-1} (v) \right) \right) = \pi \left(v, f(v)\right) $ \\
$\Rightarrow \psi (\tilde{v}) = \pi (\tilde{v}, \tilde{w}) = \tilde{u},
\psi (w) = \varphi (v) \in M \\
\varphi : \tilde{V} \rightarrow \tilde{U} $ ist Homöomorphismus \\
$\Rightarrow \varphi (v) $ ist offen in $M$ \\
$ \Rightarrow U \coloneqq \psi (W) $ ist offen bezüglich $M$
$\Rightarrow U $ ist Umgebung von $\tilde{u} $ bezüglich $M$ \\
$\xRightarrow{\tilde{u} \text{ beliebig}} $ Behauptung. 

\end{proof}

\begin{sa}
(Charakterisierung von Mannigfaltigkeiten mit umgebenedem Raum) \\
$M \subset \mathbb{R}^n $ sei d-dimensionale $C^q$-Mannigfaltigkeit. \\
$\Longleftrightarrow \forall u \in M $ existiert eine Umgebung $\tilde{U}$ von $u$ 
bezüglich $\mathbb{R}^n$, \\
$\tilde{V} \subset \mathbb{R}^n $, 
$\tilde{\psi}: \tilde{U} \rightarrow \tilde{V} $ wobei $\tilde{\psi} $ 
ein $C^q$-Diffeomorphismus ist und 
    \begin{equation*}
    \tilde{\psi} \left( \tilde{U} \cap M \right) =
    \tilde{V} \cap \left( \mathbb{R}^d \times {0} \right) 
    \end{equation*}

\textbf{Skizzen fehlen!}

\end{sa}

\textbf{Bemerkung:} Diese Charakterisierung von Mannigfaltigkeiten benutzt den umgebenden Raum und wird häufig als Definition der Mannigfaltigkeit benutzt.    
    
\begin{proof}
$"\Leftarrow"$: $\psi $ eingeschränkt auf $\tilde{U} \cap M $ liefert Karten 
$\Rightarrow$ Behauptung. \\
$"\Rightarrow"$: fixiere $\tilde{u} \in M $, wähle $\tilde{U} \subset M $,
$W \subset \mathbb{R}^d $, $f \in C^q \left( W, \mathbb{R}^{n-d} \right)$ \\
gemäß Satz 1 oBdA $\pi = $ \textit{id} \\
zerlege $u = (v,w) \in \mathbb{R}^d \times \mathbb{R}^{n-d}$, 
$\tilde{u} = \left( \tilde{v}, f \left( \tilde{v} \right) \right) $ \\
sei $\hat{U} \coloneqq W \times \mathbb{R}^{n-d} \eqqcolon \hat{V} $,
liefert "Zylinder" aus $U$ und $W$ in Beweis zu Satz 29.6 \\
sei $\tilde{\varphi}: \hat{V} \rightarrow \hat{U} $ mit
$\tilde{\varphi} (v,w) \coloneqq (v, f(v) + w) \Rightarrow \tilde{\varphi} \in C^q $ \\
$\tilde{\varphi}' \left( \tilde{v}, 0 \right) =
    \begin{pmatrix}
    \textit{id}_d & 0 \\
    f'(v)         & \textit{id}_{n-d}
    \end{pmatrix}
$ ist regulär \\
$\xRightarrow{\text{Satz ü. inverse Fkt.}} \exists $ 
Umgebung $ \tilde{U} \subset \hat{U}$ von $\tilde{U}$,
Umgebung $ \tilde{V} \subset \hat{V} $ von $ \left( \tilde{v}, 0 \right) $, sodass \\
$\tilde{\psi} \coloneqq \tilde{\varphi}^{-1} \in C^q \left( \tilde{U}, \tilde{V} \right) $
exisitiert. \\
wegen $\tilde{\varphi} \left( \tilde{V} \cap \left( \mathbb{R}^d \times {0} \right) \right)
= \tilde{U} \cap M $ folgt die Behauptung.
\end{proof}

\begin{folgerung}
Sei $M \subset \mathbb{R}^n$ d-dimensionale $C^q$-Mannigfaltigkeit \\
und $\varphi: V \subset \mathbb{R}^d \rightarrow U \subset M $ Parameter um $u \in M $ \\
$\Longrightarrow \exists \tilde{U}, \tilde{V} \subset \mathbb{R}^n $ offen und 
$\tilde{\varphi} : \tilde{V} \rightarrow \tilde{U} $ 
mit $ U \subset \tilde{U}, V \times {0} \subset \tilde{V} $, \\
$\tilde{\varphi} $ ist $C^q$-Diffeomorphismus und
$\tilde{\varphi} (x, 0) = \varphi (x) \forall x \in V $
\end{folgerung}

\begin{proof}
Folgt aus Beweisen von Satz 29.6 und 29.7
\end{proof}

\begin{theo}
(lokale Darstellung von Mannigfaltigkeiten als Niveaumenge)\\
$M \subset \mathbb{R}^n $ sei d-dimensionale $C^q$-Mannigfaltigkeit. \\
$\Longleftrightarrow \forall u \in M $ existiert eine Umgebing $\tilde{U}$ von $u$
bezüglich $\mathbb{R}^n$ und \\
$f \in C^q \left( \tilde{U}, \mathbb{R}^{n-d} \right)$ 
mit $\textit{rang } f' (u) = n-d $ und \\
$\tilde{U} \cap M = \left\lbrace \tilde{u} \in \tilde{U} | f (\tilde{u}) = 0 \right\rbrace $
\end{theo}

\textbf{somit:} $M$ ist eine $C^q$-Mannigfaltigkeit genau dann, 
wenn $M$ die lokale Niveaumenge einer $C^q$-Funktion $f$ ist. \\
\textbf{Bemerkung:} $c \in \mathbb{R}^{n-d} $ heißt \textit{regulärer Wert} von
$f \in C^q \left( \tilde{U}, \mathbb{R}^{n-d} \right), \tilde{U} \subset \mathbb{R}^n $
offen, falls $\textit{rang } f' (u) = n-d \forall u \in \tilde{U} $ mit $f(u) = c $ \\
Folglich ist $M \coloneqq \left\lbrace u \in \tilde{U} | f(u) = c \right\rbrace $ 
eine d-dimensionale $C^q$-Mannigfaltigkeit, falls $c$ ein regulärer Wert von $f$ ist.

\begin{proof}
$"\Leftarrow":$ gemäß Bsp. 5 erhält man lokale Parametriesierung \\
$\Rightarrow$ Behauptung. \\
$"\Rightarrow":$ fixiere $\tilde{u} \in M$, 
wähle $\tilde{U}, \tilde{V} \subset \mathbb{R}^n, 
\tilde{\psi}: \tilde{U} \rightarrow \tilde{V} $ gemäß Satz 29.7 \\
sei $f \coloneqq \left( \tilde{\psi}_{d+1}, \ldots , \tilde{\psi}_n \right)$,
offenbar $f \in C^q \left(  \tilde{U}, \mathbb{R}^{n-d} \right) $ \\
mit $\tilde{\psi}$ aus dem Beweis zu Satz 29.7:
$\tilde{\psi}' (\tilde{u}) = \tilde{\varphi}' (\tilde{v}, 0)^{-1} $ ist regulär \\
$\Rightarrow f'(\tilde{u}) $ hat vollen Rang, d.h. $\textit{rang } f'(\tilde{u}) = n-d $ \\
nach Konstruktion $ \left\lbrace u \in \tilde{U} | f(u) = 0 \right\rbrace = U \cap M
\Rightarrow $ Behauptung.
\end{proof}

\begin{lem}
(Kartenwechsel) \\
Sei $M \in \mathbb{R}^n $ d-dimensionale Mannigfaltigkeit \\
und $\varphi_1^{-1}, \varphi_2^{-1} $ Karten mit zugehörigem Kartengebiet 
$U_1 \cap U_2 \neq \emptyset $ \\
$\Longrightarrow 
\varphi_2^{-1} \bullet \varphi_1 : \varphi_1^{-1} \left( U_1 \cap U_2 \right)
\rightarrow \varphi_2^{-1} \left( U_1 \cap U_2 \right) $ ist $C^q$-Diffeomorphismus.

\textbf{Skizze fehlt!}
\end{lem}

\begin{proof}
Ersetze $\varphi_1, \varphi_2 $ mit $\tilde{\varphi}_1, \tilde{\varphi}_2 $
gemäß Folgerung 29.8 \\
$\Rightarrow$ Einschränkung von $\tilde{\varphi}_2^{-1} \bullet \tilde{\varphi}_1 $
liefert Behauptung. 
\end{proof}

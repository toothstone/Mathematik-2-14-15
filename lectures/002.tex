\begin{sa}

(lokale Darstellung einer Mannigfaltigkeit als Graph)

Sei $M \subset  \mathbb{R}^{n}$ eine d-dimensionale $C^{q}$-Mannigfaltigkeit.

$\Longleftrightarrow \forall u \in M \subset \mathbb{R}^n $ existiert eine Umgebung
$U$ von $u$  bezüglich $M$, $W \subset \mathbb{R}^n $ offen, 
$f \in C^q (W, \mathbb{R}^{n-d})$ und eine Permutation $\pi$ von Koordinaten in
$\mathbb{R}^n $ mit $ \psi (W) = U $ für $ \psi (v) \coloneqq \psi (v, f(v)) 
\forall v \in W $ (das heißt $U$ ist Graph von f).

\underline{somit:} $M$ ist $C^q$-Mannigfaltigkeit genau dann, wenn $M$ lokaler Graph
einer $C^q$-Funktion $f$ ist (vergleiche Beispiel 29.2 und 4).

\end{sa}

\begin{proof}

$"\Leftarrow"$ folgt aus Beispielen 29.2 und 29.4. \\
$"\Rightarrow"$ fixiere $\tilde{u} \in M $, sei $\varphi : \tilde{V} \in \mathbb{R}^d
\rightarrow \tilde{U} \subset \mathbb{R}^n $ und der zugehörige $C^q$-Parameter 
$\tilde{u} = \varphi (\tilde{x})$ \\
$\varphi' (x)$ ist regulär $\xRightarrow{\text{ evtl. $\pi$ der Zeilen }} \varphi_I ' 
(\tilde{x}) \subseteq \mathbb{R}^{d \times d} $ ist regulär für $\varphi (x) =
    \begin{pmatrix}
    \varphi_I (x) \left[ \in \mathbb{R}^d \right]\\
    \varphi_{II} (x) \left[ \in \mathbb{R}^{n-d} \right]
    \end{pmatrix}
$ \\
Zerlege $u = \pi (v,w) $ mit $v \in \mathbb{R}^d $, 
d.h. $\tilde{u} = \pi (\tilde{v},\tilde{w})$

\textbf{Skizzen fehlen!}

$\xRightarrow{\text{Theorem über inverse Fkt.}} \exists V \subset \tilde{V} $ offen,
$\tilde{x} \in V, W \subset \mathbb{R}^d $ offen, $\tilde{v} \in W $
mit $\varphi_I ^{-1} : W \rightarrow V $ Homöomorphismus und $C^q$-Abbildung,
$\varphi_I ^{-1} (\tilde{v}) = \tilde{x} $ \\
mit $f(v) \coloneqq \varphi_{II} \left(\varphi_I ^{-1} (v)\right) \forall v \in W $ ist 
$f \in C^q \left(W, \mathbb{R}^{n-d}\right) $ \\
und $\psi (v) \coloneqq \varphi \left(\varphi_I ^{-1} (v) \right) =
\left( \varphi_I \left(\varphi_I ^{-1} (v) \right( ,
\varphi_{II} \left(\varphi_I ^ {-1} (v) \right) \right) = \pi \left(v, f(v)\right) $ \\
$\Rightarrow \psi (\tilde{v}) = \pi (\tilde{v}, \tilde{w}) = \tilde{u},
\psi (w) = \varphi (v) \in M \\
\varphi : \tilde{V} \rightarrow \tilde{U} $ ist Homöomorphismus \\
$\Rightarrow \varphi (v) $ ist offen in $M$ \\
$ \Rightarrow U \coloneqq \psi (W) $ ist offen bezüglich $M$
$\Rightarrow U $ ist Umgebung von $\tilde{u} $ bezüglich $M$ \\
$\xRightarrow{\tilde{u} \text{ beliebig}} $ Behauptung. 

\end{proof}


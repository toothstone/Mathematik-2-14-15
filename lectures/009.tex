Weiterhin ist anzumerken:\\

\begin{enumerate}
    \item
    $\rot F = \rot (F + \tilde{F}) $ falls $\tilde{F} = \div g $ 
    (d.h. $\tilde{F}$ ist Gradientenfeld).
    \item
    $\div F = \div (F + \tilde{F}) $ falls $\tilde{F} = \rot G $
    (d.h. $\tilde{F}$ ist Rotation eines Vektorfeldes).
\end{enumerate}

\begin{definition}[Kohärente Orientierung]
\mbox{}\\
Sei $M \subset \mathbb{R}^3 $ eine 2-dimensionale Mannigfaltigkeit mit dem ENF $\nu $
und sei $W \subset M $ offen bezüglich $M$, 
$\partial_M W $ bezeichne den Rand von $W$ bezüglich $M$.

\textbf{Skizze fehlt}

Ein Randpunkt $u \in \partial_M W $ heißt \textbf{regulär}, falls für ein Kartengebiet $U$
und eine zugehörige Parametrisierung $\varphi$ der Punkt $x = \varphi^{-1} (u) $
ein regulärer Randpunkt von $\varphi^{-1} (U \cap W) \subset \mathbb{R}^2 $ ist.\\
(Da ein Kartenwechsel ein Diffeomorphismus ist, ist diese Definition unabhängig vom
speziellen $\varphi$.)

$W \subset M $ hat einen \textbf{glatten Rand} $\partial_M W $ bezüglich $M$
falls alle  $u \in \partial_M W $ regulär sind.\\
In diesem Fall ist $\partial_M W $ eine 1-dimensionale Mannigfaltigkeit.\\
(Denn $\varphi$ auf $\varphi^{-1} (\partial_M W \cap U) $ 
ist die zugehörige Parametrisierung\\
und $\varphi^{-1} (\partial_M W \cap U) $ ist eine 1-dimensionale Mannigfaltigkeit.)

Somit ist $\partial_M W $ lokal als reguläre Kurve darstellbar und eine Einheitstangente
$t(u) $ exisitiert.\\
$t: \partial_M W \rightarrow \mathbb{R}^3 $ 
\textbf{orientiert} $\partial_M W $ \textbf{kohärent} zu $M$, 
falls $t(u) $ der Tangenteneinheitsvektor an $\partial_M W \ \forall u $ und
eine stetige Abbildung ist und 
$\nu(u) \times t(u) \in T_u M $ für alle $u$ 'zur Menge $W$ zeigt'.
(Man sagt auch, $W$ liegt 'links vom Rand'.)

\textbf{Skizze fehlt}

\end{definition}

\begin{satz}[Integralsatz von Stokes - klassisch im $\mathbb{R}^3 $]
\mbox{}\\
Sei $F: \Omega \subset \mathbb{R}^3 \rightarrow \mathbb{R}^3 $,
$\Omega$ offen, ein $C^1$-Vektorfeld;
sei $M \subset \Omega $ eine 2-dimensionale Mannigfaltigkeit, orientiert,
mit dem ENF $\nu$ und sei $W \subset M $ beschränkt,
mit glattem Rand $\partial_M W $ welcher mit $t$ zu $M$ kohärent orientiert ist.

\begin{equation}
    \Longrightarrow
    \int\limits_W \rot F(u) \cdot \nu(u) \mathrm{d}a
    =
    \int\limits_{\partial_M W} F(u) \cdot t(u) \mathrm{d}a
\end{equation}

\end{satz}

Das heißt das Integral über die Wirbel des Vektorfelds $F$ 'in der Fläche' $W$\\
(d.h.  $\nu \cdot \rot F $)
ist gleich der Zirkulation von $F$ entlang des Randes $\partial_M W $.

\begin{proof}

$W$ möge im Kartengebiet $U$ von $M$ liegen (sonst ZdE nötig),
die zugehörige Parametrisierung sei $\varphi: V \subset \mathbb{R}^2 \rightarrow U $,
die Koordinate $x = (x_1, x_2) $ liegt in $V$, 
$u = (u_1, u_1, u_3) $ in $M$ bzw. $\Omega$.
$G$ sei $\varphi^{-1} (W) \subset V $, offen und beschränkt mit glattem Rand $\partial G $.

Man will nun (16) auf den Gaußschen Satz in $G \subset \mathbb{R}^2 $ zurückführen.

Zunächst hat man:

$\nu(\varphi(x)) = 
\frac
    {\pdiff{\varphi}{x_1} (x) \times \pdiff{\varphi}{x_2} (x)}
    {\left| \pdiff{\varphi}{x_1} (x) \times \pdiff{\varphi}{x_2} (x) \right|}
$
(Vergleiche Bsp 29.12: $a \wedge b \stackrel{\mathbb{R}^3}{=} a \times b $) und

$\sqrt{\mathrm{det } \varphi'(x)^T \varphi'(x)}
=
\left| \pdiff{\varphi}{x_1} (x) \times \pdiff{\varphi}{x_2} (x) \right| $
(vgl. 30.1, 30.2)

Als Integral auf der Mannigfaltigkeit $W$ ist somit die linke Seite in (16):

$\int\limits_W \rot F(u) \cdot \nu(u) \mathrm{d}a \\
\stackrel{\text{Def.}}{=}
\int\limits_g \mathrm{rot_u } \ F(\varphi(x)) 
\frac
    {\pdiff{\varphi}{x_1} (x) \times \pdiff{\varphi}{x_2} (x)}
    {\left| \pdiff{\varphi}{x_1} (x) \times \pdiff{\varphi}{x_2} (x) \right|}
\sqrt{\mathrm{det } \varphi'(x)^T \varphi'(x)}
\mathrm{d}x \\
=
\int\limits_G \mathrm{rot_u } \ F(\varphi(x)) \cdot 
\pdiff{\varphi}{x_1} (x) \times \pdiff{\varphi}{x_2} (x)
\mathrm{d}x $

\leftskip=30pt Wir nutzen im Folgenden zwecks Kompaktheit folgende Notation:\\
$F_l^k \coloneqq \pdiff{F_k}{u_l}$,
$\varphi_l^k \coloneqq \pdiff{\varphi_k}{x_l} $ wobei 
$(\varphi = (\varphi_1, \varphi_2, \varphi_3)) $\\

\leftskip=0pt
$
=
\int\limits_G
\begin{pmatrix}
    F_2^3 - F_3^2 \\
    F_3^1 - F_1^3 \\
    F_1^2 - F_2^1 \\
\end{pmatrix}
\cdot
\begin{pmatrix}
    \varphi_1^2 \varphi_2^3 - \varphi_1^3 \varphi_2^2\\
    \varphi_1^3 \varphi_2^1 - \varphi_1^1 \varphi_2^2\\
    \varphi_1^1 \varphi_2^2 - \varphi_1^2 \varphi_2^1\\
\end{pmatrix}
$

Schreibe im Weiteren nur die Terme mit $F^1$:

\begin{equation*}
    =
    \int\limits_G F_1^1 \cdot 0
    +
    F_2^1 (\varphi_1^2 \varphi_2^1 - \varphi_1^1 \varphi_2^2)
    +
    F_3^1 (\varphi_1^3 \varphi_2^1 - \varphi_1^1 \varphi_2^3)
    +
    \ldots \mathrm{d}x
    \tag{$\heartsuit$}
\end{equation*}

Für die rechte Seite in (16) sei

$x \rightarrow \tilde{x}(s) = (\tilde{x}_1 (s), \tilde{x}_2 (s)) $
die Parametrisierung der 1-dim. Mf $\partial G $ mit
$s \in I \subset \mathbb{R} \\
\Rightarrow s \rightarrow \psi(s) \coloneqq \varphi(\tilde{x}(s)) $
ist die Parametrisierung der 1-dim. Mf $\partial_M W $
und \\
$t(\psi(s)) = \frac{\psi'(s)}{|\psi'(s)|} $;
$\psi'(s) = \varphi`(\tilde{x}(s)) \cdot \tilde{x}(s) \\
\Rightarrow
    \int\limits_{\partial_M W} F \cdot t \mathrm{d}a
\stackrel{\text{Def.}}{=}
    \int\limits_I F(\psi(s)) \cdot t(\psi(s))
    \underbrace{
        \sqrt{\det \varphi'(s)^T \varphi'(s)}
        }_{
        |\psi'(s)|
        }
    \mathrm{d}s \\
=
    \int\limits_I F(\varphi(\tilde{x}(s))) \cdot
    \left(
        \varphi'(\tilde{x}(s)) 
        \underbrace{
            \frac{
                \tilde{x}'(s)
                }{
                |\tilde{x}'(s)|
                }
            }_{
            \coloneqq \tilde{t}(\tilde{x}(s))
            }
    \right)
    |\tilde{x}'(s)| \mathrm{d}s
$

\textbf{Skizze fehlt}    

$
=
    \int\limits_{\partial G} F (\varphi(x)) \cdot
    \left(
        \varphi'(x) \cdot \tilde{t}(x)) \mathrm{d}s
    \right)
$

\hspace{30pt}
    $
    \tilde{\nu}(\tilde{x}(s))
    =
    \frac{1}{
    | \tilde{x}(s) |}
    \begin{pmatrix}
        \tilde{x}_2'(s) \\
        \tilde{x}_1'(s) \\
    \end{pmatrix}
    $
    ist die äußere Einheitsnormale in $\tilde{x}(s) \in \partial G $ an $G$

$
=
    \int\limits_{\partial G} F^1(\varphi(x))
    \left(
        \varphi_1^1 \tilde{x}_1' + \varphi_2^1 \tilde{x}_2'
    \right)
    \frac{1}{| \tilde{x}' |}
    + \ldots \mathrm{d}a \\
=
    \int\limits_G F^1(\varphi(x))
    \begin{pmatrix}
        \varphi_2^1(x) \\
        -\varphi_1^1(x) \\
    \end{pmatrix}
    \tilde{\nu}
    + \ldots \mathrm{d}a \\
\stackrel{\textbf{Gauß}}{=}
    \int\limits_G \mathrm{div_x } \ F^1(\varphi(x))
    \begin{pmatrix}
        \varphi_2^1(x) \\
        -\varphi_1^1(x) \\
    \end{pmatrix}
    + \ldots \mathrm{d}x \\
=
    \int\limits_G 
    \mathrm{div_u } \ F^1 \pdiff{\varphi}{x_1} \varphi_2^1(x)
    + \underbrace{F_1 \varphi_{21}^1}_{\text{Schwarzscher Satz}}
    - \mathrm{div_u } \ F^1 \pdiff{\varphi}{x_2} \varphi_1^1
    - \underbrace{F^1 \varphi_{12}^1}_{\sum = 0}
    + \ldots \mathrm{d}x \\
=
    \int\limits_G F_1^1 
    \underbrace{\left( \varphi_1^1 \varphi_2^1 - \varphi_2^1 \varphi_1^1 \right)}_{=0}
    + F_2^1 \left( \varphi_1^2 \varphi_2^1 - \varphi_2^2 \varphi_1^1 \right)
    + F_3^1 \left( \varphi_1^3 \varphi_2^1 - \varphi_2^3 \varphi_1^1 \right)
    + \ldots \mathrm{d}x
$

Vergleich mit($\heartsuit$) liefert die Behauptung aus (16).

\end{proof}

\textbf{Hauptsatz der Vektoranalysis}\\
Falls für ein unbekanntes Vektorfeld
$F: \Omega \subset \mathbb{R}^3 \rightarrow \mathbb{R}^3 $
die Quellen, die Wirbel und der Fluss durch den Rand bekannt sind, so ist $F$ eindeutig
bestimmt, d.h. für gegebene Funktionen\\
$f: \Omega \rightarrow \mathbb{R}, \ G: \Omega \rightarrow \mathbb{R}^3, \ 
\varphi: \partial \Omega \rightarrow \mathbb{R} $ gelte \\
$\div F = f, \ \rot F = G $ auf $\Omega, \ F \cdot \nu = \varphi $ auf $\partial \Omega $
und die Kompatibilitätsbedingung: $\div G = 0 $ auf $\Omega $ und 
$
\int\limits_{\Omega} f \mathrm{d}x 
= 
\int\limits_{\partial \Omega} \varphi \mathrm{d}a \\
\Rightarrow F $ ist eindeutig bestimmt.\\
(Falls $\Omega,\ f, \ G, \ \varphi $ hinreichend regulär sind.)\\
Wichtige Anwendungen dessen finden sich z.B. in der Elektrodynamik.


\section{Gradientenfelder}

\begin{definition}[Gradientenfeld]
\mbox{}\\
Eine Abbildung $F: \Omega \subset \mathbb{R}^n \rightarrow \mathbb{R}^n $,
$\Omega $ offen, heißt \textbf{Gradientenfeld},
falls eine differenzierbare Funktion $f: \Omega \rightarrow \mathbb{R} $ exisitiert,
die $F(x) = f'(x) \forall x \in \mathbb{R} $ erfüllt.
\end{definition}

Wir wollen untersuchen, welche Vektorfelder Gradientenfelder sind, analog zur Suche
nach Stammfunktionen in Kapitel 25.

\begin{satz}[Notwendige Bedingung]

Sei $F = (F_1, \ldots, F_n): \Omega \subset \mathbb{R}^n \rightarrow \mathbb{R}^n$,
$\Omega $ offen, stetig differenzierbar und ein Gradientenfeld
\begin{equation}
    \Longrightarrow
    \pdiff{}{x_j} F_i (x) = \pdiff{}{x_i} F_j (x) \ \forall x \in \mathbb{R}^n; \
    i,j = 1, \ldots, n
\end{equation}
(1) heißt \emph{Integrabilitätsbedingung}; $\pdiff{F_i}{x_j} = \pdiff{f}{x_i x_j} $\\
Für $n=3 $ gilt: (1) $ \Leftrightarrow \rot F = 0 $
\end{satz}


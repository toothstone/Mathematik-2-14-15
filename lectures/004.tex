\begin{definition}[Parallelotop]
    Seien $a_1, \ldots, a_d \in \mathbb{R}^n (d \leq n)$ \\
    Dann heißt die Menge
    $P \left( a_1, \ldots, a_d \right) \coloneqq 
    \left\lbrace 
    \sum\limits_{j=1}^n t_j a_j | t_j \in [0,1], j = 1, \ldots, d    
    \right\rbrace$ \\
    das von $a_1, \ldots, a_d $ aufgespannte \textbf{Parallelotop} (auch d-Spat genannt).
\end{definition}

\textbf{Einschub:} Eine allgemeine Theorie für d-dimensionale Inhalte liefert das
Hausdorff-Maß $\mathcal{H}^d$, dieses ist jedoch sehr viel abstrakter und schwierig 
'auszurechnen'. Mithilfe von Mannigfaltgkeiten kommt man schneller zu Ergebnissen. \\
Weiterhin ist uns bereits das Maß über die delta-Funktion/Distribution bekannt, welches
zur Beschreibung von Punktmassen und -ladungen wichtig ist.

\begin{satz}
    \mbox{}
    Seien $a_1, \ldots, a_n \in \mathbb{R}^n$, \\
    das aufgespannte Volumen 
    $v(a_1, \ldots , a_n) \coloneqq \mathcal{L}^n (P (a_1, \ldots, a_n))$ \\
    so gilt:
    \begin{enumerate}
        \item[i)]
            $v(a_1, \ldots, \lambda a_k, \ldots, a_n) = 
            |\lambda| v(a_1, \ldots, a_n) \forall \lambda \in \mathbb{R}^n $
        \item[ii)]
            $v(a_1, \ldots, a_k + a_j, \ldots, a_n) = v(a_1, \ldots, a_n) $
            falls $k \neq j $ \\
            Dies ist bekannt als \textit{Prinzip des Cavaleri}, als Veranschaulichung kann
            ein Stapel Spielkarten dienen: Egal wie man eine Seitenfläche von einem Rechteck
            in ein Parallelogramm (oder umgekehrt) verschiebt, 
            das Volumen des Stapels bleibt gleich. \\
            \textbf{Skizze fehlt}
        \item[iii)]
            $v(a_1, \ldots, a_n) = 1 $ falls $a_1, \ldots, a_n $ ein Orthonormalsystem
            in $\mathbb{R}^n $ bilden. \\
            (Der Parallelotop ist dann der Einheitswürfel.)
        \item[iv)]
            $v(a_1, \ldots, a_n) = |\det A|$ für $A \coloneqq (a_1 | \ldots | a_n) $ \\
            Das heißt die Determinante der Matrix mit den Spaltenvektoren $a_1, \ldots, a_n$
            liefert das Volumen des aufgespannten Parallelotops. (Vgl. lin. Algbebra)
    \end{enumerate}
\end{satz}

    \textbf{Beachte:} Die Eigenschaften i) - iii) implizieren bereits iv), die Argumentation
    dazu verläuft wie zu den aus LAG bekannten Eigenschaften der Determinante, vgl.
    auch die axiomatische Definition der Determinante.

\begin{proof}
    \mbox{}
    \begin{enumerate}
        \item[a)] 
            Angenommen, $a_1, \ldots, a_n$ sind linear abhängig. \\
            Dann ist das aufgespannte Parallelotop 'flach', da es in mindestens 
            einer Dimension an Ausdehnung fehlt.
            $\Rightarrow v(a_1, \ldots, a_n) = 0 $ \\
            $\Rightarrow $ iv) ist korrekt 
            (da die Determinante einer singulären Matrix Null ist) \\
            $\Rightarrow $ i) und ii) sind korrekt
        \item[b)]
            Angenommen, $a_1, \ldots, a_n$ sind linear unabhängig. \\
            Sei $\lbrace e_1, \ldots, e_n \rbrace$ die Standard-Orthonormalbasis in
            $\mathbb{R}^n $, dafür gilt iii) nach der Definition des Lebesgue-Maß
            (Es ist ein Quader mit allen Seitenlängen gleich Eins). \\
            Weiter seien nun $U \coloneqq P(e_1, \ldots, e_n),
            V \coloneqq P(a_1, \ldots, a_n) $ \\
            $\Rightarrow A: \mathrm{int}\ U \rightarrow \mathrm{int}\ V $
            ist ein Diffeomorphismus (A ist regulär, ist damit differenzierbar und
            besitzt ein differenzierbares Inverses). \\
            Offenbar ist $A'(y) = A \forall y $ \\
            $\xRightarrow{\text{Trafosatz, Kap. 24}}
            \mathcal{L}^n (V) = \int\limits_V \mathrm{d}x \stackrel{y = Ax}{=}
            \int\limits_U |\det A| \mathrm{d}y = |\det A| \underbrace{\mathcal{L}^n (U)}_1
            = |\det A|
            \Rightarrow $ iv) $\Rightarrow $ i), ii), iii) folgen als Eigenschaften
            der Determinanten
    \end{enumerate}
\end{proof}

\textbf{Ziel:} Bestimmung des d-dimensionalen Inhalts 
$v_d (P(a_1, \ldots, a_d))$ in $\mathbb{R}^n $
\textbf{Idee:} Betrachte $P(a_1, \ldots, a_n)$ als Teilmenge eines d-dimensionalen
Vektorraums $X$ und nimm das d-dimensional Lebesgue-Maß in $X$.\\
Somit sollte $v_d: 
\underbrace{\mathbb{R}^n \times \ldots \times \mathbb{R}^n }_{\text{d-mal}}
\rightarrow \mathbb{R}_{\geq 0} $
folgende Eigenschaften haben:
\begin{enumerate}
    \item[(v1)]
        $v_d (a_1, \ldots, \lambda a_k, \ldots, a_d) = |\lambda| v_d(a_1, \ldots, a_d)
        \forall \lambda \in \mathbb{R} $
    \item[(v2)]
        $v_d (a_1, \ldots, a_k + a_j, \ldots, a_d) = v_d(a_1, \ldots, a_d) $
        falls $k \neq j$ (Prinzip des Cavaleri)
    \item[(v3)]
        $v_d(a_1, \ldots, a_d) = 1 $ falls $\lbrace a_1, \ldots, a_d \rbrace $
        orthonormal zueinander sind.
\end{enumerate}

\begin{satz}
    $v_d$ ist durch (v1), (v2), (v3) eindeutig bestimmt, und es gilt:
    \begin{equation}
        v_d(a_1, \ldots, a_d) = 
        \sqrt[]{\det \underbrace{A^T A}_{\in \mathbb{R}^{d \times d}}}
        \text{ mit } 
        A \coloneqq\underbrace{(a_1 | \ldots | a_d)}_{\in \mathbb{R}^{n \times d}}
    \end{equation}
\end{satz}

\textbf{Bemerkung:}
\mbox{}
\begin{enumerate}
    \item
        für $d=n $ liefert (30.1) Gleichung iv) in Satz 1
    \item
        $A^T A $ ist stets symmetrisch und positiv definit
        $\left( \left\langle x, A^T Ax \right\rangle =
        \langle Ax, Ax \rangle = \|Ax\|^2 \geq 0 \right)$
        und somit ist auch stets $\det A^T A \geq 0 $
    \item
        $v_d (a_1, \ldots, a_d) = 0 \Leftrightarrow a_1, \ldots, a_d$ linear abhängig
\end{enumerate}

\begin{proof}
    Selbststudium, verwende:\\
    $A^T A = 
    \begin{pmatrix}
        \alpha_{11} & \ldots & \alpha_{1n} \\
        \vdots      & \ddots & \vdots \\
        \alpha_{n1} & \ldots & \alpha_{nn}
    \end{pmatrix}
    $mit $\langle a_i, a_j \rangle $ \\
    und argumentiere wie bei den Eigenschaften der Determinante.
\end{proof}

\begin{beispiel}
    $d = n-1 $: Sei $a_1, \ldots, a_{n-a} \in \mathbb{R}^n, 
    a \coloneqq a_1 \wedge \ldots \wedge a_{n-1} $
    \begin{equation}
        \Rightarrow v_{n-1} (a_1, \ldots, a_{n-1}) = |a|_2
    \end{equation}
    Das heißt die euklidische Länge des äußeren Produkts liefert das Volumen.\\
    Denn: $
    \begin{pmatrix}
        a^T \\
        \rule[.5ex]{1em}{1pt}\\
        A^T
    \end{pmatrix}
    \bullet
    \begin{pmatrix}
        a & | & A 
    \end{pmatrix}
    =
    \begin{pmatrix}
        \langle a,a \rangle & 0 \\
        0 & A^T A
    \end{pmatrix}
    $ wegen $\langle a_i, a_j \rangle = 0 \forall j $; $A$ wie in (1)\\
    $\Rightarrow |a|_2^2 \det A^T A = (\det (a|A))^2 \stackrel{(29.4)}{=}
    |a|_2^4 \xRightarrow{\text{(1)}} $ (2)
\end{beispiel}

\textbf{Frage:} Existiert für eine Mannigfaltigkeit $M$ eine Transforamtion, so dass das
Volumen eines Quaders $Q \in \mathbb{R}^d $ auf das eines Parallelotops 
$P \subset T_uM \in \mathbb{R}^n $ abgebildet wird: \\
$v_d(\text{Quader }Q) \xrightarrow{\varphi'(x)} v_d(\text{Parallelotop } P) $ ?\\
\textbf{Skizzen fehlen}\\
Für einen Quader $Q = P(b_1, \ldots, b_d) \subset \mathbb{R}^d $ ist 
$P(a_1, \ldots, a_d) \subset T_uM \in \mathbb{R}^n $ das zugehörige Parallelotop, falls
$a_j = \varphi'(x) b_j $,für $ j=1, \ldots, d $

\begin{satz}
    Sei $M$ eine d-dimensionale Mannigfaltigkeit,\\
    $\varphi$ eine Parametrisierung um $\varphi(x) = u \in M $
    und sei \\
    $Q \coloneqq P(b_1, \ldots, b_d) \subset \mathbb{R}^d $ ein Quader 
    $(b_j \in \mathbb{R}^d),
    a_j \coloneqq \varphi'(x) b_j,\ j= 1, \ldots, d $
    \begin{equation}
        \Longrightarrow 
        v_d(a_1, \ldots, a_d) = 
        \sqrt{\det \varphi'(x)^T \varphi'(x)}\ v_d(b_1, \ldots, b_d)
    \end{equation}
\end{satz}

\begin{definition}[Maßtensor und Gramsche Determinante]
    \mbox{} \\
    $\varphi'(x)^T \varphi'(x) \in \mathbb{R}^{d \times d} $
    heißt \textbf{Maßtensor} von $\varphi$ in $x$ und\\
    $g^\varphi (x) \coloneqq \det \varphi'(x)^T \varphi'(x) $
    heißt \textbf{Gramsche Determinante} von $\varphi$ in $x$.
\end{definition}

\begin{proof}
    Sei $B = (b_1, \ldots, b_d) \in \mathbb{R}^{d \times d},\ 
    A = (a_1, \ldots, a_d) \in \mathbb{R}^{n \times d} \\
    \xRightarrow{\text{(1)}} v_d(a_1, \ldots, a_d) 
    = \sqrt{\det A^T A} = \sqrt{\det ((\varphi'(x) B)^T \varphi'(x) B)}
    = \sqrt{\det \varphi'(x)^T \varphi'(x)} 
    \underbrace{\sqrt{\det B^T B}}_{= v_d(b_1, \ldots, b_d)}
    $
\end{proof}

Sei $M \subset \mathbb{R}^n $ eine d-dimensionale Mannigfaltigkeit,
$\varphi: V \rightarrow U $ eine lokale Parametriesierung,
$f: \rightarrow \mathbb{R} $ eine Funktion auf dem Kartengebiet $U$.\\
Motiviert durch die Riemann-Summen (Kap. 22)\\
$\sum f(u_i) v_d(P_i) = \sum f(\varphi(x_i)) \sqrt{g^\varphi (x_i)} v_d(Q_i) $
mit $P_i = \varphi'(x_i) Q_i $
definieren wir:

\begin{definition}[Integral über über Kartengebiet]
    \begin{equation}
        \int\limits_U f \mathrm{d}a 
        \coloneqq \int\limits_V f(\varphi(x)) \sqrt{g^\varphi (x)} \mathrm{d}x
    \end{equation}
    als \textbf{Integral von $f$} über das Kartengebiet $U$ falls die rechte
    Seite existiert.\\
    $f$ heißt dann \textbf{integrierbar} auf $U$.
\end{definition}

\textbf{Bemerkung:}
\begin{enumerate}
    \item[-]
        die rechte Seite in (30.4) ist ein Lebesgue-Integral in $\mathbb{R}$
    \item[-]
        damit die Definition von (30.4) sinnvoll ist, sollte die rechte Seite
        unabhängig von $\varphi$ sein
    \item[-]
        mittels des Hausdorff-Maß $\mathcal{H}^d $ kann
        $\int\limits_U f \mathrm{d}a $
        völlig analog zum Lebesgue-Maß definiert werden
        $\left(\int\limits_U f(u) \mathrm{d} \mathcal{H} (u) \right) $
    \item[-]
        für n-dimensionale Mannigfaltigkeiten $M \subset \mathbb{R}^n: \\
        \int\limits_U f \mathrm{d}a = $ Lebesgue-Integral 
        $\int\limits_U f \mathrm{d}x $ falls dieses existiert
\end{enumerate}

\begin{satz}
    Sei $M \subset \mathbb{R}^n $ eine d-dimensionale Mannigfaltigkeit,\\
    $U \subset M $ ein Kartengebiet, $f:U \rightarrow \mathbb{R} $ und
    $\varphi_i: V_i \rightarrow U,\ i=1,2 $ seien zugehörige Parametrisierungen. \\
    $\Longrightarrow \int\limits_{V_1} f(\varphi_1(x)) \sqrt{g^{\varphi_1} (x)} \mathrm{d}x
    = \int\limits_{V_2} f(\varphi_2(x)) \sqrt{g^{\varphi_2} (x)} \mathrm{d}x
    $ falls ein Integral exisitiert.
\end{satz}

\textbf{Somit:} (4) ist unabhängig von der Parametrisierung
\begin{equation}
    f(.) \text{ int'bar auf } U
    \Longleftrightarrow f(\varphi(.)) \sqrt{g^\varphi (x)} 
    \text{ int'bar auf } V
    \text{ für eine Param. }
    \varphi: V \rightarrow U
\end{equation}

\begin{proof}
    $\psi \coloneqq \varphi_1^{-1} \bullet \varphi_2: V_2 \rightarrow V_1 $
    ist ein Diffeomorphismus nach Lemma 5 \\
    $\xRightarrow{\text{Trafosatz}} 
    \int\limits_{V_1} f(\varphi_1(x)) \sqrt{g^{\varphi_1} (x)} \mathrm{d}x
    \stackrel{x=\psi(y)}{=}
    \int\limits_{V_2} f(\varphi_1(\psi(y)))
    \underbrace{
        \sqrt{\det \varphi_1'(\psi(y))^T \varphi_1'(\psi(y))}
        \underbrace{
            |\det \psi'(y)|
            }_{
            = \sqrt{\det \psi'(y)^T \psi'(y)}}
        }_{
        = \sqrt{\det \psi'^T \varphi_1'^T \psi' \varphi_1'}
        = \sqrt{\det (\psi' \varphi_1')^T \psi' \varphi_1'}
        }
    \mathrm{d}y
    $ wegen $
    \varphi_2 (y) = \varphi_1 (\psi(y))
    \xRightarrow{\text{Kettenregel}}
    \varphi_2' (y) = \varphi_1' (\psi(y)) \psi'(y)
    \Rightarrow $ Behauptung.
\end{proof}
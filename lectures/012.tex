Wir haben gesehen, dass die heuristische Methode ein sehr 
effektives Werkzeug zum Lösen von Differenzialgleichungen 
mit getrennten Variablen ist. Doch was rechtfertigt sie 
mathematisch?\newpage

\begin{satz}
Seien $f:I_x\subset\mathbb{R}\rightarrow\mathbb{R}$ und 
$g:I_n\subset\mathbb{R}\rightarrow\mathbb{R}$ zwei stetige 
Funktionen und $I_x$, $I_n$ zwei Intervalle mit einem $x_0\in I_x$ 
und einem $u_0\in\mathrm{int\ } I_n$. Außerdem ist 
$g(u_0)\neq 0$.\\
Dann besitzt das Anfangswertproblem
\begin{equation*}
u'(x)=g(u(x))f(x) \ \ \mathrm{mit\ }u(x_0)=u_0
\end{equation*}
eine eindeutige Lösung auf der (eventuell einseitigen) Umgebung 
von $x_0$, die man durch Auflösung von (2.1) nach $u$ erhält.
\end{satz}

\begin{proof}
Zunächst eine kleine Wiederholung: Sei $\varphi:I\rightarrow
\mathbb{R}$ im offenen Intervall $I$ stetig differenzierbar mit 
$\varphi'(x)\neq 0$. Dann existiert eine stetig differenzierbare 
Umkehrabbildung $\varphi^{-1}$. \\
Nun setzten wir
\begin{equation*}
G(u)\coloneqq\int_{u_0}^u\frac{1}{g(s)}\mathrm{d}s, \ \ \
F(x)\coloneqq\int_{x_0}^x f(t) \mathrm{d}t
\end{equation*}
Da $g(u_0)\neq 0$ ist, existiert $G$ in einer Umgebung von $u_0$ 
und $G'=\frac{1}{g}\neq 0$. \\
Wir lösen nun einfach (2.1), das heißt $G(u)=F(x)$ ist lokal 
möglich, und erhalten
\begin{equation*}
u(x)=G^{-1}\left(F(x)\right)
\end{equation*}
Da $F(x_0)=0$ ist, wird
\begin{equation*}
u(x_0) = G^{-1} \left( F(x_0) \right) =G^{-1}(0)=u_0
\end{equation*}
Damit ist zunächst das Anfangswertproblem erfüllt. Nun ist noch 
nachzuprüfen, ob $u(x)$ auch der Differenzialgleichung genügt.
Dazu differenzieren wir $G\left(u(x)\right)=F(X)$ und erhalten 
mit Kettenregel
\begin{equation*}
G'\left(u(x)u'(x)\right)=F'(x)=f(x)
\end{equation*}
Da $G'=\frac{1}{g}$ ist, haben wir
\begin{equation*}
u'(x)=g\left(u(x)\right)f(x)
\end{equation*}
Damit ist die Differenzialgleichung erfüllt. Bleibt nur noch
die Eindeutigkeit zu zeigen.\\
Nehmen wir an, $v(x)$ wäre eine andere Lösung. Offenbar ist 
in einer Umgebung von $x_0$ immer $g(v(x))\neq 0$. Deswegen 
können wir schreiben
\begin{equation*}
\frac{v'(x)}{g(v(x)}=f(x)
\end{equation*}
\begin{equation*}
\int_{x_0}^x\frac{v'(x)}{g(v(x)}\mathrm{d}t = 
\int_{x_0}^xf(t)\mathrm{d}t
\end{equation*}
\begin{equation*}
\int_{v_0}^{v(t)}\frac{1}{g(s)}\mathrm{d}s = 
\int_{x_0}^xf(t)\mathrm{d}t
\end{equation*}
Nun steht da
\begin{equation*}
G(v(x))=F(x)
\end{equation*}
\begin{equation*}
v(x)=G^{-1}(F(x))=u(x)
\end{equation*}
und die Behauptung ist gezeigt.\\
Falls $g(u_0)=0$ ist, ist die Lösung einfach trivial konstant, 
nämlich $u(x)=u_0$.
\end{proof}
\ \\
\linebreak
Wir haben bereits gesehen, dass man gewisse Fälle auf 
Differenzialgleichungen mit getrennten Variablen zurückführen 
kann. Deshalb wollen wir einige weitere Beispiel genauer 
untersuchen.\\
\linebreak


\item[iv)] $u'(x)=f(ax+bu(x)+c)$\\ \linebreak
Wir wählen als Ansatz 
\begin{equation*}
v(x)\coloneqq ax+bu(x)+c
\end{equation*}
Und setzten durch einsetzen sofort, dass $v$ die 
Differenzialgleichung 
\begin{equation*}
v'=a+bu'=a+f(v)
\end{equation*}
erfüllt. Dabei handelt es sich wieder um eine Gleichung mit 
getrennten Variablen, für die wir nur $v(x)$ bestimmen müssen. 
Nach Rücksubstitution erhalten wir dann die ursprüngliche 
Lösung.
\begin{equation*}
u(x)=\frac{1}{b}\left(v(x)-ax-c\right)
\end{equation*}

\begin{beispiel*}
\begin{equation*}
u'=(x+u)^2
\end{equation*}
\begin{equation*}
v(x)=x+u(x)
\end{equation*}
\begin{equation*}
v'=1+v^2
\end{equation*}
\begin{equation*}
\int_{v_0}^v\frac{1}{1+s^2}\mathrm{d}s = 
\int_{x_0}^x\mathrm{d}t
\end{equation*}
Daraus ergibt sich dann
\begin{equation*}
	v(x)=\tan(x+c)
\end{equation*}
\begin{equation*}
u(x)=\tan(x+c)-x
\end{equation*}
Wir führen noch einmal die Probe durch:
\begin{equation*}
u*\frac{1}{\cos^2(x+c)}-1=\frac{\sin^2(x+c)}{\cos^2(x+c)} = 
(u+c)^2
\end{equation*}
\end{beispiel*}


\item[v)] $u'=f\left(\frac{u}{x}\right)$ für $x=0$\\
\linebreak
Wir wählen als Ansatz $v=\frac{u}{x}$ und erhalten
\begin{equation*}
v'=\frac{u'-u}{x^2}=\frac{u'}{x}-\frac{1}{x}\frac{u}{x} = 
\frac{f(v)-v}{x}
\end{equation*}
was wir wieder durch die heuristische Methode lösen können.
\begin{equation*}
u(x)=xv(x)
\end{equation*}

\begin{beispiel*}
$u'=\frac{u}{x}-\frac{x^2}{u^2}$ mit dem Anfangswert $u(1)=1$ 
$x>0$
\begin{equation*}
v\coloneqq\frac{u}{x}
\end{equation*}
Durch Zufall ergibt sich sogar derselbe Anfangswert
\begin{equation*}
v(1)=1
\end{equation*}
Einsetzen in die Differenzialgleichung liefert
\begin{equation*}
v'=-\frac{1}{xv^2}
\end{equation*}
\begin{equation*}
\int_1^v s^2 \mathrm{d}s = -\int_1^x\frac{1}{t}\mathrm{d}t
\end{equation*}
\begin{equation*}
\frac{1}{3}v^3-\frac{1}{3}=-\ln(x)
\end{equation*}
\begin{equation*}
	v(x)=\sqrt[3]{1-3\ln(x)} \ \ \mathrm{falls\ } 1-3\ln(x)>0
\end{equation*}
\begin{equation*}
u(x)=x\sqrt[3]{1-3\ln(x)}
\end{equation*}
Probe:
\begin{equation*}
u'=\sqrt[3]{1-3\ln(x)}+x\frac{1}{3}(1-3\ln(x))^{-\frac{2}{3}} 
\left(-\frac{3}{x}\right) = 
\frac{u}{x} - \frac{x^2}{u^2}
\end{equation*}
\end{beispiel*}
\end{enumerate}

\subsection{Lineare Differenzialgleichungen 1. Ordnung}

Allgemeine explizite Darstellung:
\begin{equation*}
u'(x)=g(x)u(x)+f(x)
\end{equation*}
falls $f(x)\equiv 0$, nennt man sie auch \emph{homogene}, sonst 
\emph{inhomogene} Differenzialgleichung. \\
Im folgenden seien $f$ und $g$ stetig auf $I\subset\mathbb{R}$.

\begin{enumerate}

	\item[i)] \emph{homogene lineare Differenzialgleichung: } 
	$u'(x)=g(x)u(x)$ mit $u(0)=u_0$\\
	\linebreak
	Durch Trennung der Variablen erhalten wir für $u_0\neq 0$
	\begin{equation*}
	\int_{u_0}^u\frac{1}{s}\mathrm{d}s = 
	\int_{x_0}^x g(t) \mathrm{d}t
	\end{equation*}
	mit
	\begin{equation*}
	G(x)\coloneqq \int_{x_0}^x g(t) \mathrm{d}t
	\end{equation*}
	erhalten wir die bereits bekannte Lösung
	\begin{equation}
	u(x)=u_0e^{G(x)}
	\end{equation}
	die im Übrigen ihr Vorzeichen nicht ändert. Außerdem erfüllt 
	diese Lösung auch für $u_0=0$ die Gleichung.\\
	Die Frage ist nun: Gibt es noch weitere Lösungen? Um dies 
	zu beantworten, nehmen wir wieder $v(x)$ mit $v(x_0)=0$ und 
	$v(x_1)=v_1\neq v(x_0)$ für $x_1\neq 0$ eine weitere Lösung 
	wäre. Wir betrachten das Anfangswertproblem
	\begin{equation*}
	v'=g(x)v \ \ \ \mathrm{mit\ }v(x_1)=v_1
	\end{equation*}
	Satz 34.1 liefert uns die eindeutige Lösung
	\begin{equation*}
	v(x)=v_1e^{\int_{x_0}^xg(t)\mathrm{d}t}
	\end{equation*}
	damit ist aber $v(x_0)\neq 0$, was zum Widerspruch führt. 
	Das heißt also, dass (35.2) eine eindeutige Lösung ist.
	
	\item[ii)] \emph{inhomogene lineare Differenzialgleichung: }
	$u'(x)=g(x)u(x)+f(x)$ mit $u(x_0)=u_0$\\
	\linebreak
	Um dies zu lösen, wenden wir auf (35.2) die sogenannte 
	\textbf{Variation der Konstanten} an.
	\begin{equation*}
	u(x)=C(x)e^{\int_{x_0}^xg(t)\mathrm{d}t}
	\end{equation*}
	das heißt, wir variieren die Konstante in 
	der allgemeinen Lösung der homogenen Gleichung. Für $C$ muss 
	dann gelten 
	\begin{equation*}
	u'-gu=C'e^{G(x)}+gCe^{G(x)}-gCe^{G(x)}\stackrel{!}{=}f
	\end{equation*}
	Das heißt
	\begin{equation*}
	C'=f(x)e^{-G(t)} \ \ \ \mathrm{mit\ }C(x_0)=u_0
	\end{equation*}
	So erhalten wir durch Integration
	\begin{equation*}
	C(x)=u_0+\int_{x_0}^xf(t)e^{-G(t)}\mathrm{d}t
	\end{equation*}
	und damit
	\begin{equation}
	u(x) = 
	\left(u_0+\int_{x_0}^xe^{-G(t)}\mathbb{d}t\right)e^{G(x)}
	\ \ \ \mathrm{mit\ } G(x)=\int_{x_0}^x g(s) \mathrm{d}s
	\end{equation}
\end{enumerate}
\ \\
\linebreak
\begin{satz}
Seien $f,g$ stetig auf $I\subset\mathbb{R}$ und $x_0\in I$. \\
Dann besitzt das Anfangswertproblem 
\begin{equation*}
u'=g(x)u+f(x),\ u(x_0)=u_0
\end{equation*}
auf $I$ die eindeutige Lösung (35.3).\\
\linebreak
\emph{Beachte:} Die Lösung existiert auf dem gesamten Intervall.
\end{satz}
Bevor wir zum Beweis kommen, müssen wir zunächst ein paar 
allgemeinere Begriffe klären. Die Menge aller Lösungen einer 
Differenzialgleichung nennen wir auch 
\textbf{allgemeine Lösung}, welche Parameter enthält, die 
durch Angabe von Anfangsbedingungen festgelegt werden. \\
Für lineare Differenzialgleichungen gilt das sogenannte 
\textbf{Superpositionsprinzip}, nachdem sich die allgemeine 
Lösung einer inhomogenen Gleichung als Summe aus der allgemeinen 
Lösung des zugehörigen homogenen Problems und einer einzigen 
speziellen Lösung schreiben lässt.
\begin{equation*}
\underbrace{u_{i}(x)}_{\mathrm{allgemeine\ inhomogene\ Lösung}}
= \underbrace{u_h(x)}_{\mathrm{allgemeine\ homogene\ Lösung}} + 
\underbrace{u_s(s)}_{\mathrm{spezielle\ inhomogene\ Lösung}}
\end{equation*}
Das lässt sich einfach nachprüfen, in dem wir $u_1$ und $u_2$ 
betrachten, die beide Lösungen der inhomogenen Gleichung
\begin{equation*}
u'=u_1'-u_2'=g(u_1-u_2)=gu
\end{equation*}
sein sollen.
\begin{proof}
Offenbar sind alle Ausdrücke in (35.3) definiert. Durch 
Differenziation von (35.3) erhalten wir
\begin{equation*}
u'=f(x)e^{-G(x)}\cdot e^{G(x)} + 
ge^{G(x)}\left(u_0+\int_{x_0}^x f(t)e^{-G(t)}\mathrm{d}t\right) 
= ue^{-G(x)}=f+gu
\end{equation*}
Damit sind sowohl Anfangswertproblem als auch die 
Differenzialgleichung erfüllt. Für die Eindeutigkeit 
wählen wir wieder $v$ als andere Lösung und betrachten $w=u-v$. 
Dann gilt
\begin{equation*}
w'-gw=u'-v'-gu-gv=f-f=0
\end{equation*}
\begin{equation*}
w(x_0)=u(x_0)-v(x_0)=u_0-u_0=0
\end{equation*}
$w$ löst also die homogene Differenzialgleichung mit ihrem 
Anfangswertproblem. Nach (35.2) ist diese Lösung eindeutig $w=0$ für alle $x$ und somit $v(x)=u(x)$.
\end{proof}
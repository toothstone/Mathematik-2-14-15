Nachdem wir nun auf Kartengebieten integrieren können, 
lassen sich natürlich eine Vielzahl von Eigenschaften untersuchen. 

\begin{satz}
	Seien $f,g,f_k:U\rightarrow\mathbb{R}$ Funktionen, 
	die vom Kartengebiet $U$ der Mannigfaltigkeit $M\subset\mathbb{R}^n$ 
	abbilden, dann gelten die folgenden Beziehungen:
	\begin{enumerate}
		\item 	$f$ ist einerseits genau dann auf $U$ integrierbar, 
				wenn $|f|$ integrierbar auf $U$ ist und auch genau dann, 
				wenn es $f^+$ und $f^-$ sind.
		\item 	Sind $f$ und $g$ integrierbar auf $U$ und $c\in\mathbb{R}$, 
				so ist
				\begin{equation*}
					\int_Ucf\pm g\mathrm{d}a = 
					c\int_Uf\mathrm{d}a\pm\int_Ug\mathrm{d}a
				\end{equation*}
		\item 	Sind $f$ und $g$ integrierbar und $g$ beschränkt auf $U$, 
				dann ist auch $f\circ g$ integrierbar auf $U$.
		\item 	Sind $f$ und $g$ integrierbar und $f\leq g$ auf $U$, 
				dann dies auch für die Integrale:
				\begin{equation*}
					\int_Uf\mathrm{d}a\leq\int_Ug\mathrm{d}a
				\end{equation*}
		\item	\textbf{Monotone Konvergenz:} Seien alle $f_k$ 
				integrierbar und $f_1\leq f_2\leq ...$ auf $U$, 
				außerdem sei die Folge der Integrale $\left(\int_Uf_k\mathrm{d}a\right)$ 
				beschränkt sowie $f(u)=\lim\limits_{k\rightarrow\infty}f_k(u) \ \ \forall u\in U$, 
				dann ist $f$ integrierbar auf $U$ mit
				\begin{equation*}
					\int_Uf\mathrm{d}a = 
					\lim\limits_{k\rightarrow\infty}\int_Uf_k\mathrm{d}a
				\end{equation*}
		\item 	\textbf{Majorisierte Konvergenz:} Seien $g$ und alle $f_k$ 
				integrierbar und weiterhin $f_k\leq g \ \ \forall k$ auf $U$. 
				Außerdem möge $f(u) = \lim\limits_{k\rightarrow\infty}f_k(u) \ \ \forall u\in U$ 
				sein. Dann ist $f$ integrierbar auf $U$ mit
				\begin{equation*}
					\int_Uf\mathrm{d}a = 
					\lim\limits_{k\rightarrow\infty}\int_Uf_k\mathrm{d}a
				\end{equation*}
	\end{enumerate}
\end{satz}
\newpage
\begin{proof}
	Wir werden diesen Beweis sehr kurz halten, indem wir einige 
	Äquivalenzen finden, aus denen unter Verwendung der Eigenschaften 
	des Lebesque-Maßes die Behauptung folgen wird.\\
	Wir setzen zunächst $\varphi:V\rightarrow U$ als die Parametrisierung 
	des	Kartengebietes $U$. 
	\begin{enumerate}
		\item 	$f$ ist genau dann integrierbar, wenn es 
				$f\left(\varphi(.)\right)\sqrt{g^\varphi(.)}$ ist. 
				Dies folgt aus 	der Definition der Gramschen Determinante. 
		\item 	$f\leq g$ ist äquivalent dazu, dass 
				$f\left(\varphi(.)\right)\sqrt{g^\varphi(.)} \leq 
				g\left(\varphi(.)\right)\sqrt{g^\varphi(.)}$ ist.
	\end{enumerate}
	Daraus folgen unter Verwendung der Eigenschaften des Lebesque-Maßes 
	schließlich die Behauptungen. 
\end{proof}

\section{Integrale auf Mannigfaltigkeiten}

Im letzten Kapitel haben wir uns dem Integrieren auf Kartengebieten 
gewidmet. Das möchten wir natürlich nun auch auf ganzen Mannigfaltigkeiten 
tun können. Da stellt sich jedoch die Frage, wie dieses Integral dann 
genau aussehen soll. \\
Wir könnten uns natürlich überlegen, die Mannigfaltigkeit einfach 
mit Kartengebieten $U_\beta$, wobei $\beta$ hier eine vollkommen 
beliebige Indexmenge sein kann, zu überdecken und die Integration 
dann auf jedem Gebiet einzeln durchzuführen. An dieser Stelle stoßen 
wir jedoch auf ein Problem. Kartengebiete sind immer offen! Sie 
werden sich also, um die Mannigfaltigkeit Überdecken zu können, 
überlappen müssen. Damit wird das berechnete Integral immer zu 
groß sein. Deshalb müssen wir uns etwas anderes ausdenken.\\
\linebreak
Den Ausweg liefert eine sehr hilfreiche Methoden; die sogenannte 
\emph{Zerlegung der Eins}, für die wir $\alpha\equiv 1$ auf der 
Mannigfaltigkeit so setzen, dass $\sum\limits_{j=1}^\infty \alpha_j(u)=1$ ist.

\begin{definition}[Zerlegung der Eins]
	Eine Menge stetiger Funktionen $\alpha_j:M\rightarrow [0,1]$ 
	mit $j\in\mathbb{N}$ heißt \textbf{Zerlegung der Eins} auf 
	einer Menge $M\subset\mathbb{R}^n$, falls folgende Bedingungen 
	erfüllt sind:
	\begin{enumerate}
		\item 	$\sum\limits_{j=1}^\infty a_j(u) = 1   \ \ \ \forall u\in M$
		\item 	Die Zerlegung ist lokal endlich, das heißt für alle 
				$u\in M$ existiert eine Umgebung $U\subset M$ bezüglich 
				$M$, so dass auf dieser Umgebung \emph{fast überall} 
				$\alpha_j=0$ ist.
	\end{enumerate}
\end{definition}

\begin{definition}[Unterordnung]
	Sei $\mathcal{U}$ eine bezüglich $M$ offene Überdeckung von 
	$M\subset\mathbb{R}^n$. Dann heißt die Zerlegung der Eins $\{\alpha_j\}$ 
	der Überdeckung $\mathcal{U}$ \textbf{untergeordnet}, falls es 
	für alle $j\in\mathbb{N}$ ein $U_j\in\mathcal{U}$ gibt, so dass
	\begin{equation*}
		\mathrm{supp\ }\alpha_j\subset U_j
	\end{equation*} 
	wobei mit ''$\mathrm{supp}$'' hier der Träger von $\alpha_j$, 
	gegeben durch 
	\begin{equation*}
		\mathrm{supp\ }\alpha_j=\overline{\left\{u\in M \middle| \alpha_j(u)\neq 0 \right\} }
	\end{equation*}
	gemeint ist.
\end{definition}

\begin{satz}[Existenz der Zerlegung der Eins]
	Sei $M\subset\mathbb{R}^n$ und $\mathcal{U}$ eine bezüglich $M$ 
	offene Überdeckung von $M$, dann existiert immer eine Zerlegung 
	der Eins $\{\alpha_j\}$ von $M$, die $\mathcal{U}$ untergeordnet 
	ist.
\end{satz}

Hierzu muss man anmerken, dass wir $\alpha_j$ immer so konstruieren 
können, das es unendlich oft differenzierbar ist. Später betrachten 
wir dann die die Überdeckung einer Mannigfaltigkeit mit Kartengebieten. 

\begin{proof}
	Dieser Beweis ist sehr technisch und langweilig. Deshalb wird hier 
	auf die Literatur (\emph{Königsberger}) verwiesen.
\end{proof}

\begin{definition}[Integrierbarkeit auf Mannigfaltigkeiten]
	Sei $M\subset\mathbb{R}^n$ eine Mannigfaltigkeit, $f:M\rightarrow\mathbb{R}$, 
	$\mathrm{supp\ }\subset U \subset M$, wobei $U$ ein Kartengebiet 
	von $M$ ist. Dann heißt $f$ \textbf{integrierbar auf M} falls 
	$f_{|U}$ (die Funktion eingeschränkt auf U) auf dem Kartengebiet 
	integrierbar ist.
	\begin{equation}
		\int_Mf\mathrm{d}a=\int_Mf_{|U}\mathrm{d}a
	\end{equation}
	heißt dann das \textbf{Integral} von $f$ auf $M$.
\end{definition}
\newpage

\begin{lemma}[Kriterium für Integration auf Kartengebieten]
Sei $M\subset\mathbb{R}^n$ eine Mannigfaltigkeit, $f:M\rightarrow\mathbb{R}$, 
	$\mathrm{supp\ }\subset U \subset M$, wobei $U$ ein Kartengebiet 
	von $M$ ist, und $\{\alpha_j\}$ eine Zerlegung der Eins auf $M$. 
	Dann ist die Integrierbarkeit von $f$ auf $M$ zu folgenden 
	Aussagen äquivalent:
	\begin{enumerate}
		\item 	$f\alpha_j$ ist für alle $j\in M$ integrierbar auf $M$.
		\item	$\sum\limits_{j=1}^\infty\int_M|f|\alpha_j\mathrm{d}a<\infty$
	\end{enumerate}
	Daraus folgt dann auch
	\begin{equation}
		\int_Mf\mathrm{d}a=\sum\limits_{j=1}^\infty\int_Mf\alpha_j\mathrm{d}a
	\end{equation}
\end{lemma}

\begin{proof}
	Ist $f$ integrierbar auf $M$, so folgt 1. unmittelbar aus Satz 
	30.7. Außerdem ist
	\begin{equation*}
		\sum\limits_{j=1}^\infty\int_M|f|\alpha_j\mathrm{d}a = 
		\lim\limits_{ k \rightarrow \infty }\sum\limits_{j=1}^k\int_M|f|\alpha_j\mathrm{d}a \leq 
		\int|f|\mathrm{d}a < \infty
	\end{equation*}
	Durch die majorisierte Konvergenz folgt (31.2).\\
	Gelten 1. und 2., so folgt durch die Monotone Konvergenz, dass 
	$|f|$ integrierbar und somit (nach Satz 30.7) auch $f$ integrierbar 
	ist.
	
\end{proof}

\begin{definition}[Integral auf Mannigfaltigkeiten]
	Sei $M\subset\mathbb{R}^n$ eine Mannigfaltigkeit und $\mathcal{U}$ 
	eine offene Überdeckung von $M$ mit Kartengebieten. Dann heißt 
	$f:M\rightarrow\mathbb{R}$ \textbf{integrierbar auf M}, falls eine
	Zerlegung der Eins existiert, die $\mathcal{U}$ untergeordnet ist, 
	so dass
	\begin{enumerate}
		\item 	$f\alpha_j$ für alle $j\in\mathbb{N}$ integrierbar
				 auf $M$ ist.
		\item 	$\sum\limits_{j=1}^\infty\int_M|f|\alpha_j\mathrm{d}a<\infty$
	\end{enumerate}	 
	Wir nennen dann 
	\begin{equation}
		\int_Mf\mathrm{d}a\coloneqq
		\sum\limits_{j=1}^\infty\int_Mf\alpha_j\mathrm{d}a
	\end{equation}
	auch \textbf{Integral auf M}.
\end{definition}

\begin{satz}[Rechtfertigung der Integralbegriffe]
	Die Definitionen der Integrierbarkeit von $f$ auf $M$ und des 
	Integrals $\int_Mf\mathrm{d}a$ sind unabhängig von der 
	konkreten Überdeckung $\mathcal{U}$ und der Zerlegung der Eins $\{\alpha_j\}$.
\end{satz}

\emph{Beweisidee:} Sei $f:M\rightarrow\mathbb{R}$ integrierbar auf 
$M$ und $\mathcal{U}$ und $\{\alpha_j\}$ wie in der Definition. Dann 
zeigen für weitere Überdeckungen $\tilde{\mathcal{U}}$ und ihre 
untergeordneten Zerlegungen der Eins $\{\alpha_j\}$, dass 
\begin{equation*}
	\sum\limits_{j=1}^\infty\int_M\alpha_j\mathrm{d}a = 
	\sum\limits_{j=1}^\infty\int_M\tilde{\alpha}_j\mathrm{d}a
\end{equation*}
Dazu verwenden wir Lemma 31.2.\\
\linebreak

\begin{definition}[Integral auf Teilmengen von Mannigfaltigkeiten]
	Ist $M\subset\mathbb{R}^n$ eine Mannigfaltigkeit und $A$ eine 
	Teilmenge derselben, so heißt die Funktion $f:A\rightarrow\mathbb{R}$ 
	integrierbar auf $A$, falls
	\begin{equation*}
		f_A\coloneqq\left\{\begin{matrix}
		f \ \ \mathrm{auf\ }A \\ 0 \ \ \mathrm{sont}
		\end{matrix}\right.
	\end{equation*}
	auf $M$ integrierbar ist.
	Wir setzen dann
	\begin{equation*}
		\int_Af\mathrm{d}a\coloneqq\int_Mf_A\mathrm{d}a
	\end{equation*}
	als das Integral von $f$ auf $A$. Dasselbe $A$ heißt dann auch
	(endlich) \textbf{messbar in M}, falls $f\equiv 1$ integrierbar 
	auf $A$ ist. Natürlich nennen wir dann
	\begin{equation*}
		v_d(A)\coloneqq\int_A\mathrm{d}a
	\end{equation*}
	den d-dimensionalen Inhalt oder das d-dimensionale Maß von A.
\end{definition}
Der feine Zusatz ''endlich'' ist zu beachten, denn unter dem normalen 
Lebesque-Maß sind wir auch $\mathcal{L}^n(A)=\infty$ gewohnt. Hier 
ist $v_d(A)$ jedoch strikt endlich.

\begin{definition}[d-Nullmengen]
$A\subset M$ heißt \textbf{d-Nullmenge}, falls $v_d(A)=0$ ist. 
d-Nullmengen auf $M$ entsprechen natürlich $\mathcal{L}$-Nullmengen 
im Parameterbereich.
\end{definition}

\begin{satz}
Ist $M\subset\mathbb{R}^n$ eine Mannigfaltigkeit und $A\subset M$ 
kompakt bezüglich $M$, sowie $f:A\rightarrow\mathbb{R}$ eine stetige 
Abbildung. Dann ist $f$ auf $A$ integrierbar.
\end{satz}
Wir wissen, dass $A$ kompakt ist, wenn das Urbild $\varphi^-1(A)$ kompakt 
ist, da $\varphi$ ja homöomorph ist.\\
\emph{Beweisidee:} Da $A$ kompakt ist, gibt es eine endliche 
Überdeckung mit Kartengebieten. Da $f(\varphi(.))\sqrt{g^\varphi(.)}$ 
stetig ist, ist es integrierbar auf kompakten Mengen.\\
\linebreak
Wir können nun, da wir statt nur auf Kartengebieten, auf ganzen 
Mannigfaltigkeiten integrieren können, die Integraleigenschaften 
aus Satz 30.7 neu formulieren:
\begin{satz}[Eigenschaften des Integrals über Mannigfaltigkeiten]
	Sei $M\subset\mathbb{R}^n$ eine Mannigfaltigkeit, und 
	$f,g,f_k:M\rightarrow\mathbb{R}$ Funktionen, die von der 
	Mannigfaltigkeit nach $\mathbb{R}$ abbilden, dann gelten die 
	folgenden Beziehungen:
	\begin{enumerate}
		\item 	$f$ ist einerseits genau dann auf $M$ integrierbar, 
				wenn $|f|$ integrierbar auf $M$ ist und auch genau dann, 
				wenn es $f^+$ und $f^-$ sind.
		\item 	Sind $f$ und $g$ integrierbar auf $M$ und $c\in\mathbb{R}$, 
				so ist
				\begin{equation*}
					\int_Mcf\pm g\mathrm{d}a = 
					c\int_Mf\mathrm{d}a\pm\int_Mg\mathrm{d}a
				\end{equation*}
		\item 	Sind $f$ und $g$ integrierbar und $g$ beschränkt auf $M$, 
				dann ist auch $f\circ g$ integrierbar auf $M$.
		\item 	Sind $f$ und $g$ integrierbar und $f\leq g$ auf $M$, 
				dann dies auch für die Integrale:
				\begin{equation*}
					\int_Mf\mathrm{d}a\leq\int_Mg\mathrm{d}a
				\end{equation*}
		\item	\textbf{Monotone Konvergenz:} Seien alle $f_k$ 
				integrierbar und $f_1\leq f_2\leq ...$ auf $M$, 
				außerdem sei die Folge der Integrale $\left(\int_Mf_k\mathrm{d}a\right)$ 
				beschränkt sowie $f(u)=\lim\limits_{k\rightarrow\infty}f_k(u) \ \ \forall u\in M$, 
				dann ist $f$ integrierbar auf $M$ mit
				\begin{equation*}
					\int_Mf\mathrm{d}a = 
					\lim\limits_{k\rightarrow\infty}\int_Mf_k\mathrm{d}a
				\end{equation*}
		\item 	\textbf{Majorisierte Konvergenz:} Seien $g$ und alle $f_k$ 
				integrierbar und weiterhin $f_k\leq g \ \ \forall k$ auf $M$. 
				Außerdem möge $f(u) = \lim\limits_{k\rightarrow\infty}f_k(u) \ \ \forall u\in M$ 
				sein. Dann ist $f$ integrierbar auf $M$ mit
				\begin{equation*}
					\int_Mf\mathrm{d}a = 
					\lim\limits_{k\rightarrow\infty}\int_Mf_k\mathrm{d}a
				\end{equation*}
	\end{enumerate}
\end{satz}
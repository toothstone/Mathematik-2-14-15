\documentclass[a4paper,12pt,portrait]{book}	  
\title{Mathematik f\"ur Physiker I}
\author{Friedemann Schuricht\\ \\ \"ubertragen von\\Lukas K\"orber und Friedrich Zahn}
\date{Wintersemester 2014/2015}


\usepackage[ngerman]{babel}
\usepackage[utf8]{inputenc}
\usepackage[titles]{tocloft}
\usepackage{amsmath}
\usepackage{mathtools}
\usepackage{amsthm}	 
\usepackage{graphicx}      
\usepackage{fancyhdr}
\usepackage{xcolor}
\usepackage[upright]{fourier}
\usepackage[frak=mma]{mathalfa}
\usepackage{sectsty}
\usepackage{lipsum}
\usepackage{bm}


%layout options
\sectionfont{\color{hblue}}
\chapterfont{\color{dblue}}
\pagestyle{headings}
\setlength\parindent{0pt}
\numberwithin{equation}{section}

\renewcommand\theequation{\maybe{\arabic{chapter}}\arabic{section}.\arabic{equation}}
\DeclareRobustCommand\maybe[1]{\ifnum#1=\value{chapter}\relax\else\uppercase\expandafter{\romannumeral#1}.\fi}

\renewcommand*\cftchapnumwidth{0.9cm}
\renewcommand{\thechapter}{\Roman{chapter}} 

\renewcommand*\cftsecnumwidth{0.7cm}
\renewcommand*\thesection{\arabic{section}}

%differential operators
\newcommand{\diff}[2]{\frac{\mathrm{d}{#1}}{\mathrm{d}{#2}}}
\newcommand{\ddiff}[2]{\frac{\mathrm{d}^2{#1}}{\mathrm{d}{#2}^2}}
\newcommand{\dddiff}[2]{\frac{\mathrm{d}^3{#1}}{\mathrm{d}{#2}^3}}
\newcommand{\ndiff}[2]{\frac{\mathrm{d}^n{#1}}{\mathrm{d}{#2}^n}}
\newcommand{\pdiff}[2]{\frac{\partial{#1}}{\partial{#2}}}

%hilbert operators
\newcommand{\bra}[1]{\left\langle #1 \right|}
\newcommand{\ket}[1]{\left| #1 \right\rangle}
\newcommand{\lara}[1]{\left\langle #1 \right\rangle}
\newcommand{\scap}[2]{\left\langle #1 \ \middle| \ #2\right\rangle}
\newcommand{\exlara}[3]{\left\langle #1 \ \middle| \ #2 \ \middle| \ #3\right\rangle}

%various operators
\newcommand{\graph}{\text{graph\ }}
\newcommand{\rang}{\text{rang\ }}
\renewcommand{\sup}{\text{sup\ }}
\renewcommand{\inf}{\text{inf\ }}
\renewcommand{\dim}{\text{dim\ }}
\renewcommand{\ker}{\text{ker\ }}
\renewcommand\qedsymbol{$\textbf{q.e.d.}$}
\usepackage{xcolor}
\usepackage{amsthm}	 


\definecolor{dblue}{HTML}{183CE0}
\definecolor{hblue}{HTML}{0E75F7}

\newtheoremstyle{theoremstyle} % name
    {\topsep}                    % Space above
    {}                    % Space below
    {\upshape}                   % Body font
    {}                           % Indent amount
    {\sffamily\bfseries}                   % Theorem head font
    {}                          % Punctuation after theorem head
    {.5em}                       % Space after theorem head
    {\thmname{#1}\thmnumber{ #2}\thmnote{ (#3)}}  % Theorem head spec (can be left empty, meaning ‘normal’)
    
    \newtheoremstyle{otherstyle} % name
    {\topsep}                    % Space above
    {50px}                    % Space below
    {\upshape}                   % Body font
    {}                           % Indent amount
    {\sffamily\bfseries}                   % Theorem head font
    {}                          % Punctuation after theorem head
    {.5em}                       % Space after theorem head
    {\thmname{#1}\thmnumber{ #2}\thmnote{ (#3)}}  % Theorem head spec (can be left empty, meaning ‘normal’)

\theoremstyle{theoremstyle}
\newtheorem{theo}{Theorem}[section]	% please use "theorem"
\newtheorem{sa}[theo]{Satz}			% please use "satz"
\newtheorem{lem}[theo]{Lemma}		% please use "lemma"
\theoremstyle{otherstyle}
\newtheorem{folgerung}[theo]{Folgerung}
\newtheorem*{definition}{Definition}
\newtheorem{beispiel}{Beispiel}[section]
\newtheorem*{beispiel*}{Beispiel}

\newenvironment{theorem}
	{\par\nobreak\color{dblue}\noindent\rule{420pt}{2pt}\vspace{5pt}\begin{theo}}
	{\end{theo}\rule{50pt}{1pt}\vspace{50px}\color{black}\par}
   
\newenvironment{satz}
	{\par\nobreak\color{dblue}\noindent\rule{420pt}{2pt}\vspace{5pt}\begin{sa}}
	{\end{sa}\rule{50pt}{1pt}\vspace{50px}\color{black}\par}
   
\newenvironment{lemma}
	{\par\nobreak\color{dblue}\noindent\rule{420pt}{2pt}\vspace{5pt}\begin{lem}}
	{\end{lem}\rule{50pt}{1pt}\vspace{50px}\color{black}\par}
	
\begin{document}
\maketitle
\tableofcontents
\setcounter{chapter}{7}
\chapter*{Überblick}
Diese Vorlesung wird sich mit folgenden Tehmen befassen:
    \begin{enumerate}
    \item \textbf{Integration auf Mannigfaltigkeiten}
    \item \textbf{Differenzialgleichungen}, sowohl gewöhnlich, als auch partiel
    \item \textbf{Funktionalanalysis} in Banach- und Hilberträumen (insbesondere
    unendlich dimensionale Räume z.B. von Folgen und Funktionen)
    \item \textbf{Funktionstheorie}, der Theorie von komplexwertigen Funktionen
    und z.B. $\mathbb{C}$-Differenzierbarkeit
    \end{enumerate}

\chapter{Integration auf Mannigfaltigkeiten}
\emph{Literaturtipp:} Königsberger Analysis 2, Springer
\setcounter{section}{28}

\section{Mannigfaltigkeiten}
Sei $\varphi\in C^q(V,\mathbb{R}^n)$ mit $q\in\mathbb{N}_{\geq 1}$, 
also $q$-fach stetig differenzierbar, wobei $V\subset\mathbb{R}^d$ offen ist, 
dann heißt $\varphi$ \textbf{regulär}, falls
    \begin{equation}
    \varphi'(x):\mathbb{R}^d\rightarrow\mathbb{R}^n \ \text{regulär (d.h. injektiv)}
    \end{equation}
Falls $\varphi$ regulär für alle $x\in V$ ist, heißt es auch 
\textbf{regulär auf V} beziehungsweise \\
\textbf{reguläre $C^q$-Parametrisierung} (manchmal auch $C^q$-Immersion). \\
$V$ ist dann der \textbf{Parameterbereich} von $\varphi$.\\
\emph{Bemerkung:} $\varphi(V)$ wird selten auch \textbf{Spur} von $\varphi$ genannt.\\
\linebreak
\linebreak
Aus der Linearen Algebra wissen wir, dass aus (29.1) sofort 
    \begin{equation}
    d\leq n
    \end{equation}
folgt. Dies sei in Kapitel VIII immer erfüllt! (29.2) ist außerdem äquivalent dazu, dass 
$\rang \varphi'(x)=d$.\\
    \includegraphics[scale=0.5]{pictures/001-01}

\begin{beispiel}[reguläre Kurven $\varphi:I\subset\mathbb{R}\rightarrow\mathbb{R}^n$]
Dabei ist $I$ offen und der Tagentialvektor nirgendwo identisch mit dem Nullvektor, also 
$\varphi'(x)\neq 0$

\begin{enumerate}
    \item $\varphi:(0,2\pi)\rightarrow\mathbb{R}^2$ mit $\varphi(t)=\begin{pmatrix}
        \cos kt \\ \sin kt
    \end{pmatrix}$ und $k\in\mathbb{N}_{\geq 2}$\\

\begin{center}\includegraphics[scale=0.3]{pictures/001-02}\\ 
\end{center}

Der Einheitskreis wird hier k-mal durchlaufen. 
Da $\varphi'(x)\neq 0$, ist $\varphi$ regulär.
\item $\varphi(-\pi,\pi)\rightarrow\mathbb{R}^2$ mit $\varphi(t)=(1+2\cos t)
    \begin{pmatrix}
    \cos t \\ \sin t
    \end{pmatrix}$\\

\begin{center}
\includegraphics[scale=0.3]{pictures/001-03}\\
\end{center}

$\varphi(\pm\frac{2\pi}{3})=
    \begin{pmatrix}
    0 \\ 0
    \end{pmatrix}
$, $ 
\varphi(0)=
    \begin{pmatrix}
    3 \\ 0
    \end{pmatrix}
$\\$
    \begin{pmatrix}
    1 \\ 0
    \end{pmatrix} $ 
\ gehört \textbf{nicht} zur Kurve ("$=\varphi(\pm\pi)$") und $\varphi$ ist regulär.

\item $\varphi:(-1,1)\rightarrow\mathbb{R}^2$ \ mit \ $\varphi(t)=\begin{pmatrix}
    t^3 \\ t^2
\end{pmatrix}$ \ ist wegen $\varphi'(0)=0$ \ \textbf{nicht} regulär\\

    \begin{center}
    \includegraphics[scale=0.3]{pictures/001-04}
    \end{center}

\end{enumerate}

\end{beispiel}
\ \linebreak

\begin{beispiel}[Parametrisierung von Graphen]
Sei $f\in C^q(V,\mathbb{R}^{n-d})$,\\
$V\subset\mathbb{R}^d$. Betrachtet wird $\varphi:V\rightarrow\mathbb{R}^n$ mit $\varphi(x)=(x,f(x))$\\
    \begin{center}
    \includegraphics[scale=0.5]{pictures/001-05}\\
    \end{center}
$\varphi$ ist regulär, da offenbar $\varphi\in C^q(V,\mathbb{R}^n)$ 
und $\varphi'=
    \begin{pmatrix} 
    id^d \\ 
    f'(x)
    \end{pmatrix}     
\in \mathbb{R}^{n \times d}$ \ ist.\\
\linebreak
\end{beispiel}

Es folgt eine Wiederholung zur \textbf{Relativtopologie} (vgl. Kapitel 14). Wir wissen, dass $U\subset M$ genau dann offen bezüglich $M$ ist, wenn es ein $\tilde{U}\subset\mathbb{R}^n$ gibt, dass offen ist, und das $U=\tilde{U}\cap M$ erfüllt. Später wird $M$ eine Mannigfaltigkeit sein und wir werden untersuchen, was in ihr offen ist.\\

\begin{center}
\includegraphics[scale=0.5]{pictures/001-06}\\
\end{center}
Auf dieser Grundlage lässt sich auch der Begriff der \textbf{Umgebung} definieren:\\
$U\subset M$ heißt nämlich genau dann Umgebung von $u\in M$ bezüglich $M$, wenn es ein bezüglich $M$ offenes $U_0\subset M$ gibt, in dem $u$ liegt und das Teilmenge von $U$ ist.\linebreak\linebreak
\newpage
\textbf{\textsf{Beispiel für}} \ $M\subset\mathbb{R}^n$.\\

\begin{center}
\includegraphics[scale=0.5]{pictures/001-07}\\
\end{center}
\hspace{70pt} offen bzgl. M \hspace{70pt} nicht offen bzgl. M\\

\begin{definition}[Mannigfaltigkeiten]
Wir nennen $M\subset\mathbb{R}^n$ eine \textbf{d-dimensionale $C^q$-Mannigfaltigkeit} ($q\in\mathbb{N}_{\geq 1}$), falls

\begin{enumerate}
\item es für alle $u\in M$ eine (offene) Umgebung $U$ von $u$ bezüglich $M$ gibt und
\item es eine reguläre $C^q$-Parametrisierung $\varphi:V\subset\mathbb{R}^d\rightarrow\mathbb{R}^n$ ($V$ ist offen) existiert, die homöomorph ist und in die Mannigfaltigkeit abbildet (also \  $\varphi(V)=U$).
\end{enumerate}

\emph{Wiederholung: Eine stetige Abbildung heißt homöomorph, falls eine Umkehrabbildung existiert, die auch stetig ist.}\\
\end{definition}

In der Literatur wird $M$ auch manchmal als $C^q$-\emph{Unter}mannigfaltigkeit bezeichnet. Wir werden jedoch später zeigen, dass die verschiedenen Definitionen von Mannigfaltigkeiten gleichwertig sind.\\
Da ab jetzt immer hauptsächlich $C^1$-Mannigfaltigkeiten auftauchen werden, werden wir diese in Zukunft einfach "Mannigfaltigkeiten" nennen.\\

\begin{center}
\includegraphics[scale=0.5]{pictures/001-08}\\
\end{center}

Die Umkehrabbildung $\varphi^{-1}$ beziehungsweise ($\varphi^{-1}, U$) nennt man die \textbf{Karte} von $M$ um $u\in M$, wobei $U$ das zugehörige \textbf{Kartengebiet}, $\varphi$ selbst die Parametrisierung und $V$ der Parameterbereich ist.\\
Karten können eine Mannigfaltigkeit jedoch nur lokal beschreiben. Aus diesem Grund führt man den Begriff des Atlas, der eine globale Beschreibung ermöglicht, ein:\\
Die Menge $\{\varphi^{-1}_\alpha | \alpha\in A\}$ heißt \textbf{Atlas} der Mannigfaltigkeit M, falls die zugehörigen Kartengebiete $U_\alpha$ jene vollständig überdecken.\\
\linebreak
Weiterhin wichtig ist der Begriff der sogenannten \textbf{Einbettung}, bei der es sich um eine reguläre Parametrisierung handelt, die homöomorph ist. Wir vereinbaren, dass es sich im folgenden bei allen Parametrisierungen von Mannigfaltigkeiten stets um Einbettungen handelt.
\begin{beispiel}[Beweise bitte Selbstudium]\ \\

\begin{enumerate}
\item Der Kreis aus Beispiel 1.1 ist eine 1-dimensionale $C^\infty$-Mannigfaltigkeit, obwohl der Kreis k-fach durchlaufen wird. Der Atlas benötigt mindestens zwei Karten.
\item Die Kurven aus Biespiel 1.2 und 1.3 sind keine Mannigfaltigkeiten, da $\varphi$ nicht überall homöomorph ist.
\item Jedes offene $M\subset\mathbb{R}^n$ ist eine n-dimensionale $C^\infty$-Mannigfaltigkeit mit $\{id\}$ als Atlas.
\end{enumerate}
\end{beispiel}

\begin{beispiel} Sei $M:=\graph f$ aus Beispiel 2. Offenbar ist 
$\varphi:V\subset\mathbb{R}^d\rightarrow M\subset\mathbb{R}^n$ 
eine Einbettung. Das macht $M$ zu einer d-dimensionalen $C^q$-Mannigfaltigkeit.
\end{beispiel}

\begin{beispiel}
Sei $f:D\subset\mathbb{R}^n\rightarrow\mathbb{R}^{n-d}$ ($D$ offen) q-fach stetig differenzierbar für $q\geq 1$. Offenbar ist

\begin{equation}
\rang f'(u)=n-d \ \ \ \ \ \ \ \forall u\in D
\end{equation}
Wir nennen $M=\{u\in D \ | \ f(n)=0\}$ die Niveaumenge von $f$\\

\begin{center}\includegraphics[scale=0.5]{pictures/001-09}\\
\end{center}

Fixieren wir $\tilde{u}=(\tilde{x},\tilde{y})=(x_1,...,x_d,y_1,...,y_{n-d})\in M$
, so sehen wir mit (29.3) und eventuellen Koordinatenvertauschungen, dass 
$f(\tilde{x},\tilde{y})$ regulär ist. Der \emph{Satz über implizite Funktionen}
sichert uns nun, dass es eine Umgebung $V\subset\mathbb{R}^d$ von $\tilde{x}$, 
eine Umgebung $W\subset\mathbb{R}^{n-d}$ von $\tilde{y}$ und ein 
$\psi:V\rightarrow W\in C^q(V,W)$ gibt, das $(x,\psi(x))\in M$ erfüllt und homöomorph ist.\\
Es folgt, dass $\varphi:V\subset\mathbb{R}^d\rightarrow\mathbb{R}^n$ mit
$\varphi(x)=(x,\psi(x))$ eine homöomorphe, reguläre Einbettung  und $\varphi(V)$ 
Umgebung von $\tilde{u}\in M$ bezüglich von $M$ ist. Daraus können wir nun schließen, 
dass $M$ eine d-dimensionale $C^q$-Mannigfaltigkeit ist.\\
\linebreak
\emph{Bemerkung: $M=\graph f$ und $M=\{f=0\}$ sind grundlegende Konstruktionen 
für Mannigfaltigkeiten. \textbf{Lokal} ist jede Mannigfaltigkeit von dieser Gestalt!}
\end{beispiel}

\begin{satz}[lokale Darstellung einer Mannigfaltigkeit als Graph]
\mbox{} \\
Sei $M \subset \mathbb{R}^{n}$ eine d-dimensionale $C^{q}$-Mannigfaltigkeit. \\
$\Longleftrightarrow \forall u \in M \subset \mathbb{R}^n $ existiert eine Umgebung
$U$ von $u$  bezüglich $M$, $W \subset \mathbb{R}^n $ offen, 
$f \in C^q (W, \mathbb{R}^{n-d})$ und eine Permutation $\pi$ von Koordinaten in
$\mathbb{R}^n $ mit $ \psi (W) = U $ für $ \psi (v) \coloneqq \psi (v, f(v)) 
\forall v \in W $ (das heißt $U$ ist Graph von f).
\end{satz}

\textbf{somit:} $M$ ist $C^q$-Mannigfaltigkeit genau dann, wenn $M$ lokaler Graph
einer $C^q$-Funktion $f$ ist (vergleiche Beispiel 2 und 4).

\begin{proof}

$"\Leftarrow"$: folgt aus Beispielen 2 und 4. \\
$"\Rightarrow"$: fixiere $\tilde{u} \in M $, sei $\varphi : \tilde{V} \in \mathbb{R}^d
\rightarrow \tilde{U} \subset \mathbb{R}^n $ und der zugehörige $C^q$-Parameter 
$\tilde{u} = \varphi (\tilde{x})$ \\
$\varphi' (x)$ ist regulär $\xRightarrow{\text{ evtl. $\pi$ der Zeilen }} \varphi_I ' 
(\tilde{x}) \subseteq \mathbb{R}^{d \times d} $ ist regulär für $\varphi (x) =
    \begin{pmatrix}
    \varphi_I (x) \left[ \in \mathbb{R}^d \right]\\
    \varphi_{II} (x) \left[ \in \mathbb{R}^{n-d} \right]
    \end{pmatrix}
$ \\
Zerlege $u = \pi (v,w) $ mit $v \in \mathbb{R}^d $, 
d.h. $\tilde{u} = \pi (\tilde{v},\tilde{w})$\\
\begin{center}
\includegraphics[scale=0.5]{pictures/MA2_0010}\\
\end{center}
$\xRightarrow{\text{Theorem über inverse Fkt.}} \exists V \subset \tilde{V} $ offen,
$\tilde{x} \in V, W \subset \mathbb{R}^d $ offen, $\tilde{v} \in W $
mit $\varphi_I ^{-1} : W \rightarrow V $ Homöomorphismus und $C^q$-Abbildung,
$\varphi_I ^{-1} (\tilde{v}) = \tilde{x} $ \\
mit $f(v) \coloneqq \varphi_{II} \left(\varphi_I ^{-1} (v)\right) \forall v \in W $ ist 
$f \in C^q \left(W, \mathbb{R}^{n-d}\right) $ \\
und $\psi (v) \coloneqq \varphi \left(\varphi_I ^{-1} (v) \right) =
\left( \varphi_I \left(\varphi_I ^{-1} (v) \right( ,
\varphi_{II} \left(\varphi_I ^ {-1} (v) \right) \right) = \pi \left(v, f(v)\right) $ \\
$\Rightarrow \psi (\tilde{v}) = \pi (\tilde{v}, \tilde{w}) = \tilde{u},
\psi (w) = \varphi (v) \in M \\
\varphi : \tilde{V} \rightarrow \tilde{U} $ ist Homöomorphismus \\
$\Rightarrow \varphi (v) $ ist offen in $M$ \\
$ \Rightarrow U \coloneqq \psi (W) $ ist offen bezüglich $M$
$\Rightarrow U $ ist Umgebung von $\tilde{u} $ bezüglich $M$ \\
$\xRightarrow{\tilde{u} \text{ beliebig}} $ Behauptung. 

\end{proof}

\begin{satz}[Charakterisierung von Mf mit umgebendem Raum]
\mbox{} \\
$M \subset \mathbb{R}^n $ sei d-dimensionale $C^q$-Mannigfaltigkeit. \\
$\Longleftrightarrow \forall u \in M $ existiert eine Umgebung $\tilde{U}$ von $u$ 
bezüglich $\mathbb{R}^n$, \\
$\tilde{V} \subset \mathbb{R}^n $, 
$\tilde{\psi}: \tilde{U} \rightarrow \tilde{V} $ wobei $\tilde{\psi} $ 
ein $C^q$-Diffeomorphismus ist und 
    \begin{equation*}
    \tilde{\psi} \left( \tilde{U} \cap M \right) =
    \tilde{V} \cap \left( \mathbb{R}^d \times {0} \right) 
    \end{equation*}
\begin{center}
\includegraphics[scale=0.5]{pictures/MA2_0011}\\
\end{center}
\end{satz}

\textbf{Bemerkung:} Diese Charakterisierung von Mannigfaltigkeiten benutzt den umgebenden Raum und wird häufig als Definition der Mannigfaltigkeit benutzt.    
    
\begin{proof}
$"\Leftarrow"$: $\psi $ eingeschränkt auf $\tilde{U} \cap M $ liefert Karten 
$\Rightarrow$ Behauptung. \\
$"\Rightarrow"$: fixiere $\tilde{u} \in M $, wähle $\tilde{U} \subset M $,
$W \subset \mathbb{R}^d $, $f \in C^q \left( W, \mathbb{R}^{n-d} \right)$ \\
gemäß Satz 29.1 oBdA $\pi = $ \textit{id} \\
zerlege $u = (v,w) \in \mathbb{R}^d \times \mathbb{R}^{n-d}$, 
$\tilde{u} = \left( \tilde{v}, f \left( \tilde{v} \right) \right) $ \\
sei $\hat{U} \coloneqq W \times \mathbb{R}^{n-d} \eqqcolon \hat{V} $,
liefert "Zylinder" aus $U$ und $W$ in Beweis zu Satz 29.1 \\
sei $\tilde{\varphi}: \hat{V} \rightarrow \hat{U} $ mit
$\tilde{\varphi} (v,w) \coloneqq (v, f(v) + w) \Rightarrow \tilde{\varphi} \in C^q $ \\
$\tilde{\varphi}' \left( \tilde{v}, 0 \right) =
    \begin{pmatrix}
    \textit{id}_d & 0 \\
    f'(v)         & \textit{id}_{n-d}
    \end{pmatrix}
$ ist regulär \\
$\xRightarrow{\text{Satz ü. inverse Fkt.}} \exists $ 
Umgebung $ \tilde{U} \subset \hat{U}$ von $\tilde{U}$,
Umgebung $ \tilde{V} \subset \hat{V} $ von $ \left( \tilde{v}, 0 \right) $, sodass \\
$\tilde{\psi} \coloneqq \tilde{\varphi}^{-1} \in C^q \left( \tilde{U}, \tilde{V} \right) $
exisitiert. \\
wegen $\tilde{\varphi} \left( \tilde{V} \cap \left( \mathbb{R}^d \times {0} \right) \right)
= \tilde{U} \cap M $ folgt die Behauptung.
\end{proof}

\begin{folgerung}
Sei $M \subset \mathbb{R}^n$ d-dimensionale $C^q$-Mannigfaltigkeit \\
und $\varphi: V \subset \mathbb{R}^d \rightarrow U \subset M $ Parameter um $u \in M $ \\
$\Longrightarrow \exists \tilde{U}, \tilde{V} \subset \mathbb{R}^n $ offen und 
$\tilde{\varphi} : \tilde{V} \rightarrow \tilde{U} $ 
mit $ U \subset \tilde{U}, V \times {0} \subset \tilde{V} $, \\
$\tilde{\varphi} $ ist $C^q$-Diffeomorphismus und
$\tilde{\varphi} (x, 0) = \varphi (x) \forall x \in V $
\end{folgerung}

\begin{proof}
Folgt aus Beweisen von Satz 29.1 und 29.2
\end{proof}

\begin{theorem}[lokale Darstellung von Mf als Niveaumenge]
\mbox{} \\
$M \subset \mathbb{R}^n $ sei d-dimensionale $C^q$-Mannigfaltigkeit. \\
$\Longleftrightarrow \forall u \in M $ existiert eine Umgebing $\tilde{U}$ von $u$
bezüglich $\mathbb{R}^n$ und \\
$f \in C^q \left( \tilde{U}, \mathbb{R}^{n-d} \right)$ 
mit $\textit{rang } f' (u) = n-d $ und \\
$\tilde{U} \cap M = \left\lbrace \tilde{u} \in \tilde{U} | f (\tilde{u}) = 0 \right\rbrace $
\end{theorem}

\textbf{somit:} $M$ ist eine $C^q$-Mannigfaltigkeit genau dann, 
wenn $M$ die lokale Niveaumenge einer $C^q$-Funktion $f$ ist. \\

\textbf{Bemerkung:} $c \in \mathbb{R}^{n-d} $ heißt \textit{regulärer Wert} von
$f \in C^q \left( \tilde{U}, \mathbb{R}^{n-d} \right) $, \\
$\tilde{U} \subset \mathbb{R}^n $
offen, falls $\textit{rang } f' (u) = n-d \forall u \in \tilde{U} $ mit $f(u) = c $ \\
Folglich ist $M \coloneqq \left\lbrace u \in \tilde{U} | f(u) = c \right\rbrace $ 
eine d-dimensionale $C^q$-Mannigfaltigkeit, falls $c$ ein regulärer Wert von $f$ ist.

\begin{proof}
$"\Leftarrow":$ gemäß Bsp. 5 erhält man lokale Parametriesierung \\
$\Rightarrow$ Behauptung. \\
$"\Rightarrow":$ fixiere $\tilde{u} \in M$, 
wähle $\tilde{U}, \tilde{V} \subset \mathbb{R}^n, 
\tilde{\psi}: \tilde{U} \rightarrow \tilde{V} $ gemäß Satz 29.2 \\
sei $f \coloneqq \left( \tilde{\psi}_{d+1}, \ldots , \tilde{\psi}_n \right)$,
offenbar $f \in C^q \left(  \tilde{U}, \mathbb{R}^{n-d} \right) $ \\
mit $\tilde{\psi}$ aus dem Beweis zu Satz 29.2:
$\tilde{\psi}' (\tilde{u}) = \tilde{\varphi}' (\tilde{v}, 0)^{-1} $ ist regulär \\
$\Rightarrow f'(\tilde{u}) $ hat vollen Rang, d.h. $\textit{rang } f'(\tilde{u}) = n-d $ \\
nach Konstruktion $ \left\lbrace u \in \tilde{U} | f(u) = 0 \right\rbrace = U \cap M
\Rightarrow $ Behauptung.
\end{proof}

\begin{lemma}[Kartenwechsel] 
\mbox{} \\
Sei $M \in \mathbb{R}^n $ d-dimensionale Mannigfaltigkeit \\
und $\varphi_1^{-1}, \varphi_2^{-1} $ Karten mit zugehörigem Kartengebiet 
$U_1 \cap U_2 \neq \emptyset $ \\
$\Longrightarrow 
\varphi_2^{-1} \circ \varphi_1 : \varphi_1^{-1} \left( U_1 \cap U_2 \right)
\rightarrow \varphi_2^{-1} \left( U_1 \cap U_2 \right) $ ist $C^q$-Diffeomorphismus.\\
\begin{center}
\includegraphics[scale=0.5]{pictures/MA2_0012}\\
\end{center}
\end{lemma}

\begin{proof}
Ersetze $\varphi_1, \varphi_2 $ mit $\tilde{\varphi}_1, \tilde{\varphi}_2 $
gemäß Folgerung 29.3 \\
$\Rightarrow$ Einschränkung von $\tilde{\varphi}_2^{-1} \circ \tilde{\varphi}_1 $
liefert Behauptung. 
\end{proof}

\begin{definition}
Sei $M \subset \mathbb{R}^n $ d-dimensionale Mannigfaltigkeit. \\
Ein Vektor $v \in \mathbb{R}^n $ heißt \textbf{Tangentialvektor} in $u \in M $ an $M$, \\
falls eine stetig differenzierbare Kurve 
$\gamma: (-\delta, \delta) \rightarrow M (\delta > 0) $ exisitiert mit \\
$\gamma (0) = u $ und $\gamma' (0) = v $. \\
Die Menge aller Tangentialvektoren $T_uM$ heißt Tangentialraum.\\
\begin{center}
\includegraphics[scale=0.5]{pictures/MA2_0013}\\
\end{center}
\end{definition}

\begin{satz}
Sei $M \in \mathbb{R}^n $ eine d-dimensionale Mannigfaltigkeit, \\
$u \in M $, $ \varphi : V \rightarrow U $ der zugehörige Parameter um $u$ \\
$\Longrightarrow T_uM $ ist d-dimensionaler $( \mathbb{R}-) $ Vektorraum und \\
    \begin{equation}
    T_uM = \underbrace{\varphi'(x)}_{L \left( \mathbb{R}^d, \mathbb{R}^n \right) }
    \left( \mathbb{R}^d \right) \text{ für } x = \varphi^{-1} (u) 
    \end{equation}
wobei $T_uM$ unabhängig vom speziellen Parameter $\varphi$ ist.
\end{satz}

\begin{proof}
Sei $\gamma: (-\delta, \delta) \rightarrow M eine C^1$-Kurve mit $\gamma(0) = u $ \\
$\Rightarrow g \coloneqq \varphi^{-1} \circ \gamma $ ist $C^1$-Kurve, 
$g: (-\delta, \delta) \rightarrow \mathbb{R}^d $ mit $ g(0) = x $ und
    \begin{equation*}
    \gamma' (0) = \varphi' (x) g'(0) \text{, } \varphi' (x) \text{ist regulär.}
    \tag{$\spadesuit$}
    \end{equation*}
Offenb. liefert auch jede $C^1$-Kurve $g$ in $\mathbb{R}^d $ durch $x$
eine $C^1$-Kurve $\gamma$ in $M$ mit $(\spadesuit)$ \\
Die Menge aller Tangentialvektoren $g'(0)$ von $C^1$-Kurven $g$ in $\mathbb{R}^d $
ist offenbar $\mathbb{R}^d $ \\
$\Rightarrow $ 
29.4 $ \xRightarrow{\varphi' (x) \text{ ist regulär}} \textit{dim } T_uM = d $ \\
da $(\spadesuit)$ für jeden Parameter $\varphi$ gilt, ist $T_uM$ unabhängig von $\varphi$.
\end{proof}

\textbf{Bemerkung:}
Man bezeichnet auch $(u, T_uM) \subset M \times \mathbb{R}^n$
als Tangentialraum und
$TM = \bigcup\limits_{U \in M} (u, T_uM) \subset M \times \mathbb{R}^n $
als Tangentialbündel.

\begin{beispiel}
Sei $M \subset \mathbb{R}^n $ offen  \\
$\Rightarrow $ $M$ ist ist n-dimensionale Mannigfaltigkeit und 
$T_uM = \mathbb{R}^n \forall u \in M $
\end{beispiel}

\begin{definition}
Sei $M \subset \mathbb{R}^n $ d-dimensinale Mannigfaltigkeit. \\
Ein Vektor $w \in \mathbb{R}^n $ heißt \textbf{Normalenvektor} in $u \in M $ an $M$, falls\\
$\langle w,v \rangle = 0 \forall v \in T_uM $
(d.h. $w \bot v \forall v \in T_uM$) \\
Die Menge aller Normalenvektoren $N_uM = T_uM^{\bot} $ heißt \\
\textbf{Normalenraum} von $M$ in $u$. 
\end{definition}
\begin{satz}
Sei $f\in C^1\left(V,\mathbb{R}^{n-d}\right)$ mit $V$ offen und $c\in\mathbb{R}^{n-d}$ ein regulärer Wert von $f$. Dann ist die Niveaumenge $M\coloneqq\{v\in V \ | \ f(u)=c\}$ eine d-dimensionale Mannigfaltigkeit, für die gilt:
\begin{eqnarray*}
T_uM &=&\{v\in\mathbb{R}^n \ | \ f'(u)v=0\} \ \ \left(=\ker f'(u)\right) \ \ \forall u\in M \\
N_uM&=&\{w\in\mathbb{R}^n \ | \ w=f'(u)^Tv, v\in\mathbb{R}^{n-d}\} \ \ \forall u\in M 
\end{eqnarray*}
Das heißt also, die Spalten von $f'(u)^T$ bilden eines Basis des Normalenraums von $M$.
\end{satz}

\begin{beispiel}
Sei $f=\begin{pmatrix}
f_1 \\ f_2
\end{pmatrix}\in C^1\left(\mathbb{R}^3,\mathbb{R}^2\right)$ und $0\in\mathbb{R}^2$ ein regulärer Wert von $f$. Dann ist
\begin{equation*}
M:=\{u\in\mathbb{R}^3 \ | \underbrace{\ f_1(u)=0, \ f_2(u)=0}_{\text{Schnitt \ zweier \ Flächen}}\}
\end{equation*}
eine 1-dimensionale Mannigfaltigkeit. Dann steht der Gradient $f'_i(u)^T$ senkrecht auf $\{f_i=0\}$. $f'_1(u)^T$ und $f'_2(u)^T$ sind also Normalen zu $M$ in $u$.\\
\textbf{Grafik fehlt}\\
$v$ ist hier die Tangente, da $\scap{f'_i(u)^T}{v}=0$ ist (für $i=1,2$).
\end{beispiel}

\begin{proof}
Wir wissen bereits, dass $M$ eine Mannigfaltigkeit ist. Wählen wir nun die $C^1$-Kurve $\gamma$ auf $M$ mit $\gamma(0)=u$ und $\gamma'(0)=v$, so sehen wir, dass $f(\gamma(t))=c \ \forall t$ ist. $f'(u)$ steht senkrecht auf $v$ (also $\scap{f'(u)}{v}=0$) und da $\rang f'(u)=d$ ist, muss $\dim \ker f'(u)=d$ sein. Damit ist die Behauptung für $T_uM$ (wegen $\dim T_uM)=d$) gezeigt.\\
Nun wählen wir $w=f'(u)^T\tilde{v}$ und $w\in T_uM$. Offenbar ist $\scap{w}{v}=\scap{\tilde{v}}{f'(u)v}=0$. Damit ist $w$ im Normalenraum $N_uM$. Da $\rang f'(u)^T=n-d$  und $\dim N_uM=n-d$, folgt die Behauptung.
\end{proof}

\begin{beispiel}
Wir betrachten $M\coloneqq\mathcal{O}(n)=\{A\in\mathbb{R}^{n\times n}\ |\ A^TA=id \}$. 
Es handelt sich dabei um eine $\frac{n(n-1)}{2}$-dimensionale Mannigfaltigkeit von Matritzen. 
Man nennt $\mathcal{O}$ auch \emph{Orthogonale Gruppe} oder \emph{Lie-Gruppe}. 
Sie bildet die Menge aller orthogonale Matrizen des $\mathbb{R}^{n\times n}$. 
Offenbar ist $id$ das neutrale Element der Gruppe. Der Tangentialraum an diesem Element wird auch \emph{Lie-Algebra} genannt:
\begin{equation*}
T_{id}M=\{B\in\mathbb{R}^{n\times n}\ |\ B+B^T=0\}
\end{equation*}
Dies ist die Menge der schiefsymetrischen Matrizen. Warum ist das so?\\
\linebreak
Sei $f:\mathbb{R}^{n\times n}\rightarrow\mathbb{R}_{sym}^{n\times n}$ mit $f(A)=A^TA$ eine stetig differenzierbare Funktion mit $f'(A)B=A^TB+B^TA$ ($\forall B\in\mathbb{R}^{n\times n}$). Letzterer Ausdruck ist ebenfalls eine symetrische Matrix. 
$id$ ist ein regulärer Wert, denn sei $f(A)=id$ und $S\in\mathbb{R}_{sym}^{n\times n}$. $f'(A)B=S$ hat die Lösung $B=\frac{1}{2}AS$, denn $\frac{1}{2}A^TAS+\frac{1}{2}SA^TA=\frac{1}{2}S+\frac{1}{2}S=S$.
Letzteres hätte man auch in der Grundschule herausbekommen!\\
Aus vorheriger Überlegung folgt, dass $f'(A)$ vollen Rang hat. Aus Satz 4 wissen wir nun, dass die Dimension der Mannigfaltigkeit beträgt:
\begin{equation*}
d=\dim\mathbb{R}^{n\times n}-\dim\mathbb{R}_{sym}^{n\times n}=n^2-\frac{n(n+1)}{2}=\frac{n(n-1)}{2}
\end{equation*}
Satz 7 gewährleistet nun, dass
\begin{equation*}
T_{id}M=\{B\in\mathbb{R}^{n\times n} \ | \ id^TB+B^Tid=0\}
\end{equation*}
\end{beispiel}

\begin{definition}[Hyperflächen und Einheitsnormalenfelder]
Eine $(n-d)$-dimensionale Mannigfaltigkeit $M\subset\mathbb{R}^n$ heißt auch \textbf{Hyperfläche}. Die stetige Abbildung
\begin{equation*}
\nu:M\rightarrow\mathbb{R}^{n}
\end{equation*} 
heißt dann \textbf{Einheitsnormalenfeld}, falls 
\begin{equation*}
\nu(u)\in N_uM\ \mathrm{und\ } ||\nu(u)||=1 \ \ \forall u\in M
\end{equation*}
\textbf{Grafik fehlt.}
\end{definition}

\begin{lemma}
Ist $M\subset\mathbb{R}^n$ eine zusammenhängende Hyperflääche, so existiert entweder \emph{kein} oder \emph{genau zwei} Einheitsnormalenfelder. 
\end{lemma}

\begin{proof}
Als Vorbetrachtung lässt sich sagen, dass wenn $\nu$ ein Einheitsnormalenfeld ist, auch $-\nu$ eins sein muss.\\
Nehmen wir nun an, es gäbe zwei verschiedene Einheitsnormalenfelder $\nu$ und $\tilde{\nu}$. 
Wir können ausnutzen, dass die beiden Felder normiert sind und stellen sofort fest, da $\dim N_uM=1$ ist, dass das Skalarprodukt der beiden nur
\begin{equation*}
s(u)\coloneqq\scap{\nu(u)}{\tilde{\nu}(u)}=\pm 1
\end{equation*}
sein kann. Wir können nun eine weitere wichtige Eigenschaft von $s$ und $M$ ausnutzen.
Da $M$ zusammenhängend und $s$ stetig auf $M$ ist, lässt sich der Zwischenwertsatz zu Hilfe nehmen, 
der uns liefert, dass $s$ für alle $u$ konstant sein \emph{muss}, nämlich entweder $s(u)=1$ oder $s(u)=-1$.
Damit kann $\tilde{\nu}$ nur entweder gleich $\nu$ oder gleich $-\nu$ sein.
\end{proof}

\begin{beispiel}[Möbiusband]
Das wohlbekannte Möbiusband besitzt \emph{kein} Einheitsnormalenfeld.\\
\textbf{grafk fehlt.} 
\end{beispiel}

\begin{beispiel}
Wir betrachten die Konstruktion von Einheitsnormalenfeldern für Hyperflächen $M\coloneqq \{f=0\}$.
Sei $f\in C^1\left(V,\mathbb{R}\right)$ mit $V$ offen und $0$ ein regulärer Wert von $f$. 
Wir können leicht ein Einheitsnormalenfeld definieren und wählen
\begin{equation*}
\nu(u)\coloneqq\frac{f'(u)}{||f'(u)||}
\end{equation*}
\end{beispiel}

Wir wollen nun im folgenden weitere Operationen auf Mannigfaltigkeiten untersuchen.
Im $\mathbb{R}^n$ ist uns das Kreuzprodukt $...\times...$ wohlbekannt.
Es ist zweckmäßig diesen Begriff auf beliege Dimensionen zu verallgemeinern.
\begin{definition}[Äußeres Produkt]
Nehmen wir uns die Vektoren $a_1,a_2,...,a_{n-1}\in\mathbb{R}^n$ und schreiben sie einfach als Spaltenvektoren hintereinander in eine Matrix:
\begin{equation*}
A\coloneqq(a_1|a_2|...|a_{n-1})\in\mathbb{R}^{n\times(n-1)}
\end{equation*}
Wir entfernen nun aus dieser die $k$-te Zeile und nennen sie dann $A_k\in\mathbb{R}^{(n-1)\times(n-1)}$. 
Den Ausdruck
\begin{equation*}
a_1\wedge a_2\wedge...\wedge a_{n-1}\coloneqq\alpha=\begin{pmatrix}
\alpha_1 \\ \alpha_2 \\ \vdots \\ \alpha_k
\end{pmatrix}
\end{equation*}
nennen wir \textbf{äußeres Produkt} von $a_1,a_2,...,a_{n-1}$, wobei
\begin{equation*}
\alpha_k=(-1)^{k-1}\cdot\det A_k
\end{equation*}
\end{definition}
Wir können sofort einige interessante Eigenschaften ablesen, die uns an das bekannte Kreuzprodukt erinnern:
\begin{enumerate}
\item $\alpha$ steht senkrecht auf allen $a_1,a_2,...,a_{n-1}$.
\item Das \emph{Volumen} des von $a_1,a_2,...,a_{n-1}$ aufgespannten Parallelotops  entspricht gerade der Norm $||\alpha||$ des äußeres Produkts.
\end{enumerate}

\begin{beispiel}
Die eben untersuchten Eigenschaften bringen uns dazu, im $\mathbb{R}^3$ das äußerde Produkt mit dem Kreuzprodukt zu identifizieren.
\begin{equation*}
\alpha_1\wedge\alpha_2\equiv\alpha_1\times\alpha_2 \ \ \ \ \ \text{für\ } n=3
\end{equation*}
\end{beispiel}

\begin{lemma}
Sind $b, a_1,a_2,...,a_{n-1}\in\mathbb{R}^n$, so ist
\begin{equation}
\scap{b}{a_1\wedge a_2\wedge...\wedge a_{n-1}}=\det (b | a_1|a_2|...|a_{n-1})
\end{equation}
wobei
\begin{equation*}
a_1\wedge a_2\wedge...\wedge a_{n-1} \ \bot a_i
\end{equation*}
und
\begin{equation*}
a_1\wedge a_2\wedge...\wedge a_{n-1} \ \ \ \ \ \left\{\begin{matrix}
=0 \ \ \text{falls\ }a_i\ \text{linear abhängig} \\
\ \ \ \ \ \neq 0 \ \ \text{falls\ }a_i\ \text{linear unabhängig}
\end{matrix}\right.
\end{equation*}
\end{lemma}

\begin{proof}
Wir können die Determinante in (29.4) nach der 1. Spalte $b$ entwickeln. Aus $b=a_i$ folgen die Bedingungen.
\end{proof}

\begin{beispiel}
Konstruieren wir ein Einheitsnormalenfeld mittels der Parametrisierung $\varphi:V\subset\mathbb{R}^{n-1}\rightarrow\mathbb{R}^n$ mit $V$ offen. 
$M=\{\varphi(v)\}$ sei die entsprechende Hyperfläche. Nach Satz 6 wissen wir, dass $\pdiff{}{x_j}\varphi(x)$ für alle $x$ und für alle $j=1,2,...,n-1$ im Tangentialraum $T_{\varphi(x)}M$ liegt. Wir erkennen außerdem, dass 
\begin{equation*}
N(x)\coloneqq\varphi_{x_1}(x)\wedge\varphi_{x_2}(x)\wedge ... \wedge\varphi_{x_{n-1}}
\end{equation*}
in $N_{\varphi(x)}M$ liegt und können damit
\begin{equation*}
\nu(x)\coloneqq\frac{N(x)}{||N(x)||}
\end{equation*}
als Einheitsnormalenfeld von $M$ wählen. Man beachte, dass $\varphi'(x)$ für alle $x$ regulär ist!
\end{beispiel}
Wir kommen zum Abschluss dieses Kapitels. Wir haben uns mit Mannigfaltigkeiten und ihren Eigenschaften beschäftigt. Als nächstes möchten wir die Integration auf ihnen untersuchen, beschränken uns dabei jedoch noch auf Kartengebiete.

\section{Integration auf Kartengebieten}
\setcounter{equation}{0}     %Quick fix for equation-counters not resetting with section
Wir stellen uns zunächst die interessante Frage, wie man den Oberflächeninhalt beziehungsweise das d-dimensionale Äquivalent dazu von einer Mannigfaltigkeit bestimmen kann. 
Die Idee wäre natürlich, wie etwa bei der Integration über $\mathbb{R}$, sie durch \emph{ebene} Mannigfaltigkeiten (etwa mit Dreiecken) stückweise zu approximieren.
\begin{equation*}
\textbf{Fläche}(M)=\sup \sum\limits_\Delta \textbf{Dreiecksflächen}
\end{equation*}
Wir stellen jedoch mit großem Entsetzen schnell fest, dass diese Methode nur für Kurven ($d=1$) funktioniert.\\
\linebreak
Schauen wir uns zum Beispiel eine Zylinderfläche $M\subset\mathbb{R}^3$ an. 
Lassen wir die Feinheit $\sigma$ beliebig klein werden - heute ist ja schließlich alles nano! - so wachsen die Dreiecksflächen immer weiter, 
bis die Fläche von $M$ über alle Grenzen hinaus wächst. Wir müssen uns also wohl sofort wieder von dieser Methode verabschieden.\\
\emph{Über dieses Dilemma nachlesen kann man übrigens in Hildebrand, Analysis 2 unter ''Schwartz'scher Stiefel''.}\\
\linebreak
Versuchen wir also etwas neues (für $d=2$) zu finden. Wir nehmen hierzu tangentionale Parallelogramme (äußere Approximation).
$\varphi'(x):\mathbb{R}^d\rightarrow\mathbb{R}^n$ ist linear. Die Methode gestaltet sich also zu
\begin{equation*}
\textbf{Fläche}(M)=\lim\limits_{\sigma\rightarrow 0}\left(\sum\textbf{Fläche\ }\varphi'(x_j)(Q)\right)
\end{equation*}
\begin{definition}
    Seien $a_1, \ldots, a_d \in \mathbb{R}^n (d \leq n)$ \\
    Dann heißt die Menge
    $P \left( a_1, \ldots, a_d \right) \coloneqq 
    \left\lbrace 
    \sum\limits_{j=1}^n t_j a_j | t_j \in [0,1], j = 1, \ldots, d    
    \right\rbrace$ \\
    das von $a_1, \ldots, a_d $ augespannte \textbf{Parallelotop} (auch d-Spat genannt).
\end{definition}

\textbf{Einschub:} Eine allgemeine Theorie für d-dimensionale Inhalte liefert das
Hausdorff-Maß $\mathcal{H}^d$, dieses ist jedoch sehr viel abstrakter und schwierig 
'auszurechnen'. Mithilfe von Mannigfaltgkeiten kommt man schneller zu Ergebnissen. \\
Weiterhin ist uns bereits das Maß über die delta-Funktion/Distribution bekannt, welches
zur Beschreibung von Punktmassen und -ladungen wichtig ist.

\begin{satz}
    \mbox{}
    Seien $a_1, \ldots, a_n \in \mathbb{R}^n$, \\
    das aufgespannte Volumen 
    $v(a_1, \ldots , a_n) \coloneqq \mathcal{L}^n (P (a_1, \ldots, a_n))$ \\
    so gilt:
    \begin{enumerate}
        \item[i)]
            $v(a_1, \ldots, \lambda a_k, \ldots, a_n) = 
            |\lambda| v(a_1, \ldots, a_n) \forall \lambda \in \mathbb{R}^n $
        \item[ii)]
            $v(a_1, \ldots, a_k + a_j, \ldots, a_n) = v(a_1, \ldots, a_n) $
            falls $k \neq j $ \\
            Dies ist bekannt als \textit{Prinzip des Cavaleri}, als Veranschaulichung kann
            ein Stapel Spielkarten dienen: Egal wie man eine Seitenfläche von einem Rechteck
            in ein Parallelogramm (oder umgekehrt) verschiebt, 
            das Volumen des Stapels bleibt gleich. \\
            \textbf{Skizze fehlt}
        \item[iii)]
            $v(a_1, \ldots, a_n) = 1 $ falls $a_1, \ldots, a_n $ ein Orthonormalsystem
            in $\mathbb{R}^n $ bilden. \\
            (Der Parallelotop ist dann der Einheitswürfel.)
        \item[iv)]
            $v(a_1, \ldots, a_n) = |\det A|$ für $A \coloneqq (a_1 | \ldots | a_n) $ \\
            Das heißt die Determinante der Matrix mit den Spaltenvektoren $a_1, \ldots, a_n$
            liefert das Volumen des aufgespannten Parallelotops. (Vgl. lin. Algbebra)
    \end{enumerate}
\end{satz}

    \textbf{Beachte:} Die Eigenschaften i) - iii) implizieren bereits iv), die Argumentation
    dazu verläuft wie zu den aus LAG bekannten Eigenschaften der Determinante, vgl.
    auch die axiomatische Definition der Determinante.

\begin{proof}
    \mbox{}
    \begin{enumerate}
        \item[a)] 
            Angenommen, $a_1, \ldots, a_n$ sind linear abhängig. \\
            Dann ist das aufgespannte Parallelotop 'flach', da es in mindestens 
            einer Dimension an Ausdehnung fehlt.
            $\Rightarrow v(a_1, \ldots, a_n) = 0 $ \\
            $\Rightarrow $ iv) ist korrekt 
            (da die Determinante einer singulären Matrix Null ist) \\
            $\Rightarrow $ i) und ii) sind korrekt
        \item[b)]
            Angenommen, $a_1, \ldots, a_n$ sind linear unabhängig. \\
            Sei $\lbrace e_1, \ldots, e_n \rbrace$ die Standard-Orthonormalbasis in
            $\mathbb{R}^n $, dafür gilt iii) nach der Definition des Lebesgue-Maß
            (Es ist ein Quader mit allen Seitenlängen gleich Eins). \\
            Weiter seien nun $U \coloneqq P(e_1, \ldots, e_n),
            V \coloneqq P(a_1, \ldots, a_n) $ \\
            $\Rightarrow A: \mathrm{int}\ U \rightarrow \mathrm{int}\ V $
            ist ein Diffeomorphismus (A ist regulär, ist damit differenzierbar und
            besitzt ein differenzierbares Inverses). \\
            Offenbar ist $A'(y) = A \forall y $ \\
            $\xRightarrow{\text{Trafosatz, Kap. 24}}
            \mathcal{L}^n (V) = \int\limits_V \mathrm{d}x \stackrel{y = Ax}{=}
            \int\limits_U |\det A| \mathrm{d}y = |\det A| \underbrace{\mathcal{L}^n (U)}_1
            = |\det A|
            \Rightarrow $ iv) $\Rightarrow $ i), ii), iii) folgen als Eigenschaften
            der Determinanten
    \end{enumerate}
\end{proof}

\textbf{Ziel:} Bestimmung des d-dimensionalen Inhalts 
$v_d (P(a_1, \ldots, a_d))$ in $\mathbb{R}^n $
\textbf{Idee:} Betrachte $P(a_1, \ldots, a_n)$ als Teilmenge eines d-dimensionalen
Vektorraums $X$ und nimm das d-dimensional Lebesgue-Maß in $X$.\\
Somit sollte $v_d: 
\underbrace{\mathbb{R}^n \times \ldots \times \mathbb{R}^n }_{\text{d-mal}}
\rightarrow \mathbb{R}_{\geq 0} $
folgende Eigenschaften haben:
\begin{enumerate}
    \item[(v1)]
        $v_d (a_1, \ldots, \lambda a_k, \ldots, a_d) = |\lambda| v_d(a_1, \ldots, a_d)
        \forall \lambda \in \mathbb{R} $
    \item[(v2)]
        $v_d (a_1, \ldots, a_k + a_j, \ldots, a_d) = v_d(a_1, \ldots, a_d) $
        falls $k \neq j$ (Prinzip des Cavaleri)
    \item[(v3)]
        $v_d(a_1, \ldots, a_d) = 1 $ falls $\lbrace a_1, \ldots, a_d \rbrace $
        orthonormal zueinander sind.
\end{enumerate}

\begin{satz}
    $v_d$ ist durch (v1), (v2), (v3) eindeutig bestimmt, und es gilt:
    \begin{equation}
        v_d(a_1, \ldots, a_d) = 
        \sqrt[]{\det \underbrace{A^T A}_{\in \mathbb{R}^{d \times d}}}
        \text{ mit } 
        A \coloneqq\underbrace{(a_1 | \ldots | a_d)}_{\in \mathbb{R}^{n \times d}}
    \end{equation}
\end{satz}

\textbf{Bemerkung:}
\mbox{}
\begin{enumerate}
    \item
        für $d=n $ liefert (30.1) Gleichung iv) in Satz 1
    \item
        $A^T A $ ist stets symmetrisch und positiv definit
        $\left( \left\langle x, A^T Ax \right\rangle =
        \langle Ax, Ax \rangle = \|Ax\|^2 \geq 0 \right)$
        und somit ist auch stets $\det A^T A \geq 0 $
    \item
        $v_d (a_1, \ldots, a_d) = 0 \Leftrightarrow a_1, \ldots, a_d$ linear abhängig
\end{enumerate}

\begin{proof}
    Selbststudium, verwende:\\
    $A^T A = 
    \begin{pmatrix}
        \alpha_{11} & \ldots & \alpha_{1n} \\
        \vdots      & \ddots & \vdots \\
        \alpha_{n1} & \ldots & \alpha_{nn}
    \end{pmatrix}
    $mit $\langle a_i, a_j \rangle $ \\
    und argumentiere wie bei den Eigenschaften der Determinante.
\end{proof}

\begin{beispiel}
    $d = n-1 $: Sei $a_1, \ldots, a_{n-a} \in \mathbb{R}^n, 
    a \coloneqq a_1 \wedge \ldots \wedge a_{n-1} $
    \begin{equation}
        \Rightarrow v_{n-1} (a_1, \ldots, a_{n-1}) = |a|_2
    \end{equation}
    Das heißt die euklidische Länge des äußeren Produkts liefert das Volumen.\\
    Denn: $
    \begin{pmatrix}
        a^T \\
        \rule[.5ex]{1em}{1pt}\\
        A^T
    \end{pmatrix}
    \bullet
    \begin{pmatrix}
        a & | & A 
    \end{pmatrix}
    =
    \begin{pmatrix}
        \langle a,a \rangle & 0 \\
        0 & A^T A
    \end{pmatrix}
    $ wegen $\langle a_i, a_j \rangle = 0 \forall j $, $A$ wie in (1)\\
    $\Rightarrow |a|_2^2 \det A^T A = (\det (a|A))^2 \stackrel{(29.4)}{=}
    |a|_2^4 \xRightarrow{\text{(1)}} $ (2)
\end{beispiel}

\textbf{Frage:} Existiert für eine Mannigfaltigkeit $M$ eine Transforamtion, so dass das
Volumen eines Quaders $Q \in \mathbb{R}^d $ auf das eines Parallelotops 
$P \subset T_uM \in \mathbb{R}^n $ abgebildet wird: \\
$v_d(\text{Quader }Q) \xrightarrow{\varphi'(x)} v_d(\text{Parallelotop } P) $ ?\\
\textbf{Skizzen fehlen}\\
Für einen Quader $Q = P(b_1, \ldots, b_d) \subset \mathbb{R}^d $ ist 
$P(a_1, \ldots, a_d) \subset T_uM \in \mathbb{R}^n $ das zugehörige Parallelotop, falls
$a_j = \varphi'(x) b_j $,für $ j=1, \ldots, d $

\begin{satz}
    Sei $M$ eine d-dimensionale Mannigfaltigkeit,\\
    $\varphi$ eine Parametrisierung um $\varphi(x) = u \in M $
    und sei \\
    $Q \coloneqq P(b_1, \ldots, b_d) \subset \mathbb{R}^d $ ein Quader 
    $(b_j \in \mathbb{R}^d), \\
    a_j \coloneqq \varphi'(x) b_j,\ j= 1, \ldots, d $ \\
    \begin{equation}
        \Longrightarrow 
        v_d(a_1, \ldots, a_d) = 
        \sqrt{\det \varphi'(x)^T \varphi'(x)}\ v_d(b_1, \ldots, b_d)
    \end{equation}
\end{satz}

\begin{definition}
    $\varphi'(x)^T \varphi'(x) \in \mathbb{R}^{d \times d} $
    heißt \textbf{Maßtensor}    von $\varphi$ in $x$ und\\
    $g^\varphi (x) \coloneqq \det \varphi'(x)^T \varphi'(x) $
    heißt \textbf{Gramsche Determinante} von $\varphi$ in $x$.
\end{definition}

\begin{proof}
    
\end{proof}
Nehmen wir uns die Funktion $f\equiv 1$ her, 
die offensichtlich integrierbar auf jedem Kartengebiet $U\subset M$ ist, 
dann ist
\begin{equation}
	v_d(U)\coloneqq \int_U 1 \mathrm{d}a 
	\ \ \left( =\int_V 1 \sqrt{g^\varphi(x)}\mathrm{d}x\right)
\end{equation}
der d-dimensionale Inhalt (Maß, Länge, Fläche, Volumen,...) von $U$ 
wobei $\sqrt{g^\varphi(x)}$ das Flächenelement von $U$ bezüglich $\varphi$ ist.\\
Mit dieser Definition entspricht $v_d(U)\equiv\mathcal{H}^d(U)$ direkt 
mit dem d-dimensionalen Hausdorffmaß überein (siehe Literatur).\\
Wir stellen außerdem nach (30.4) fest, dass $v_d(U)$ genau dann verschwindet, 
wenn $U$ ein Nullmenge ist:
\begin{equation*}
	v_d(U)=0 \ \Leftrightarrow \ \mathcal{L}^d(\varphi^{-1}(U))=0
\end{equation*}

\begin{beispiel}
Wir wollen $\int_Mf\mathrm{d}a$ auf einer Halbsphäre mit Radius $r$, 
gegeben durch
\begin{equation*}
	M\coloneqq\left\{u=(u_1,u_2,u_3)\in\mathbb{R}^3\middle| ||u||=r, u_1>0\right\}
\end{equation*}
berechnen und parametrisieren $M$ dafür in Kugelkoordinaten durch
\begin{equation*}
	\varphi(x_1,x_2)=r\begin{pmatrix}
	\cos x_1 \cos x_2 \\ \cos x_2 \sin x_1 \\ \sin x_2
	\end{pmatrix}
	\ \ \text{für\ } (x_1,x_2)\in V\coloneqq \left(-\frac{\pi}{2},\frac{\pi}{2}\right)^2
\end{equation*}
Offenbar ist $\varphi:V\rightarrow M$ stetig differenzierbar, regulär und homöomorph. 
Es handelt sich also tatsächlich um eine echte Parametrisierung von $M$. 
Das macht natürlich $M$ zu einer Mannigfaltigkeit und sogar zu einem Kartengebiet.
Um das Volumen zu berechnen benötigen wir zunächst
\begin{equation*}
	\varphi'(x)=r\begin{pmatrix}
	-\cos x_2 \sin x_1 & -\sin x_2 \cos x_1 \\
	 \cos x_2 \cos x_1 & -\sin x_2 \sin x_1 \\
	 0 & \cos x_2
	\end{pmatrix}
\end{equation*}
um über 
\begin{equation*}
	\varphi'(x)^T\varphi'(x)=r^2\begin{pmatrix}
	\cos^2 x_2 & 0 \\ 0 & 1 
	\end{pmatrix}
\end{equation*}
das Flächenelement
\begin{equation*}
	\sqrt{g^\varphi(x)}=r^2\cos x_2
\end{equation*}
zu berechnen.\\
Damit können wir das zu berechnende Integral folgendermaßen ausdrücken:
\begin{equation*}
	\int_Mf\mathrm{d}a = r^2\int_Vf(\varphi(x))\cos x_2\mathrm{d}x = 
	r^2\int\limits_{-\frac{\pi}{2}}^{\frac{\pi}{2}}\cos x_2\int\limits_{-\frac{\pi}{2}}^{\frac{\pi}{2}}f(\varphi(x)\mathrm{d}x_1\mathrm{d}x_2 
\end{equation*}
Wählen wir nun zum Beispiel $f(u)=u_1^2+u_2^2$, dann ist $f(\varphi(x)=r^2\cos^2 x_2$ und das Integral wird zu
\begin{equation*}
	\int_M(u_1^2+u_2^2)\mathrm{d}a=r^4\int\limits_{-\frac{\pi}{2}}^{\frac{\pi}{2}}\cos x_2 \int\limits_{-\frac{\pi}{2}}^{\frac{\pi}{2}}\mathrm{d}x_1\mathrm{d}x_2 =
\end{equation*}
\begin{equation*}
	= \pi r^4 \int\limits_{-\frac{\pi}{2}}^{\frac{\pi}{2}} \cos^3 x_2\mathrm{d}x_2 = \pi r^4 \left[\sin x_2-\frac{1}{3}\sin^3 x_2\right]_{-\frac{\pi}{2}}^{\frac{\pi}{2}} = 
	2\pi r^4\left(1-\frac{1}{3}\right)=\frac{4}{3}\pi r^4
\end{equation*}
Ist nun $f(u)=1$, dann ist
\begin{equation*}
	v_d(M)=\pi r^2 \int_M\mathrm{d}a=\pi r^2 \int\limits_{-\frac{\pi}{2}}^{\frac{\pi}{2}} \cos x_2 \mathrm{d}x_2 = 
	\pi r^2 \left[\sin x_2\right]_{-\frac{\pi}{2}}^{\frac{\pi}{2}} = 2 \pi r^2
\end{equation*}
was genau genau der halben Sphärenfläche entspricht. 
Es ist zu bemerken, dass wir in unserer Rechnung den Rand von $M$ komplett vernachlässigt haben. 
Wir werden jedoch später zeigen, dass derartige Nullmengen keinen Beitrag leisten. 
\end{beispiel}

\begin{satz}[Integral über $(n-1)$-dimensionale Graphen]
\ \\ Sei $g:V\subset\mathbb{R}^{n-1}\rightarrow\mathbb{R}$ mit $V$ offen eine stetig differenzierbare Funktion und 
\begin{equation*}
	\Gamma\coloneqq\left\{\left(x,g(x)\right)\in\mathbb{R}^n \middle| x\in V \right\}
\end{equation*}
der Graph von $g$. Dann gilt für $f:\Gamma\rightarrow\mathrm{R}$
\begin{equation}
\int_\Gamma f\mathrm{d}a=\int_Vf\left(x,g(x)\right)\sqrt{1+|g(x)|^2}\mathrm{d}x
\end{equation}
falls die rechte Seite existiert.
\end{satz}

\begin{proof}
$\Gamma$ ist eine $(n-1)$-Mannigfaltigkeit (vgl. Bsp. 29.2) 
und auch Kartengebiet bezüglich der Parametrisierung $\varphi:V\rightarrow\Gamma$ 
mit $\varphi(x)=\left(x,g(x)\right)$. Wir setzen 
\begin{equation*}
	\gamma\coloneqq\sqrt{\det \varphi'(x)^T\varphi'(x)}
\end{equation*}
und sehen mit (30.1), dass
\begin{equation*}
	\gamma=v_{n-1}\left(\varphi_{x_1}(x)\middle|...\middle|\varphi_{x_{n-1}}(x)\right)
\end{equation*}
und dann wiederum mit (30.2)
\begin{equation*}
	\gamma=||\varphi_{x_1}(x)\wedge ... \wedge \varphi_{x_{n-1}}(x)||
\end{equation*}
Da aber auch
\begin{equation*}
	\varphi_{x_1}(x)\wedge ... \wedge \varphi_{x_{n-1}}(x)=(-1)^n\begin{pmatrix}
	g'(x) \\ -1
	\end{pmatrix}\in\mathbb{R}^n
\end{equation*}
gilt, können wir $\gamma$ auch als
\begin{equation*}
	\gamma=\sqrt{1+||g'(x)||^2}
\end{equation*}
schreiben. Damit erhalten wir
\begin{equation*}
	\int_\Gamma f\mathrm{d}a=\int f(\varphi(x))\sqrt{1+||g'(x)||^2}\mathrm{d}x
\end{equation*}
falls die rechte Seite existiert. Damit gilt für den Inhalt von $\Gamma$ (falls er existiert):
\begin{equation}
	v_{n-1}(\Gamma)=\int_V\sqrt{1+||g'(x)||^2}\mathrm{d}x
\end{equation}
\end{proof}

\begin{beispiel}
Betrachten wir die Halbsphäre
\begin{equation*}
	S_+^{n-1}\coloneqq\left\{ x\in\mathbb{R}^n \middle| |x|=1,\ x_n>0 \right\}
\end{equation*}
die offenbar für alle $x\in B_1(x)\subset\mathbb{R}^{n-1}$ der Graph von $g(x)=\sqrt{1-|x|^2}$ ist. 
Mit (30.8) sehen wir sofort
\begin{equation*}
	v_{n-1}\left(S_+^{n-1}\right)=\int\limits_{B_1(0)\subset\mathbb{R}^{n-1}}\sqrt{1+\frac{|x|^2}{1-|x|^2}}\mathrm{d}x = \int\limits_{B_1(0)}\frac{1}{\sqrt{1+|x|^2}}\mathrm{d}x
\end{equation*}
Wir können an dieser Stelle ohne Beweis annehmen, dass $f$ rotationssymetrisch auf $B_1(0)\subset\mathbb{R}^{n-1}$ ist (d.h. $f(x)=f(|x|)\ $). Dann verwenden wir (aus Königsberger Analysis 2, Kap 8.2)
\begin{equation}
	\int\limits_{B_r(0)}f(x)\mathrm{d}x = 
	n\kappa_n \int\limits_0^r\tilde{f}(\rho)\rho^{n-1}\mathrm{d}\rho
	\ \ \ \ \ \text{für\ }B_r(0)\subset\mathbb{R}^n \ 	\text{und\ } \kappa_n\coloneqq\mathcal{L}^n \left(B_r(0)\right)
\end{equation}
und münzen dies auf $(n-1)$ um:
\begin{equation*}
	v_{n-1}\left(S_+^{n-1}\right) =
	(n-1)\kappa_{n-1}\int\limits_0^1\frac{r^{n-2}}{\sqrt{1-r^2}}\mathrm{d}r = 
	(n-1)\kappa_{n-1}\int\limits_0^1 r^n\frac{1}{r^2\sqrt{1-r^2}}\mathrm{d}r = 
\end{equation*}	
Dies integrieren wir partiell und erhalten
\begin{equation*}
	= n(n-1)\kappa_{n-1}\int\limits_0^1 r^{n-1}\frac{\sqrt{1-r^2}}{r}\mathrm{d}r=n\int\limits_{B_1(0)}\sqrt{1-|x|^2}\mathrm{d}r=\frac{n}{2}\kappa_n
\end{equation*}	
Was haben wir damit herausbekommen? Wir setzen
\begin{equation*}
	\omega\coloneqq\left(S^{n-1}\right)=2v_{n-1}\left(S_+^{n-1}\right)
\end{equation*}
als die Oberfläche der Sphäre $S^{n-1}\subset\mathbb{R}^n$ und sehen, dass für alle $n\in\mathbb{N}_{\geq 2}$
\begin{equation}
	\omega_n=n\kappa_n
\end{equation}
Das ist ein erstaunliches Resultat, welches wir uns an zwei Beispielen verdeutlichen wollen.\\
\begin{center}
$\begin{matrix}
n=2: & \ \ \ \ \ \  & v_{n-1} & = & 2\pi & = & 2\cdot v_n & = & 2 \cdot \pi \\
n=3: & \ \ \ \ \ \  & v_{n-1} & = & 4\pi & = & 3\cdot v_n & = & 3 \cdot \frac{4}{3}\pi
\end{matrix}$
\end{center}
Wir können dieses Resultat sogar auf beliebige Kugeln skalieren:
\begin{equation*}
	v_n\left(B_r(0)\right)=\mathcal{L}^n\left(B_r(0)\right)=r^n\kappa_n
\end{equation*}
Mithilfe des Transformationssatzes können wir das ganze sogar noch umschreiben 
und erhalten ein Beziehung die sich später in der Differentialgleichungstheorie 
als maßgeblich herausstellen wird:
\begin{equation*}
	v_{n-1}\left(\partial B_r(0)\right)=r^{n-1}\omega_n=r^{n-1}n\kappa_n
\end{equation*}
\end{beispiel}

\begin{beispiel}[Kurvenintegral]
Wir betrachten die Kurve $\varphi:I\subset\mathbb{R}\rightarrow\mathbb{R}^n$, 
wobei $I$ ein offenes Intervall ist, so dass
\begin{equation*}
	C\coloneqq\varphi(I)
\end{equation*}
eine 1-dimensionale Mannigfaltigkeit ist. Wir erinnern uns, dass $\varphi$ 
genau dann regulär ist, wenn $\varphi'(t)\neq 0$ ist. Offenbar ist 
\begin{equation*}
	\det \varphi'(t)^T\varphi'(t)=|\varphi'(t)|^2
\end{equation*}
und so können wir, falls es existiert, für ein $f:C\rightarrow\mathbb{R}$ 
mit $I=(a,b)$ formulieren
\begin{equation}
 \int_Cf\mathrm{d}a=\int\limits_a^bf(\varphi(t))|\varphi'(t)|\mathrm{d}t
\end{equation}
Dieses Integral nennen wir das Kurvenintegral von $f$ entlang des Weges $C$. 
Auf diesem Wege können wir natürlich auch den 1-dimensionalen Inhalt, $C$ 
, den wir \textbf{Bogenlänge} nennen, bestimmen, in dem wir einfach $f\equiv 1$ setzen.
\begin{equation}
	v_1(C)=\int\limits_a^b|\varphi'(t)|\mathrm{d}t
\end{equation}
Falls wir nun noch ein $\varphi$ finden, sodass $|\varphi'(t)|=1$ ist, 
dann nennen wir dieses $\varphi$ \textbf{Bogenlängenparametrisierung} 
von $C$, da uns in diesem Fall $t$ direkt die Bogenlänge liefet:
\begin{equation*}
	t_2-t_1=v_1(\varphi(t_2-t_1))
\end{equation*}
Wir können natürlich auch durch einen Kartenwechsel umparametrisieren 
und erhalten so für 
\begin{equation*}
	\sigma(s)=\int\limits_a^s|\varphi'(t)|\mathrm{d}t 
	\tag{$\ast$}
\end{equation*}
immer ein $\psi:(0,v_s(C))\rightarrow\mathbb{R}^n$ finden, 
das mit $\psi(I)=\varphi(\sigma^{-1}(I))$ stets eine Bogenlängenparametrisierung 
ist. Das können wir ganz leicht zeigen:\\
Offenbar ist $\sigma$ stetig differenzierbar und sogar monoton wachsend, 
weil $\varphi'$ regulär ist. Das bedeutet, dass ein $\sigma^{-1}$ 
existiert, das ebenfalls stetig differenzierbar ist. 
So können wir folgern
\begin{equation*}
	|\psi'(\tau)| = 
	\left| \varphi' ( \sigma^{-1} (\tau)\cdot\sigma^{-1'} (\tau)\right|= 
	\left| \varphi' ( \sigma^{-1}\right|\frac{1}{|\sigma'(\sigma^{-1}(\tau))|} 
	\overset{(\ast)} = 1
\end{equation*}
Für jede Kurve existiert also \emph{genau eine} ausgezeichnete Parametrisierung, 
nämlich die Bogenlängenparametrisierung.
\end{beispiel}
Wir wollen uns dem Thema der Kurvenlänge nun auf eine etwas andere Weise annähern.
\begin{definition}[Rektifizierbarkeit]
Für eine beliebige stetige Funktion $\varphi:[a,b]\rightarrow\mathbb{R}^n$ 
heißt die zugehörige Kurve $C=\varphi([a,b])$ \textbf{rektifizierbar}, 
falls 
\begin{equation*}
	l(C)\coloneqq\sup_Z\left\{\sum\limits_{j=1}^k |\varphi(t_j)-\varphi(t_{j-1})| \middle| \{t_0,...,t_k\}\in Z\right\} < \infty
\end{equation*}
wobei $Z$ die Menge alle geordneten Zerlegungen von $C$ ist. 
Man kann $C$ also eine Länge zuordnen.
\end{definition}

\begin{satz}[Rektifizierbare Kurven]
Sei $\varphi:[a,b]\rightarrow\mathbb{R}^n$ eine 
stetig differenzierbare Funktion, dann ist 
\begin{enumerate}
	\item $\varphi$ rektifizierbar und
	\item $C\coloneqq\varphi((a,b))$ eine 1-dimensionale Mannigfaltigkeit 
	mit der zugehörigen Parametrisierung $\varphi$.
\end{enumerate}
Insbesondere gilt dann:
\begin{equation*}
	l(C)=v_1(C)=\mathcal{H}^1(C)
\end{equation*}
\end{satz}

Das ist beruhigend. Wir möchten uns den gesamten Beweis ersparen, dann 
er mit sehr viel Schreibarbeit verbunden ist. Deshalb hier nur eine 
\emph{Beweisskizze}:\\
$\varphi$ ist Lipschitz-stetig auf $[a,b]$, also 
\begin{equation*}
	l(\varphi[a,b])<L[a,b]
\end{equation*}
Damit ist $\varphi$ rektifizierbar und 1. gezeigt. Für 2. zeigen wir, dass
\begin{equation*}
	l(t)\coloneqq l(\varphi[a,t])<L[a,b] 
\end{equation*}
stetig differenzierbar auf $(a,b)$ mit $l'(t)=|\varphi'(t)|$ ist. 
Daraus folgt dann 
\begin{equation*}
	l(b)=\int\limits_a^bl'(t)\mathrm{d}t = 
	\int\limits_a^b|\varphi'(t)|\mathrm{d}t = 
	v_1(C)
\end{equation*}

\begin{beispiel}[Umfang des Einheitskreises]
\begin{equation*}
	\varphi:\left(-\pi,\pi\right)\rightarrow\mathbb{R}^n 
	\ \ \ \ \ \text{mit\ } \varphi(t)=\begin{pmatrix}
	\cos t \\ \sin t
	\end{pmatrix}
\end{equation*}
$C\coloneqq \varphi([-\pi,\pi])$ ist der Einheitskreis ohne den Punkt $(-1,0)^T$. 
Dann ist 
\begin{equation*}
	v_1(C)=\int\limits_{-\pi}^{\pi}|\varphi'(t)|\mathrm{d}t=\int\limits_{-\pi}^{\pi}\left\|\begin{pmatrix}
	-\sin x \\ \cos x
	\end{pmatrix}\right\|\mathrm{d}t = 
	\int\limits_{-\pi}^{\pi}1\mathrm{d}t=2\pi
\end{equation*}
Wichtig hierbei ist, dass $\pi$ die Bogenparametrisierung ist.
\end{beispiel}
Nachdem wir nun auf Kartengebieten integrieren können, 
lassen sich natürlich eine Vielzahl von Eigenschaften untersuchen. 

\begin{satz}
	Seien $f,g,f_k:U\rightarrow\mathbb{R}$ Funktionen, 
	die vom Kartengebiet $U$ der Mannigfaltigkeit $M\subset\mathbb{R}^n$ 
	abbilden, dann gelten die folgenden Beziehungen:
	\begin{enumerate}
		\item 	$f$ ist einerseits genau dann auf $U$ integrierbar, 
				wenn $|f|$ integrierbar auf $U$ ist und auch genau dann, 
				wenn es $f^+$ und $f^-$ sind.
		\item 	Sind $f$ und $g$ integrierbar auf $U$ und $c\in\mathbb{R}$, 
				so ist
				\begin{equation*}
					\int_Ucf\pm g\mathrm{d}a = 
					c\int_Uf\mathrm{d}a\pm\int_Ug\mathrm{d}a
				\end{equation*}
		\item 	Sind $f$ und $g$ integrierbar und $g$ beschränkt auf $U$, 
				dann ist auch $f\circ g$ integrierbar auf $U$.
		\item 	Sind $f$ und $g$ integrierbar und $f\leq g$ auf $U$, 
				dann dies auch für die Integrale:
				\begin{equation*}
					\int_Uf\mathrm{d}a\leq\int_Ug\mathrm{d}a
				\end{equation*}
		\item	\textbf{Monotone Konvergenz:} Seien alle $f_k$ 
				integrierbar und $f_1\leq f_2\leq ...$ auf $U$, 
				außerdem sei die Folge der Integrale $\left(\int_Uf_k\mathrm{d}a\right)$ 
				beschränkt sowie $f(u)=\lim\limits_{k\rightarrow\infty}f_k(u) \ \ \forall u\in U$, 
				dann ist $f$ integrierbar auf $U$ mit
				\begin{equation*}
					\int_Uf\mathrm{d}a = 
					\lim\limits_{k\rightarrow\infty}\int_Uf_k\mathrm{d}a
				\end{equation*}
		\item 	\textbf{Majorisierte Konvergenz:} Seien $g$ und alle $f_k$ 
				integrierbar und weiterhin $f_k\leq g \ \ \forall k$ auf $U$. 
				Außerdem möge $f(u) = \lim\limits_{k\rightarrow\infty}f_k(u) \ \ \forall u\in U$ 
				sein. Dann ist $f$ integrierbar auf $U$ mit
				\begin{equation*}
					\int_Uf\mathrm{d}a = 
					\lim\limits_{k\rightarrow\infty}\int_Uf_k\mathrm{d}a
				\end{equation*}
	\end{enumerate}
\end{satz}
\newpage
\begin{proof}
	Wir werden diesen Beweis sehr kurz halten, indem wir einige 
	Äquivalenzen finden, aus denen unter Verwendung der Eigenschaften 
	des Lebesque-Maßes die Behauptung folgen wird.\\
	Wir setzen zunächst $\varphi:V\rightarrow U$ als die Parametrisierung 
	des	Kartengebietes $U$. 
	\begin{enumerate}
		\item 	$f$ ist genau dann integrierbar, wenn es 
				$f\left(\varphi(.)\right)\sqrt{g^\varphi(.)}$ ist. 
				Dies folgt aus 	der Definition der Gramschen Determinante. 
		\item 	$f\leq g$ ist äquivalent dazu, dass 
				$f\left(\varphi(.)\right)\sqrt{g^\varphi(.)} \leq 
				g\left(\varphi(.)\right)\sqrt{g^\varphi(.)}$ ist.
	\end{enumerate}
	Daraus folgen unter Verwendung der Eigenschaften des Lebesque-Maßes 
	schließlich die Behauptungen. 
\end{proof}

\section{Integrale auf Mannigfaltigkeiten}

Im letzten Kapitel haben wir uns dem Integrieren auf Kartengebieten 
gewidmet. Das möchten wir natürlich nun auch auf ganzen Mannigfaltigkeiten 
tun können. Da stellt sich jedoch die Frage, wie dieses Integral dann 
genau aussehen soll. \\
Wir könnten uns natürlich überlegen, die Mannigfaltigkeit einfach 
mit Kartengebieten $U_\beta$, wobei $\beta$ hier eine vollkommen 
beliebige Indexmenge sein kann, zu überdecken und die Integration 
dann auf jedem Gebiet einzeln durchzuführen. An dieser Stelle stoßen 
wir jedoch auf ein Problem. Kartengebiete sind immer offen! Sie 
werden sich also, um die Mannigfaltigkeit Überdecken zu können, 
überlappen müssen. Damit wird das berechnete Integral immer zu 
groß sein. Deshalb müssen wir uns etwas anderes ausdenken.\\
\linebreak
Den Ausweg liefert eine sehr hilfreiche Methoden; die sogenannte 
\emph{Zerlegung der Eins}, für die wir $\alpha\equiv 1$ auf der 
Mannigfaltigkeit so setzen, dass $\sum\limits_{j=1}^\infty \alpha_j(u)=1$ ist.

\begin{definition}[Zerlegung der Eins]
	Eine Menge stetiger Funktionen $\alpha_j:M\rightarrow [0,1]$ 
	mit $j\in\mathbb{N}$ heißt \textbf{Zerlegung der Eins} auf 
	einer Menge $M\subset\mathbb{R}^n$, falls folgende Bedingungen 
	erfüllt sind:
	\begin{enumerate}
		\item 	$\sum\limits_{j=1}^\infty a_j(u) = 1   \ \ \ \forall u\in M$
		\item 	Die Zerlegung ist lokal endlich, das heißt für alle 
				$u\in M$ existiert eine Umgebung $U\subset M$ bezüglich 
				$M$, so dass auf dieser Umgebung \emph{fast überall} 
				$\alpha_j=0$ ist.
	\end{enumerate}
\end{definition}

\begin{definition}[Unterordnung]
	Sei $\mathcal{U}$ eine bezüglich $M$ offene Überdeckung von 
	$M\subset\mathbb{R}^n$. Dann heißt die Zerlegung der Eins $\{\alpha_j\}$ 
	der Überdeckung $\mathcal{U}$ \textbf{untergeordnet}, falls es 
	für alle $j\in\mathbb{N}$ ein $U_j\in\mathcal{U}$ gibt, so dass
	\begin{equation*}
		\mathrm{supp\ }\alpha_j\subset U_j
	\end{equation*} 
	wobei mit ''$\mathrm{supp}$'' hier der Träger von $\alpha_j$, 
	gegeben durch 
	\begin{equation*}
		\mathrm{supp\ }\alpha_j=\overline{\left\{u\in M \middle| \alpha_j(u)\neq 0 \right\} }
	\end{equation*}
	gemeint ist.
\end{definition}

\begin{satz}[Existenz der Zerlegung der Eins]
	Sei $M\subset\mathbb{R}^n$ und $\mathcal{U}$ eine bezüglich $M$ 
	offene Überdeckung von $M$, dann existiert immer eine Zerlegung 
	der Eins $\{\alpha_j\}$ von $M$, die $\mathcal{U}$ untergeordnet 
	ist.
\end{satz}

Hierzu muss man anmerken, dass wir $\alpha_j$ immer so konstruieren 
können, das es unendlich oft differenzierbar ist. Später betrachten 
wir dann die die Überdeckung einer Mannigfaltigkeit mit Kartengebieten. 

\begin{proof}
	Dieser Beweis ist sehr technisch und langweilig. Deshalb wird hier 
	auf die Literatur (\emph{Königsberger}) verwiesen.
\end{proof}

\begin{definition}[Integrierbarkeit auf Mannigfaltigkeiten]
	Sei $M\subset\mathbb{R}^n$ eine Mannigfaltigkeit, $f:M\rightarrow\mathbb{R}$, 
	$\mathrm{supp\ }\subset U \subset M$, wobei $U$ ein Kartengebiet 
	von $M$ ist. Dann heißt $f$ \textbf{integrierbar auf M} falls 
	$f_{|U}$ (die Funktion eingeschränkt auf U) auf dem Kartengebiet 
	integrierbar ist.
	\begin{equation}
		\int_Mf\mathrm{d}a=\int_Mf_{|U}\mathrm{d}a
	\end{equation}
	heißt dann das \textbf{Integral} von $f$ auf $M$.
\end{definition}
\newpage

\begin{lemma}[Kriterium für Integration auf Kartengebieten]
Sei $M\subset\mathbb{R}^n$ eine Mannigfaltigkeit, $f:M\rightarrow\mathbb{R}$, 
	$\mathrm{supp\ }\subset U \subset M$, wobei $U$ ein Kartengebiet 
	von $M$ ist, und $\{\alpha_j\}$ eine Zerlegung der Eins auf $M$. 
	Dann ist die Integrierbarkeit von $f$ auf $M$ zu folgenden 
	Aussagen äquivalent:
	\begin{enumerate}
		\item 	$f\alpha_j$ ist für alle $j\in M$ integrierbar auf $M$.
		\item	$\sum\limits_{j=1}^\infty\int_M|f|\alpha_j\mathrm{d}a<\infty$
	\end{enumerate}
	Daraus folgt dann auch
	\begin{equation}
		\int_Mf\mathrm{d}a=\sum\limits_{j=1}^\infty\int_Mf\alpha_j\mathrm{d}a
	\end{equation}
\end{lemma}

\begin{proof}
	Ist $f$ integrierbar auf $M$, so folgt 1. unmittelbar aus Satz 
	30.7. Außerdem ist
	\begin{equation*}
		\sum\limits_{j=1}^\infty\int_M|f|\alpha_j\mathrm{d}a = 
		\lim\limits_{ k \rightarrow \infty }\sum\limits_{j=1}^k\int_M|f|\alpha_j\mathrm{d}a \leq 
		\int|f|\mathrm{d}a < \infty
	\end{equation*}
	Durch die majorisierte Konvergenz folgt (31.2).\\
	Gelten 1. und 2., so folgt durch die Monotone Konvergenz, dass 
	$|f|$ integrierbar und somit (nach Satz 30.7) auch $f$ integrierbar 
	ist.
	
\end{proof}

\begin{definition}[Integral auf Mannigfaltigkeiten]
	Sei $M\subset\mathbb{R}^n$ eine Mannigfaltigkeit und $\mathcal{U}$ 
	eine offene Überdeckung von $M$ mit Kartengebieten. Dann heißt 
	$f:M\rightarrow\mathbb{R}$ \textbf{integrierbar auf M}, falls eine
	Zerlegung der Eins existiert, die $\mathcal{U}$ untergeordnet ist, 
	so dass
	\begin{enumerate}
		\item 	$f\alpha_j$ für alle $j\in\mathbb{N}$ integrierbar
				 auf $M$ ist.
		\item 	$\sum\limits_{j=1}^\infty\int_M|f|\alpha_j\mathrm{d}a<\infty$
	\end{enumerate}	 
	Wir nennen dann 
	\begin{equation}
		\int_Mf\mathrm{d}a\coloneqq
		\sum\limits_{j=1}^\infty\int_Mf\alpha_j\mathrm{d}a
	\end{equation}
	auch \textbf{Integral auf M}.
\end{definition}

\begin{satz}[Rechtfertigung der Integralbegriffe]
	Die Definitionen der Integrierbarkeit von $f$ auf $M$ und des 
	Integrals $\int_Mf\mathrm{d}a$ sind unabhängig von der 
	konkreten Überdeckung $\mathcal{U}$ und der Zerlegung der Eins $\{\alpha_j\}$.
\end{satz}

\emph{Beweisidee:} Sei $f:M\rightarrow\mathbb{R}$ integrierbar auf 
$M$ und $\mathcal{U}$ und $\{\alpha_j\}$ wie in der Definition. Dann 
zeigen für weitere Überdeckungen $\tilde{\mathcal{U}}$ und ihre 
untergeordneten Zerlegungen der Eins $\{\alpha_j\}$, dass 
\begin{equation*}
	\sum\limits_{j=1}^\infty\int_M\alpha_j\mathrm{d}a = 
	\sum\limits_{j=1}^\infty\int_M\tilde{\alpha}_j\mathrm{d}a
\end{equation*}
Dazu verwenden wir Lemma 31.2.\\
\linebreak

\begin{definition}[Integral auf Teilmengen von Mannigfaltigkeiten]
	Ist $M\subset\mathbb{R}^n$ eine Mannigfaltigkeit und $A$ eine 
	Teilmenge derselben, so heißt die Funktion $f:A\rightarrow\mathbb{R}$ 
	integrierbar auf $A$, falls
	\begin{equation*}
		f_A\coloneqq\left\{\begin{matrix}
		f \ \ \mathrm{auf\ }A \\ 0 \ \ \mathrm{sont}
		\end{matrix}\right.
	\end{equation*}
	auf $M$ integrierbar ist.
	Wir setzen dann
	\begin{equation*}
		\int_Af\mathrm{d}a\coloneqq\int_Mf_A\mathrm{d}a
	\end{equation*}
	als das Integral von $f$ auf $A$. Dasselbe $A$ heißt dann auch
	(endlich) \textbf{messbar in M}, falls $f\equiv 1$ integrierbar 
	auf $A$ ist. Natürlich nennen wir dann
	\begin{equation*}
		v_d(A)\coloneqq\int_A\mathrm{d}a
	\end{equation*}
	den d-dimensionalen Inhalt oder das d-dimensionale Maß von A.
\end{definition}
Der feine Zusatz ''endlich'' ist zu beachten, denn unter dem normalen 
Lebesque-Maß sind wir auch $\mathcal{L}^n(A)=\infty$ gewohnt. Hier 
ist $v_d(A)$ jedoch strikt endlich.

\begin{definition}[d-Nullmengen]
$A\subset M$ heißt \textbf{d-Nullmenge}, falls $v_d(A)=0$ ist. 
d-Nullmengen auf $M$ entsprechen natürlich $\mathcal{L}$-Nullmengen 
im Parameterbereich.
\end{definition}

\begin{satz}
Ist $M\subset\mathbb{R}^n$ eine Mannigfaltigkeit und $A\subset M$ 
kompakt bezüglich $M$, sowie $f:A\rightarrow\mathbb{R}$ eine stetige 
Abbildung. Dann ist $f$ auf $A$ integrierbar.
\end{satz}
Wir wissen, dass $A$ kompakt ist, wenn das Urbild $\varphi^-1(A)$ kompakt 
ist, da $\varphi$ ja homöomorph ist.\\
\emph{Beweisidee:} Da $A$ kompakt ist, gibt es eine endliche 
Überdeckung mit Kartengebieten. Da $f(\varphi(.))\sqrt{g^\varphi(.)}$ 
stetig ist, ist es integrierbar auf kompakten Mengen.\\
\linebreak
Wir können nun, da wir statt nur auf Kartengebieten, auf ganzen 
Mannigfaltigkeiten integrieren können, die Integraleigenschaften 
aus Satz 30.7 neu formulieren:
\begin{satz}[Eigenschaften des Integrals über Mannigfaltigkeiten]
	Sei $M\subset\mathbb{R}^n$ eine Mannigfaltigkeit, und 
	$f,g,f_k:M\rightarrow\mathbb{R}$ Funktionen, die von der 
	Mannigfaltigkeit nach $\mathbb{R}$ abbilden, dann gelten die 
	folgenden Beziehungen:
	\begin{enumerate}
		\item 	$f$ ist einerseits genau dann auf $M$ integrierbar, 
				wenn $|f|$ integrierbar auf $M$ ist und auch genau dann, 
				wenn es $f^+$ und $f^-$ sind.
		\item 	Sind $f$ und $g$ integrierbar auf $M$ und $c\in\mathbb{R}$, 
				so ist
				\begin{equation*}
					\int_Mcf\pm g\mathrm{d}a = 
					c\int_Mf\mathrm{d}a\pm\int_Mg\mathrm{d}a
				\end{equation*}
		\item 	Sind $f$ und $g$ integrierbar und $g$ beschränkt auf $M$, 
				dann ist auch $f\circ g$ integrierbar auf $M$.
		\item 	Sind $f$ und $g$ integrierbar und $f\leq g$ auf $M$, 
				dann dies auch für die Integrale:
				\begin{equation*}
					\int_Mf\mathrm{d}a\leq\int_Mg\mathrm{d}a
				\end{equation*}
		\item	\textbf{Monotone Konvergenz:} Seien alle $f_k$ 
				integrierbar und $f_1\leq f_2\leq ...$ auf $M$, 
				außerdem sei die Folge der Integrale $\left(\int_Mf_k\mathrm{d}a\right)$ 
				beschränkt sowie $f(u)=\lim\limits_{k\rightarrow\infty}f_k(u) \ \ \forall u\in M$, 
				dann ist $f$ integrierbar auf $M$ mit
				\begin{equation*}
					\int_Mf\mathrm{d}a = 
					\lim\limits_{k\rightarrow\infty}\int_Mf_k\mathrm{d}a
				\end{equation*}
		\item 	\textbf{Majorisierte Konvergenz:} Seien $g$ und alle $f_k$ 
				integrierbar und weiterhin $f_k\leq g \ \ \forall k$ auf $M$. 
				Außerdem möge $f(u) = \lim\limits_{k\rightarrow\infty}f_k(u) \ \ \forall u\in M$ 
				sein. Dann ist $f$ integrierbar auf $M$ mit
				\begin{equation*}
					\int_Mf\mathrm{d}a = 
					\lim\limits_{k\rightarrow\infty}\int_Mf_k\mathrm{d}a
				\end{equation*}
	\end{enumerate}
\end{satz}
Für eine explizite Berechnung von $\int\limits_M f \mathrm{d}a $
benutzt man idR keine ZdE, sondern zerlegt 
$M = \bigcup\limits_j M_j $ mit $M_j $ paarweise disjunkt und berechnet alle
$\int\limits_{M_j} f \mathrm{d}a $ \\
$\longrightarrow \int\limits_M f \mathrm{d}a 
= \sum\limits_{j=1}^k \int\limits_{M_j} f \mathrm{d}a $

\section{Integralsätze von Gauß und Stokes}

\begin{definition}[Regulärer Randpunkt]
    \mbox{} \\
    Für $\Omega \subset \mathbb{R}^n $ heißt $x \in \partial \Omega $
    \textbf{regulärer Randpunkt} von $\Omega $, falls er eine offene Zylinderumgebung
    $Q \subset \mathbb{R}^n $ besitzt, so dass 
    \emph{nach einer eventuell notwendigen Drehung des Koordinatensystems} gilt:\\
    $Q = Q' \times I $ für $Q' \subset \mathbb{R}^{n-1} $ offen, beschränkt, 
    $I \subset \mathbb{R} $ offenes Intervall und es existiert eine $C^1 $-Funktion
    $h: \tilde{Q}' \rightarrow I $ mit $\tilde{Q}' $ Umgebung von $Q' $
    in $\mathbb{R}^{n-1} $ und \\
    \begin{equation*}
        \Omega \cap Q = 
        \left\lbrace \left(x', x^n \right) \in Q' \times I \ |\ x_n \geq h(x') 
        \right\rbrace
    \end{equation*}        
     
    \begin{equation}
        \partial\Omega \cap Q = 
        \left\lbrace \left(x', x^n \right) \in Q' \times I \ |\ x_n = h(x') \right\rbrace
    \end{equation}
    $x'$ bezeichnet nach Konvention einen (n-1)-dimensionalen Vektor, $x_n$ stellt die
    'fehlende' n-te Komponente dar, $(x', x_n) $ ist also ein n-dimensionaler Vektor.
    Analog wird $Q'$ durch das Intervall $I$ zur Teilmenge vom $\mathbb{R}^n $ ergänzt.
    
    \includegraphics[scale=0.4]{pictures/007-01}
       
    Das heißt in einer Umgebung von $x$ stimmt $\partial \Omega $ mit dem Graphen von $h$
    überein und $\Omega$ liegt über dem Graphen.
    
    \includegraphics[scale=0.5]{pictures/007-02}
    
    $\partial_r \Omega $ bezeichnet die \textbf{Menge aller regulären Randpunkte}. \\
    $\partial_s \Omega \coloneqq \partial \Omega / \partial_r \Omega $ ist die 
    \textbf{Menge der singulären Randpunkte}. \\
    $\Gamma \coloneqq \partial \Omega \cap Q \subset \partial_r \Omega $
    gemäß (1) heißt \textbf{glatter} (regulärer) \textbf{Teilrand} von $\Omega$ \\
    falls $\mathcal{L}^{n-1} (\partial \Omega') = 0 $
    für zugehörige $Q' \subset \mathbb{R}^{n-1} $.
\end{definition}    
    
Es ist zu beachten, dass $\partial_r \Omega $ lokal Graph einer
$C^1$-Funktion ist und somit eine (n-1)-dimensionale Mannigfaltigkeit nach Satz 29.1. \\
Für einen glatten Teilrand $\Gamma \subset \partial_r \Omega $ gilt:
$v_{n-1} \left(\overline{\Gamma} / \Gamma \right) = 0 $\\
($\varphi \left(x' \right) = \left(x', h\left(x' \right) \right)
$ ist eine Parametrisierung der Mannigfaltigkeit und \\
$\overline{\Gamma} / \Gamma 
$ ist Bild der (n-1)-dimensionalen Nullmenge $
\partial_r Q' $)

Man erhält eine äquivalente Formulierung zu (1) mittels \\
$\gamma(x) \coloneqq h(x') - x_n \forall (x', x_n) \in Q $ ($Q$, $h$ wie oben) durch
\begin{equation*}
    \Omega \cap Q = \lbrace x \in Q \ |\ \gamma(x) \leq 0 \rbrace
\end{equation*}
\begin{equation}
    \partial \Omega \cap Q = \lbrace x \in Q \ |\ \gamma(x) = 0 \rbrace
\end{equation}

(Dies ist eine lokale Darstellung von $\partial_r \Omega $ als Niveaufläche
und liefert mit Satz 29.4 eine Mannigfaltigkeit, beachte $\gamma'(x) \neq 0 $)

\begin{definition}[Stückweise glatter Rand]
    \mbox{} \\
    $\Omega \subset \mathbb{R}^n $ hat einen \textbf{stückweise glatten Rand}, 
    falls es glatte Teilränder $\Gamma_1, \cdots, \Gamma_m $ von $\Omega$ gibt mit
    $\partial \Omega = \bigcup\limits_{j=1}^m \overline{\Gamma}_j $
\end{definition}

Zum Beispiel haben Würfel oder Polyeder einen stückweise glatten Rand.

Gemäß Bsp. 29.10 erhält man die Einheitsnormalen auf $\partial_r \Omega $
(mit $\gamma$ wie in (2) bzw. $h$ wie in (1)): \\

\begin{equation}
    \nu(x) = 
    \frac{\gamma'(x)}{|\gamma'(x)|} = 
    \frac{(h'(x'),\ -1)}{|(h'(x'),\ -1)|} \
    \forall x \in \partial_r \Omega
\end{equation}

\begin{lemma}
$\forall x \in \partial_r \Omega \exists \delta = \delta(x) > 0$:
\begin{equation}
    x + t \nu(x) \in \mathbb{R}^n / \Omega, \ x - t \nu(x) \in \Omega
    \forall t \in (0, \delta)
\end{equation}
\end{lemma}

\begin{proof}
    Selbststudium.
\end{proof}

Man beachte, dass die Koordinaten in (1), (2) und (3) eventuell bezüglich einem
gedrehten Koordinatensystem zu verstehen sind.

Da in jedem $x \in \partial_r \Omega $ nur zwei Einheitsnormalen existieren, da
$\gamma'(.) $ stetig ist und wegen (4) liefert (3) ein ENF auf der Menge
$\partial_r \Omega $ (insbesondere ist $\nu(.) $ stetig auf $\partial_r \Omega $)

Da alle $\nu(x) $ nach 'außen' zeigen, heißt $\nu$ aus (3) \textbf{äußeres ENF} von
$\partial_r \Omega $.\\
Damit ist $\partial_r \Omega $ mit $\nu$ eine orientierte (n-1)-dimensionale
Mannigfaltigkeit.

\includegraphics[scale=0.3]{pictures/007-03}

\begin{definition}[Divergenz eines Vektorfelds]
    \mbox{} \\
    Eine Abbildung $F: \mathbb{R}^n \rightarrow \mathbb{R}^n $ heißt auch
    \textbf{Vektorfeld}.\\
    Falls $F$ in $x \in M $ differenzierbar ist, heißt \\
    $\mathrm{div\ }F(x) \coloneqq \pdiff{F_1}{x_1}(x) + \cdots + \pdiff{F_n}{x_n}(x) =
    \mathrm{tr\ }\left(F'(x)\right) $\\
    \textbf{Divergenz} des Vektorfelds $F$ in $x$.
\end{definition}

\begin{satz}[Gaußscher Integralsatz - Spezialfall für Quader]
\mbox{} \\
Sei $F: U \subset \mathbb{R}^n \rightarrow \mathbb{R}^n $ ein stetig differenzierbares VF,
$U$ sei offen, $Q \subset \mathbb{R}^n $ ein Quader, $\overline{Q} \subset U \\
$
\begin{equation}
    \Longrightarrow 
    \underbrace
        {\int\limits_Q \mathrm{div\ } F(x) \mathrm{d}x
    }_{
        \text{Lebesgue-Integral in }\mathbb{R}^n}
    =
    \underbrace{
        \int\limits_{\partial Q} F(x) \nu(x) \mathrm{d}a
    }_{
        \text{Integral auf (n-1)-dim. Fläche } \partial Q
    }    
\end{equation}

\end{satz}

Gelegentlich schreibt man auch $\int\limits_{\partial Q} F \ \mathrm{d}\vec{a} $ anstatt
der rechten Seite in (5) und bezeichnet
$\mathrm{d}\vec{a} = \nu \mathrm{d}a $ als vektorielles Flächenelement auf
$\partial Q $.

\textbf{Interpretation bei n = 1:}\\
$Q = (a,b) \subset \mathbb{R}, \ \mathrm{div\ } F(x) = F'(x) $ \\
Man kann $\partial Q = \lbrace a, b \rbrace $ 
als 0-dimensionale Mannigfaltigkeit betrachten,\\
$\nu(b) = 1, \ \nu(a) = -1 $\\
in (5) wäre dann
$\int\limits_a^b F'(x) \mathrm{d}x = F(b) - F(a) $\\
Das heißt der Hauptsatz der Differential- und Integralrechnung ist
ein Spezialfall des Gaußschen Integralsatzes!

\begin{proof}
\mbox{} \\
Sei $\nu(x) = (\nu_1(x), \cdots, \nu_n(x)) $\\
Wir zeigen für beliebige $C^1$-Funktionen $f: U \rightarrow \mathbb{R} $,
dass gilt: \\
\begin{equation}
    \int\limits_Q \pdiff{}{x_k} f(x) \mathrm{d}x =
    \int\limits_{\partial Q} f(x) \nu_k (x) \mathrm{d}a
\end{equation}
Ersetzt man hier $f$ durch $F_k$ und summiert über über $k$ so folgt (5).\\
Zeige (6) (oBdA sei $k=n$):
Sei $Q = Q' \times (a,b), \ Q' \subset \mathbb{R}^{n-1} $ Quader\\
Auf $\partial_r Q $ hat man
$\nu_n (x) = 
\begin{cases}
    1 \text{ auf } Q' \times \lbrace b \rbrace \\
    -1 \text{ auf } Q' \times \lbrace a \rbrace \\
    0 \text{ auf } \partial Q' \times (a,b)
\end{cases}
$
\includegraphics[scale=0.3]{pictures/007-04}

$\Rightarrow \int\limits_{\partial Q} \nu_n f \mathrm{d}a =
\int\limits_{Q' \times \lbrace b \rbrace} f \mathrm{d}a -
\int\limits_{Q' \times \lbrace a \rbrace} f \mathrm{d}a
$ \\
Nun parametrisiert man die Mannigfaltigkeiten 
$Q' \times \lbrace a \rbrace $ bzw. $ Q' \times \lbrace b \rbrace $ durch \\
$x' \rightarrow (x', a) $ bzw. $x' \rightarrow (x', b) $ mit $x' \in Q' $ \\
Offenbar ist die Gramsche Determinante jeweils 1. \\
$\Rightarrow 
\int\limits_{\partial Q} f \nu_n \mathrm{d}a =
\int\limits_{Q'} f(x',b) \mathrm{d}x' -
\int\limits_{Q'} f(x',a) \mathrm{d}x' \\
\stackrel{\text{HS Int'rechnung}}{=}
\int\limits_{Q'} \left(
\int\limits_a^b \pdiff{}{x_n} f(x', \psi) \mathrm{d}\psi \right) \mathrm{d}x' \\
\stackrel{\text{Fubini}}{=} \int\limits_Q \pdiff{}{x_n} f(x) \mathrm{d}x
\Rightarrow $ (6) $\Rightarrow $ Behauptung.
\end{proof}
\ \\
\linebreak
Wie können wir uns das formale Konstrukt $\div F$ 
anschaulich vorstellen? Setzen wir zum Beispiel $F$ als 
Geschwindigkeitsfeld einer strömenden Flüssigkeit an, so können 
wir  
\begin{equation*}
	\int_{\partial Q}F\cdot r \ \mathrm{d}a
\end{equation*}
als ein Maß für den Massentransport (pro Zeiteinheit) durch 
die Oberfläche eines Quaders $Q$ ansehen. Ist der (skalierte) 
Mittelwert dieses Flusses 
\begin{equation*}
	\frac{1}{v_n(Q)}\int_{\partial Q}F\cdot r \ \mathrm{d}a
\end{equation*}
positiv, so spricht man anschaulich von einem ''Zufluss'' und analog dazu, falls er negativ ist, von einem ''Abfluss''. Ist 
er gleich Null, so handelt es sich um eine ausgeglichene Bilanz.

\begin{lemma}
Sei $F:U\subset\mathbb{R}^n\rightarrow\mathbb{R}$ ein stetig 
differenzierbares Vektorfeld und $U$ offen. Dann sei weiter 
$\tilde{x}\in U$ und $\{Q_k\}$ eine Folge von Quadern, deren 
Abschluss $\overline{Q}$ Teilmenge von $U$ ist, sodass 
$\tilde{x}\in Q_k \ \forall k$. Außerdem soll $r_k>0$ als 
größte Kantenlänge von $Q_k$ für $k\rightarrow\infty$ gegen Null 
gehen. Dann gilt
\begin{equation}
	\lim\limits_{k\rightarrow\infty}\frac{1}{v_n(Q)}\int_{\partial Q} F\cdot r\ \mathrm{d}a = 
	\mathrm{div\ }F(\tilde{x})
\end{equation} 
\end{lemma}

\begin{proof}
Da alle $\overline{Q}_k$ kompakt sind und $\mathrm{div\ }F$ stetig ist, existiert ein $a_k\coloneqq\min\limits_{x\in\overline{Q}_k}\mathrm{div\ }F(x)$ und ein $b_k\coloneqq\max\limits_{x\in\overline{Q}_k}\mathrm{div\ }F(x)$, so dass 
wir nach Satz 31.2 folgern können
\begin{equation*}
	a_k\cdot v_n(Q)\leq\int_{Q_k}\mathrm{div\ }F(x)\mathrm{d}x 	= \int_{\partial Q_k}F\cdot r \ \mathrm{d}a \leq 
	b_k\cdot v_n(Q_k)
 \end{equation*}
Aus 
\begin{equation*}
	\lim\limits_{k\rightarrow\infty} a_k = 
	\lim\limits_{k\rightarrow\infty} b_k = 
	\mathrm{div\ }F(\tilde{x})
\end{equation*}
folgt dann die Behauptung.
\end{proof}
\ \\
\linebreak
Somit nennt man einen Punkt $x$ \textbf{Quelle} von $F$, falls 
$\mathrm{div\ }F(x)>0$ und \textbf{Senke}, falls $\mathrm{div\ }F(x)<0$. Die Divergenz $\mathrm{div\ }F$ können wir dann 
schließlich auch als Quellendichte von $F$ bezeichnen. (32.5) 
besagt somit, dass die Summe der in $Q$ erzeugten 
beziehungsweise vernichteten Flüssigkeit durch den Rand des 
Quaders ab- oder zufließen muss. Dieser Gedanke für auf viele 
grundlegende Bilanzgleichungen der Physik. 

\begin{theorem}[Gaußscher Integralsatz]
Sei $\Omega\subset\mathbb{R}^n$ offen, beschränkt und habe 
einen stückweise glatten Rand. Außerdem sei $F:\overline{\Omega} 
\rightarrow\mathbb{R}^n$ stetig und differenzierbar auf $\Omega$ und sei $\mathrm{div\ }F$ integrierbar auf $\Omega$. 
Dann gilt
\begin{equation}
	\int_\Omega\mathrm{div\ }F(x)\mathrm{d}x = 
	\int_{\partial\Omega}F(x)\cdot\nu(x)\ \mathrm{d}a
\end{equation}
\end{theorem}
\ \\
Hierbei ist zu bemerken, dass $F:U\rightarrow\mathbb{R}^n$, 
falls es stetig differenzierbar auf der offenen Umgebung $U$ 
von $\overline{\Omega}$ ist, schon alle Bedingungen für 
Theorem 32.4 erfüllt. \\
Außerdem bleibt anzumerken, dass der Gaußsche Integralsatz 
auch richtig bleibt, wenn $\Omega$ einen Lipschitz-Rand hat, 
der sich als Graph einer Lipschitz-stetigen Funktion 
darstellen lässt.\\
\linebreak

\emph{Beweisstrategie:} Man zeigt die Behauptung zunächst 
für ein $F$ mit kompaktem Träger.\\
\linebreak

Gelegentlich wird der Gaußsche Satz auch in folgender Form definiert:

\begin{satz}[Variante des Gaußschen Satzes]
Sei $\Omega\subset\mathbb{R}^n$ offen, beschränkt und habe 
einen stückweise glatten Rand. Außerdem sei $f:\overline{\Omega}\rightarrow\mathbb{R}$ als skalare 
Funktion stetig und differenzierbar auf $\Omega$ und 
$\pdiff{}{x_n}f$ integrierbar auf $\Omega$. 
Dann gilt
\begin{equation}
	\int_\Omega\pdiff{}{x_k}f(x)\mathrm{d}x = 
	\int_{\partial\Omega}f(x)\cdot\nu_k(x)\ \mathrm{d}a
\end{equation}
wobei $\nu_k$ die k-te Komponente des Einheitsnormalenfeldes 
$\nu$ ist.
\end{satz}

\begin{proof}
Wir definieren das Vektorfeld $F(x)\coloneqq 
(0,0,. . .,f_k(x),. . .,0)$, welches alle Voraussetzungen 
für Theorem 32.4 erfüllt. Damit impliziert (32.8) gerade 
(32.9).
\end{proof}
\ \\
Es fällt auf, dass wir auch Satz 32.5 als eigentlich Satz 
von Gauß hätten formulieren können, denn mit $F = 
(F^1,...,F^n)$ erfüllen alle $f=F^k$ die Voraussetzungen 
für Satz 32.5. Dann summieren wir einfach alle $\int_\Omega F_{x_k}^k\mathrm{d}x$ auf und erhalten (32.8).

\begin{theorem}[Partielle Integration]
Sei $\Omega\subset\mathbb{R}^n$ offen, beschränkt und mit 
stückweise glattem Rand, dann gilt
\begin{enumerate}
	\item 	Sind $f,g:\overline{\Omega}\rightarrow\mathbb{R}$ 
			stetig und stetig differenzierbar auf 
			$\Omega$ und $f_{x_k},g_{x_k}$ integrierbar auf 
			$\Omega$, dann 
			\begin{equation}
				\int_\Omega f_{x_k}(x)g(x)\mathrm{d}x = 
				- \int_\Omega f(x)g_{x_k}(x)\mathrm{d}x + 
				\int_{\partial\Omega}f(x)g(x)\nu_k(x)\mathrm{d}a
			\end{equation}
	\item Sind $F:\overline{\Omega}\rightarrow\mathbb{R}^n$ 
			und $g:\overline{\Omega}\rightarrow\mathbb{R}$ 
			jeweils stetig und stetig differenzierbar auf 
			$\Omega$, sowie div $F$ und $g'$ integrierbar 
			auf $\Omega$, dann
			\begin{equation}
				\int_\Omega F(x)g'(x)\mathrm{d}x = 
				-\int_\Omega g(x)\mathrm{div\ }F(x)\mathrm{d}x + 
				\int_{\partial\Omega}g(x)F(x)\nu(x)\mathrm{d}a
			\end{equation}
\end{enumerate}
\end{theorem}

\begin{proof}
Für 1. wenden wir (32.8) auf $f\cdot g$ an und erhalten die 
Behauptung. Für 2. wenden wir (32.8) auf das Vektorfeld 
$g\cdot F$ an. Dabei ist zu beachten, dass
\begin{equation*}
	\mathrm{div\ }(gF)=g\cdot\mathrm{div\ }F + F\cdot g'
\end{equation*}
\end{proof}
\ \\
Gerade in der Potentialtheorie ($\Delta\varphi=f$) sind folgende 
Formeln wichtig:

\begin{satz}[Greensche Formeln]
Sei $\Omega\subset\mathbb{R}^n$ offen, beschränkt und mit 
stückweise glattem Rand. Außerdem seien $f,g:\overline{\Omega} 
\rightarrow\mathbb{R}$ stetig und in $C^2$ auf $\Omega$, 
dann gilt, falls die Integrale existieren, folgendes:
\begin{equation}
	\int_\Omega f'(x)\cdot g'(x)\mathrm{d}x = 
	-\int_\Omega f(x)\Delta g(x)\mathrm{d}x + 
	\int_{\partial\Omega}f(x)g'(x)\nu(x)\mathrm{d}a
\end{equation}
\begin{equation}
	\int_{\Omega}f(x)\Delta g(x)-g(x)\Delta f(x)\mathrm{d}x = 
	\int_{\partial\Omega}f(x)g'(x)\nu(x) - 
	g(x)f'(x)\nu(x)\mathrm{d}a 
\end{equation}
wobei $\Delta f(x)=\mathrm{div\ }f'(x)$.
\end{satz}

\begin{proof}
Wir verwenden (32.11) mit ($g',f$) statt ($F,G$) und kommen 
damit auf (32.12) mit ($f,g$) und ($g,f$) und erhalten damit
(32.13).
\end{proof}

\begin{beispiel}
Sei $\Omega\subset\mathrm{R}^n$ wie in Theorem 32.4 und $F(x) = 
x$. Offenbar ist $\mathrm{div\ }F(x)=n$ auf $\Omega$, daraus 
folgt mit (32.8)
\begin{equation}
	nv_n(\Omega)=\int_{\partial\Omega}x\cdot\nu(x)\mathrm{d}a
\end{equation}
\begin{enumerate}
	\item  $\Omega=B_1(0)$. Hier ist $x\cdot\nu(x)=1$ auf
			$\partial\Omega$ und nach (32.14) gilt dann
			\begin{equation*}
				n\kappa_n=\omega_n
			\end{equation*}
			\begin{center}
				\includegraphics[scale=0.5]{pictures/008-01.png}				\end{center}			 
\newpage
	\item $\Omega\subset\mathbb{R}^2$. Hier sei der Rand eine 
			$C^2$-Kurve $t\rightarrow(x(t),y(t))$ dann ist
			\begin{equation*}
				\nu(x(t),y(t)) =
				\frac{(y'(t),-x'(t)}{|(y'(t),-x'(t)|}
			\end{equation*}
			Daraus folgt mit (32.14)
			\begin{equation*}
				v_2(\Omega) = 
				\frac{1}{2}\int\limits_a^b\frac{xy'-yx'}
				{\sqrt{x'^2+y'^2}}\sqrt{x'^2+y'^2}\mathrm{d}t = 
				\frac{1}{2}\int\limits_a^bx(t)y'(t)-y(t)x'(t) \ 
				\mathrm{d}t
			\end{equation*}
			\begin{center}
				\includegraphics[scale=0.5]{pictures/008-02.png}\end{center}	
\end{enumerate}
\end{beispiel}

Nach dem Satz von Gauß kommen wir nun zu einem zweiten wichtigen Integralsatz, dem Satz von Stokes. Dazu klären wir zunächst ein 
paar Begriffe.
\begin{definition}[Rotation]
Sei $F:\Omega\subset\mathbb{R}^3\rightarrow\mathbb{R}^n$ ein 
$C^1$-Vektorfeld auf dem offenen $\Omega$ mit $F=(F^1,F^2,F^3)$. 
Das Vektorfeld $\rot F:\Omega\rightarrow\mathbb{R}^n$ 
gegeben durch
\begin{equation}
	\rot F(x)\coloneqq 
	\begin{pmatrix}
	\pdiff{}{x_2}F^3(x)-\pdiff{}{x_3}F^2(x) \\
	\pdiff{}{x_3}F^1(x)-\pdiff{}{x_1}F^3(x) \\
	\pdiff{}{x_1}F^2(x)-\pdiff{}{x_2}F^1(x)
	\end{pmatrix}
\end{equation}
heißt \textbf{Rotation} oder auch Zirkulation von $F$. Sie 
lässt sich auch sehr effektiv durch den \textbf{Nabla-Operator} 
gewinnen. 
\begin{equation*}
	\nabla\coloneqq e_1\pdiff{}{x_1}+e_2\pdiff{}{x_2} + 
	e_3\pdiff{}{x_3}
\end{equation*}
\begin{equation*}
	\rot F=\nabla\times F=\det\begin{pmatrix}
	e_1 & e_2 & e_3 \\
	\pdiff{}{x_1} & \pdiff{}{x_2} & \pdiff{}{x_3} \\
	F^1 & F^2 & F^3  
	\end{pmatrix}
	\tag{32.15'}
\end{equation*}
\end{definition}

\begin{beispiel}
Betrachten wir das Vektorfeld 
\begin{equation*}
	F(x)\coloneqq \begin{pmatrix}
	0 \\ 0 \\ \alpha \end{pmatrix} \times
	\begin{pmatrix}
	x_1 \\ x_2 \\ x_3 \end{pmatrix} = 
	\begin{pmatrix}
	-\alpha x_2 \\ \alpha x_1 \\ 0 \end{pmatrix} 
\end{equation*}
Aufgrund des Kreuzproduktes steht $F(x)$ immer senkrecht auf $x$ . Die Rotation ist
\begin{equation*}
	\rot F(x) = \begin{pmatrix}
	0 \\ 0 \\ 2\alpha
	\end{pmatrix}
\end{equation*}
Wir können das ganze so interpretieren, dass $F(x)$, falls die Rotation nicht verschwindet, nahe $x$ um die Achse 
\begin{equation*}
	\left\{ x + t\cdot\rot F(x)\middle| 
	t\in\mathbb{R}\right\}
\end{equation*}
\emph{rotiert}, wobei $|\rot F(x)|$ ein Maß für die 
\emph{Rotationsgeschwindigkeit} ist. Ist diese ungleich Null, so sagt man, dass $F$ einen Wirbel in $x$ hat.
\end{beispiel}

\begin{satz}[Rechenregeln]
Seien $F,G:\Omega\subset\mathbb{R}^3\rightarrow\mathbb{R}^3$ 
beides $C^1$-Vektorfelder auf dem offenen $\Omega$ und 
$g:\Omega\rightarrow\mathbb{R}$ ebenfalls $C^1$. Dann gilt
\begin{enumerate}
	\item 	Linearität
			\begin{equation*}
				\rot (\alpha F + \beta G) = 
				\alpha\rot F + \beta\rot G
			\end{equation*}
	\item 	Produktregel
			\begin{equation*}
				\rot (gF)=g\rot F+g'\times F
			\end{equation*}
	\item 	Gradienten sind wirbelfrei.
			\begin{equation*}
				\rot g' = 
				\rot \grad g=0
			\end{equation*}
	\item 	Rotationsfelder sind quellenfrei.
			\begin{equation*}
				\div \rot  F = 0
			\end{equation*}
\end{enumerate}
\end{satz}
\begin{proof}
Lässt sich mit (32.15) oder (32.15') leicht nachrechnen.
\end{proof}
Weiterhin ist anzumerken:\\

\begin{enumerate}
    \item
    $\rot F = \rot (F + \tilde{F}) $ falls $\tilde{F} = \div g $ 
    (d.h. $\tilde{F}$ ist Gradientenfeld).
    \item
    $\div F = \div (F + \tilde{F}) $ falls $\tilde{F} = \rot G $
    (d.h. $\tilde{F}$ ist Rotation eines Vektorfeldes).
\end{enumerate}

\begin{definition}[Kohärente Orientierung]
\mbox{}\\
Sei $M \subset \mathbb{R}^3 $ eine 2-dimensionale Mannigfaltigkeit mit dem ENF $\nu $
und sei $W \subset M $ offen bezüglich $M$, 
$\partial_M W $ bezeichne den Rand von $W$ bezüglich $M$.

\textbf{Skizze fehlt}

Ein Randpunkt $u \in \partial_M W $ heißt \textbf{regulär}, falls für ein Kartengebiet $U$
und eine zugehörige Parametrisierung $\varphi$ der Punkt $x = \varphi^{-1} (u) $
ein regulärer Randpunkt von $\varphi^{-1} (U \cap W) \subset \mathbb{R}^2 $ ist.\\
(Da ein Kartenwechsel ein Diffeomorphismus ist, ist diese Definition unabhängig vom
speziellen $\varphi$.)

$W \subset M $ hat einen \textbf{glatten Rand} $\partial_M W $ bezüglich $M$
falls alle  $u \in \partial_M W $ regulär sind.\\
In diesem Fall ist $\partial_M W $ eine 1-dimensionale Mannigfaltigkeit.\\
(Denn $\varphi$ auf $\varphi^{-1} (\partial_M W \cap U) $ 
ist die zugehörige Parametrisierung\\
und $\varphi^{-1} (\partial_M W \cap U) $ ist eine 1-dimensionale Mannigfaltigkeit.)

Somit ist $\partial_M W $ lokal als reguläre Kurve darstellbar und eine Einheitstangente
$t(u) $ exisitiert.\\
$t: \partial_M W \rightarrow \mathbb{R}^3 $ 
\textbf{orientiert} $\partial_M W $ \textbf{kohärent} zu $M$, 
falls $t(u) $ der Tangenteneinheitsvektor an $\partial_M W \ \forall u $ und
eine stetige Abbildung ist und 
$\nu(u) \times t(u) \in T_u M $ für alle $u$ 'zur Menge $W$ zeigt'.
(Man sagt auch, $W$ liegt 'links vom Rand'.)

\textbf{Skizze fehlt}

\end{definition}

\begin{satz}[Integralsatz von Stokes - klassisch im $\mathbb{R}^3 $]
\mbox{}\\
Sei $F: \Omega \subset \mathbb{R}^3 \rightarrow \mathbb{R}^3 $,
$\Omega$ offen, ein $C^1$-Vektorfeld;
sei $M \subset \Omega $ eine 2-dimensionale Mannigfaltigkeit, orientiert,
mit dem ENF $\nu$ und sei $W \subset M $ beschränkt,
mit glattem Rand $\partial_M W $ welcher mit $t$ zu $M$ kohärent orientiert ist.

\begin{equation}
    \Longrightarrow
    \int\limits_W \rot F(u) \cdot \nu(u) \mathrm{d}a
    =
    \int\limits_{\partial_M W} F(u) \cdot t(u) \mathrm{d}a
\end{equation}

\end{satz}

Das heißt das Integral über die Wirbel des Vektorfelds $F$ 'in der Fläche' $W$\\
(d.h.  $\nu \cdot \rot F $)
ist gleich der Zirkulation von $F$ entlang des Randes $\partial_M W $.

\begin{proof}

$W$ möge im Kartengebiet $U$ von $M$ liegen (sonst ZdE nötig),
die zugehörige Parametrisierung sei $\varphi: V \subset \mathbb{R}^2 \rightarrow U $,
die Koordinate $x = (x_1, x_2) $ liegt in $V$, 
$u = (u_1, u_1, u_3) $ in $M$ bzw. $\Omega$.
$G$ sei $\varphi^{-1} (W) \subset V $, offen und beschränkt mit glattem Rand $\partial G $.

Man will nun (16) auf den Gaußschen Satz in $G \subset \mathbb{R}^2 $ zurückführen.

Zunächst hat man:

$\nu(\varphi(x)) = 
\frac
    {\pdiff{\varphi}{x_1} (x) \times \pdiff{\varphi}{x_2} (x)}
    {\left| \pdiff{\varphi}{x_1} (x) \times \pdiff{\varphi}{x_2} (x) \right|}
$
(Vergleiche Bsp 29.12: $a \wedge b \stackrel{\mathbb{R}^3}{=} a \times b $) und

$\sqrt{\mathrm{det } \varphi'(x)^T \varphi'(x)}
=
\left| \pdiff{\varphi}{x_1} (x) \times \pdiff{\varphi}{x_2} (x) \right| $
(vgl. 30.1, 30.2)

Als Integral auf der Mannigfaltigkeit $W$ ist somit die linke Seite in (16):

$\int\limits_W \rot F(u) \cdot \nu(u) \mathrm{d}a \\
\stackrel{\text{Def.}}{=}
\int\limits_g \mathrm{rot_u } \ F(\varphi(x)) 
\frac
    {\pdiff{\varphi}{x_1} (x) \times \pdiff{\varphi}{x_2} (x)}
    {\left| \pdiff{\varphi}{x_1} (x) \times \pdiff{\varphi}{x_2} (x) \right|}
\sqrt{\mathrm{det } \varphi'(x)^T \varphi'(x)}
\mathrm{d}x \\
=
\int\limits_G \mathrm{rot_u } \ F(\varphi(x)) \cdot 
\pdiff{\varphi}{x_1} (x) \times \pdiff{\varphi}{x_2} (x)
\mathrm{d}x $

\leftskip=30pt Wir nutzen im Folgenden zwecks Kompaktheit folgende Notation:\\
$F_l^k \coloneqq \pdiff{F_k}{u_l}$,
$\varphi_l^k \coloneqq \pdiff{\varphi_k}{x_l} $ wobei 
$(\varphi = (\varphi_1, \varphi_2, \varphi_3)) $\\

\leftskip=0pt
$
=
\int\limits_G
\begin{pmatrix}
    F_2^3 - F_3^2 \\
    F_3^1 - F_1^3 \\
    F_1^2 - F_2^1 \\
\end{pmatrix}
\cdot
\begin{pmatrix}
    \varphi_1^2 \varphi_2^3 - \varphi_1^3 \varphi_2^2\\
    \varphi_1^3 \varphi_2^1 - \varphi_1^1 \varphi_2^2\\
    \varphi_1^1 \varphi_2^2 - \varphi_1^2 \varphi_2^1\\
\end{pmatrix}
$

Schreibe im Weiteren nur die Terme mit $F^1$:

\begin{equation*}
    =
    \int\limits_G F_1^1 \cdot 0
    +
    F_2^1 (\varphi_1^2 \varphi_2^1 - \varphi_1^1 \varphi_2^2)
    +
    F_3^1 (\varphi_1^3 \varphi_2^1 - \varphi_1^1 \varphi_2^3)
    +
    \ldots \mathrm{d}x
    \tag{$\heartsuit$}
\end{equation*}

Für die rechte Seite in (16) sei

$x \rightarrow \tilde{x}(s) = (\tilde{x}_1 (s), \tilde{x}_2 (s)) $
die Parametrisierung der 1-dim. Mf $\partial G $ mit
$s \in I \subset \mathbb{R} \\
\Rightarrow s \rightarrow \psi(s) \coloneqq \varphi(\tilde{x}(s)) $
ist die Parametrisierung der 1-dim. Mf $\partial_M W $
und \\
$t(\psi(s)) = \frac{\psi'(s)}{|\psi'(s)|} $;
$\psi'(s) = \varphi`(\tilde{x}(s)) \cdot \tilde{x}(s) \\
\Rightarrow
    \int\limits_{\partial_M W} F \cdot t \mathrm{d}a
\stackrel{\text{Def.}}{=}
    \int\limits_I F(\psi(s)) \cdot t(\psi(s))
    \underbrace{
        \sqrt{\det \varphi'(s)^T \varphi'(s)}
        }_{
        |\psi'(s)|
        }
    \mathrm{d}s \\
=
    \int\limits_I F(\varphi(\tilde{x}(s))) \cdot
    \left(
        \varphi'(\tilde{x}(s)) 
        \underbrace{
            \frac{
                \tilde{x}'(s)
                }{
                |\tilde{x}'(s)|
                }
            }_{
            \coloneqq \tilde{t}(\tilde{x}(s))
            }
    \right)
    |\tilde{x}'(s)| \mathrm{d}s
$

\textbf{Skizze fehlt}    

$
=
    \int\limits_{\partial G} F (\varphi(x)) \cdot
    \left(
        \varphi'(x) \cdot \tilde{t}(x)) \mathrm{d}s
    \right)
$

\hspace{30pt}
    $
    \tilde{\nu}(\tilde{x}(s))
    =
    \frac{1}{
    | \tilde{x}(s) |}
    \begin{pmatrix}
        \tilde{x}_2'(s) \\
        \tilde{x}_1'(s) \\
    \end{pmatrix}
    $
    ist die äußere Einheitsnormale in $\tilde{x}(s) \in \partial G $ an $G$

$
=
    \int\limits_{\partial G} F^1(\varphi(x))
    \left(
        \varphi_1^1 \tilde{x}_1' + \varphi_2^1 \tilde{x}_2'
    \right)
    \frac{1}{| \tilde{x}' |}
    + \ldots \mathrm{d}a \\
=
    \int\limits_G F^1(\varphi(x))
    \begin{pmatrix}
        \varphi_2^1(x) \\
        -\varphi_1^1(x) \\
    \end{pmatrix}
    \tilde{\nu}
    + \ldots \mathrm{d}a \\
\stackrel{\textbf{Gauß}}{=}
    \int\limits_G \mathrm{div_x } \ F^1(\varphi(x))
    \begin{pmatrix}
        \varphi_2^1(x) \\
        -\varphi_1^1(x) \\
    \end{pmatrix}
    + \ldots \mathrm{d}x \\
=
    \int\limits_G 
    \mathrm{div_u } \ F^1 \pdiff{\varphi}{x_1} \varphi_2^1(x)
    + \underbrace{F_1 \varphi_{21}^1}_{\text{Schwarzscher Satz}}
    - \mathrm{div_u } \ F^1 \pdiff{\varphi}{x_2} \varphi_1^1
    - \underbrace{F^1 \varphi_{12}^1}_{\sum = 0}
    + \ldots \mathrm{d}x \\
=
    \int\limits_G F_1^1 
    \underbrace{\left( \varphi_1^1 \varphi_2^1 - \varphi_2^1 \varphi_1^1 \right)}_{=0}
    + F_2^1 \left( \varphi_1^2 \varphi_2^1 - \varphi_2^2 \varphi_1^1 \right)
    + F_3^1 \left( \varphi_1^3 \varphi_2^1 - \varphi_2^3 \varphi_1^1 \right)
    + \ldots \mathrm{d}x
$

Vergleich mit($\heartsuit$) liefert die Behauptung aus (16).

\end{proof}

\textbf{Hauptsatz der Vektoranalysis}\\
Falls für ein unbekanntes Vektorfeld
$F: \Omega \subset \mathbb{R}^3 \rightarrow \mathbb{R}^3 $
die Quellen, die Wirbel und der Fluss durch den Rand bekannt sind, so ist $F$ eindeutig
bestimmt, d.h. für gegebene Funktionen\\
$f: \Omega \rightarrow \mathbb{R}, \ G: \Omega \rightarrow \mathbb{R}^3, \ 
\varphi: \partial \Omega \rightarrow \mathbb{R} $ gelte \\
$\div F = f, \ \rot F = G $ auf $\Omega, \ F \cdot \nu = \varphi $ auf $\partial \Omega $
und die Kompatibilitätsbedingung: $\div G = 0 $ auf $\Omega $ und 
$
\int\limits_{\Omega} f \mathrm{d}x 
= 
\int\limits_{\partial \Omega} \varphi \mathrm{d}a \\
\Rightarrow F $ ist eindeutig bestimmt.\\
(Falls $\Omega,\ f, \ G, \ \varphi $ hinreichend regulär sind.)\\
Wichtige Anwendungen dessen finden sich z.B. in der Elektrodynamik.


\section{Gradientenfelder}

\begin{definition}[Gradientenfeld]
\mbox{}\\
Eine Abbildung $F: \Omega \subset \mathbb{R}^n \rightarrow \mathbb{R}^n $,
$\Omega $ offen, heißt \textbf{Gradientenfeld},
falls eine differenzierbare Funktion $f: \Omega \rightarrow \mathbb{R} $ exisitiert,
die $F(x) = f'(x) \forall x \in \mathbb{R} $ erfüllt.
\end{definition}

Wir wollen untersuchen, welche Vektorfelder Gradientenfelder sind, analog zur Suche
nach Stammfunktionen in Kapitel 25.

\begin{satz}[Notwendige Bedingung]

Sei $F = (F_1, \ldots, F_n): \Omega \subset \mathbb{R}^n \rightarrow \mathbb{R}^n$,
$\Omega $ offen, stetig differenzierbar und ein Gradientenfeld
\begin{equation}
    \Longrightarrow
    \pdiff{}{x_j} F_i (x) = \pdiff{}{x_i} F_j (x) \ \forall x \in \mathbb{R}^n; \
    i,j = 1, \ldots, n
\end{equation}
(1) heißt \emph{Integrabilitätsbedingung}; $\pdiff{F_i}{x_j} = \pdiff{f}{x_i x_j} $\\
Für $n=3 $ gilt: (1) $ \Leftrightarrow \rot F = 0 $
\end{satz}



\begin{beispiel}[Logistisches Wachstum (gebremst)]
\mbox{}\\
$u(t) $ sei Größe einer Population.
Berücksichtige hemmende Faktoren (beschränkte Resourcen, Krankheiten, Kriege,...)

\textbf{Modellannahme:} wegen beschränkter Kapazität kann $u(t) $
eine gewisse Maximalgröße $M$ nicht überschrieten und
$\Delta u $ ist proportional zu $u, \ M-u, \ \Delta t \\
\Rightarrow \Delta u = \alpha u(M-u) \Delta t 
\xRightarrow{\text{wie oben}} 
u' = \gamma u - \tau u^2 (\gamma = \alpha M, \ \tau = \alpha $\\
Interpretation: Wachstum $u$ wird durch Term $-\tau u^2 $ für große $u$ 
stärker gebremst als für kleine $u$. \\

Allgemeine Lösung: für $u(0) = u_0 \\
u(t) = \frac{\gamma}{\tau + \left(\frac{\gamma}{u_0} - \tau \right) e^{-\gamma t}}
= \frac{M}{1+\left(\frac{M}{u_0} - 1 \right) e^{-\gamma M t}} $ \\
Sei  $\alpha, M > 0 \Rightarrow u(t) \xrightarrow{t\rightarrow\infty} M $\\
\includegraphics[scale=0.5]{pictures/011-01.png}\\                            
$u_0 \in (0,M)\\
u_0 >M\\
\includegraphics[scale=0.5]{pictures/011-02.png}\\
u_0 = M \Rightarrow u(t) = M \forall t $\\

Insbesondere beschreibt logistisches Wachstum:
\begin{itemize}
    \item Gewichtszunahme
    \item Höhenwachstum von Sonnenblumen
    \item Verbreitung von Gerüchten
\end{itemize}

\end{beispiel}

\begin{beispiel}[Freier Fall]

$v(t) = u'(t) $ Geschwindigkeit\\

\textbf{Modellannahme:} Newtonsches Kraftgesetz: $K = m u'' $\\
\includegraphics[scale=0.5]{pictures/011-03.png}\\                            
Schwerkraft nahe Erdoberfläche $K = mg$ ($g$ - 'Gravitationskonstante')\\
$\Rightarrow u'' = g $ bzw. $v' = g \\
\Rightarrow v(t) = v_0 + gt \\
\Rightarrow u(t) = u_0 + v_0 t + \frac{1}{2} g t^2 $\\
Offenbar liefert Vorgabe von $u(0) = u_0 $ und $u'(0) = v_0 $ eine eindeutige Lösung
der Differenzialgleichung (Anfangswertproblem).\\
Alternativ könnte man $u(0) = u_0 $ und $u(t_1) = u_1 $ vorschreiben:\\
$\Rightarrow u_1 = u_0 + v_0 t_1 + \frac{1}{2} g t_1^2 \\
\Rightarrow v_0 = \frac{u_1 - u_0 - \frac{1}{2} g t_1^2}{t_1} $\\
das heißt man erhält wieder eine eindeutige Lösung (Randwertproblem).
\end{beispiel}

\textbf{Wichtige Fragestellungen bei Behandlung von Differenzialgleichungen}\\
\begin{itemize}
    \item \emph{Existenz einer Lösung}
        \begin{itemize}
            \item explizite Lösungen findet man nur in einigen Spezialfällen\\
                ($\rightarrow $ Näherungslösung mittels Computer)
            \item 
                \emph{aber} Existenz einer Lösung kann 
                sehr allgemein abstrakt gezeigt werden (zumindest lokal)
        \end{itemize}
        $\Rightarrow $\emph{qualitative Untersuchungen} 
        der nicht explizit bekannten Lösungen spielen eine wichtige Rolle,
        zum Beispiel:
        \begin{itemize}
            \item Fortsetzunge
            \item Asymptotisches Verhalten
            \item Stabilität
            \item Regularität
            \item Periodizität
            \item ...
        \end{itemize}
    \item \emph{Eindeutigkeit einer Lösung}
        \begin{itemize}
            \item obige Beispiele zeigen, dass idR Parameter auftreten,
                 woraus unendlich viele Lösungen folgen
            \item durch Vorgabe von geeigneten Anfangswerten $(u(0), \ u'(0), \ldots)$
                bzw. von geeigneten  Randwerten $(u(t_0) = u_0, (u(t_1) = u_1) $ 
                ergibt sich häufig eine  eindeutige Lösung
            \item manche Probleme haben in natürlicher Weise keine eindeutige Lösung,
                z.B. Beulprobleme
        \end{itemize}
    \item \emph{Stetige Abhängigkeit der Lösung von Parametern} \\
        Parameter = Anfangswerte, Koeffizienten\\
        Problem: Parameter sind nie exakt messbar! Geringfügige Abweichungen können
        in (chaotischen) Systeme zu völlig unterschiedlichen Lösungen führen 
        (z.B. Doppelpendel). \\
        'Das ist eine ganz wichtige Sache, muss man wissen!'\\
        $\Rightarrow $ kleine Störung der Parameter sollten nur 
        kleine Veränderung der Lösung bewirken, d.h. die Lösung sollte stetig
        von den Parametern abhängen.
\end{itemize}

\begin{definition}[Korrekte Problemstellung]
Man sagt ein Problem ist \textbf{korrekt gestellt} falls die Lösung:
\begin{itemize}
    \item existiert,
    \item eindeutig ist,
    \item stetig von den Parametern abhängt.
\end{itemize}
\end{definition}

Literatur:\\
Walter: Gewöhnliche Differenzialgleichungen, Springer\\
Heuser: Gewöhnliche Differenzialgleichungen

\section{Differentialgleichungen 1. Ordnung}

Allgemeine Form: $f(x, u(x), u'(x)) = 0 $

\subsection{Explizite Dgl. 1. Ordnung - Elementar integrierbare Fälle}

Allgemeine explizite Dgl. 1. Ordnung:
$u'8x) = f(x, u(x)) $

Annahme: $f: D \subset \mathbb{R}^2 \rightarrow \mathbb{R} $ stetig
($\mathbb{R}^2 \leftrightarrow (x,u)$-Ebene)

Lösungsbegriff:
Sei $I \subset \mathbb{R} $ Intervall\\
Funktion $u: I \subset \mathbb{R} \rightarrow R $ ist Lösung der Dgl. falls
$u$ auf $I$ diffferenzierbar,\\
$(x, u(x)) \in D \ \forall x \in I $ und $u'(x) = f(x, u(x)) $ ist.

\begin{enumerate}
    \item[i)] \emph{Vorbemerkung: Richtungsfeld, Polygonzug:} $u'(x) = f(x, u(x))$\\
        Sei $u$ Lösung mit $(\tilde{x}, u(\tilde{x})) = (\tilde{x}, \tilde{u}) \in D\\
        \Rightarrow f((\tilde{x}, \tilde{u}) $ gibt Anstieg der Kurve $u(.) $ in $x$\\
        \includegraphics[scale=0.5]{pictures/011-04.png}\\
        $(x, u, f(x,u)) $ heißt \emph{Richtungsfeld}
        ohne(!) Kenntnis der Umgebung gibt es Anstieg von $u(.)$ in $x$ 
        für den Fall, dass $(x,u) $ zum Graphen von $u(.) $ gehört.\\
        \includegraphics[scale=0.5]{pictures/011-05.png}\\
        \emph{Problem:} suche Kurve $u(.) $ die zum Richtungsfeld passt.\\
        \emph{Anfangsgswert:} In 'vielen Fällen' geht durch jeden Punkt $(x,u) \in D $
        genau eine Kurve \\
        $\Rightarrow $ Vorgabe von $(x_0, u_0) = (x_0, u(x_0)) $ 
        liefert eindeutige Lösung
        
        \emph{Näherungslösung:} Polygonzug\\
        wähle $x_k = x_0 + k h, \ k= 1 \ldots n, \ h$ - Schrittweite\\
        $u_0 = u(x_0) $wird als AW vorgegeben.\\
        \includegraphics[scale=0.5]{pictures/011-06.png}\\                    
        Schrittweise setzt man $u_k = u_{k+1} + h f(x_{k-1}, u_{k-1}), \ k= 1 \ldots n $\\
        in 'vielen Fällen' konvergiert Polygonzug für $h \rightarrow 0 $ gegen Lösung.
    \item[ii)]$\mathbf{u'(x) = f(x)}$
        $\Rightarrow  u $ ist Stammfunktion von $f$
        Sei $f$ im Intervall $I \subset \mathbb{R} $ definiert und stetig, $x_0 \in I $\\
        Allgemeine Lösung: (vgl. Grundkurs)\\
        $u(x) = \int\limits_{x_0}^x f(\xi) \mathrm{d}\xi + u_0, \ u_0 \in \mathbb{R} $\\
        Vorgabe $u_0$ entspricht gerade dem AW $u(x_0) = u_0 $
    \item[iii)]$\mathbf{u'(x) = f(x) g(u(x))} $ Dgl. mit getrennten Variablen\\
        \emph{Heuristik:}
        $\diff{u}{x} = f(x) g(u) 
        \Rightarrow  \frac{1}{g(u)} \mathrm{d}u = f(x) \mathrm{d}x $\\
        Integration: $\int \frac{1}{g(u)} \mathrm{d}u = \int f(x) \mathrm{d}x $\\
        um AWP $u(x_0) = u_0 $ zu lösen nehme \\
        $\int\limits_{u_0}^u \frac{1}{g(s)} \mathrm{d}s 
        = \int\limits_{x_0}^x f(x) \mathrm{d}x $\\
        Auflösung nach $u$ liefert Lösung $u = u(x)$
        \begin{beispiel*}
        $u' = e^u \sin x, \ u(x_0) = u_0 \\
        \Rightarrow \int\limits_{u_0}^u e^{-s} \mathrm{d}s
        = \int\limits_{x_0}^x \sin \mathrm{d}t
        \Rightarrow \left[ -e^{-s} \right]_{u_0}^u = \left[ - \cos t \right]_{x_0}^x \\
        \Rightarrow e^{-u} = \cos x - \cos x_0 + e^{-u_0}
        \Rightarrow u(x) = -\ln ( \cos x 
        + \underbrace{e^{-u_0} - \cos x_0}_{\coloneqq c})\\
        $        
        \includegraphics[scale=0.5]{pictures/011-07.png}\\                            
        Es ist zu beachten, das sich abhängig von den AW der Definitionsbereich der
        Lösung ändert.\\
        Probe: $u' = 
        \frac{
        \sin x}
        {\cos x + e^{-u_0} - \cos x_0}
        = e^u \sin x
        $
        \end{beispiel*}        
\end{enumerate}


\end{document}